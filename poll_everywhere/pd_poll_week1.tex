\documentclass[poll_tutorial_format]{subfiles}
\begin{document}

\section{PD Week 1}

\begin{exercise}
Let $F(x) = e^{-x}$ for $x\in \R$? Can $F$ be a CDF?

Choose one of these answers:
\begin{enumerate}
\item Yes, but we still need to compute the normalization constant.
\item Yes, because it is an increasing function.
\item No, because it is a decreasing function.
\item No, because $\lim_{x\to\infty} e^{-x}=\infty$.
\item No, because it's the PDF of an exponentially distributed rv.
\end{enumerate}
\end{exercise}


\begin{exercise}
If $X\sim  \Unif{[-5, 10]}$, what is $C$ in $f_{X}(x) = C \1{-5, 10}$?

Choose one of these answers:
\begin{enumerate}
\item $C$ is not well defined because $f_{x}(x)$ does not have the right form for PDF.
\item $C=15$
\item $\int_0^{\infty} f_{X}(x) \d x = 10$, hence $C=1/10$.
\item $C=1/15$.
\end{enumerate}
\end{exercise}

\begin{exercise}
Take $f_X(x) = e^{-x}-1/2$ for $x\geq 0$.  Can $f$ be a PDF?

Choose one of these answers:
\begin{enumerate}
\item No, because it's a PMF of a discrete rv.
\item Yes, it's the PDF of the rv $X\sim\Exp{\lambda}$ with $\lambda = 1$.
\item No, because $f(0)  = -1/2$.
\item No, because $f(1000) < 0$.
\end{enumerate}
\end{exercise}


\begin{exercise}
Suppose $X$ does not have a second moment, what is it's variance?
Choose one of these answers:
\begin{enumerate}
\item It's variance can still be $0$ when $\E{X^2} = \infty$ and $(\E X)^2=-\infty$.
\item $\V X=0$, because its second moment does not exist, hence $\E{X^2}=0$, hence $\E X=0$.
\item $\V X = \infty$.
\end{enumerate}
\end{exercise}

\begin{exercise}
Suppose the $X$ has  $\R$ as support. Take $g(x) = \log x$, and define $Y = g(X)$. What is the support of $Y$?
Choose one of these answers:
\begin{enumerate}
\item The support of $Y$ is $\R$.
\item The support of $Y$ is $\R^+$.
\item The support of $Y$ is $(-\infty, 0]$.
\item The support of $Y$ is $(0, \infty)$.
\item $Y$ is not well defined.
\end{enumerate}

\end{exercise}

\begin{exercise}
Suppose the $X$ has  $\R$ as support. Take $g(x) = \log x$, and $h(x) = e^{x}$. Define $Y = h(g(X))$.
Choose one of these answers:
\begin{enumerate}
\item Since $h$ and $g$ are each other's inverse, the support of $Y$ is $\R$.
\item Even though $h$ and $g$ are each other's inverse, $Y$ is still not well defined.
\item If $Y=g(h(X))$ then $Y$ has can have $\R$ as support.
\end{enumerate}

\end{exercise}


\end{document}
