% arara: pdflatex: { shell: yes }
% arara: pythontex: {verbose: yes, rerun: modified }
% arara: pdflatex: { shell: yes }
% arara: move: {  files: ['study-guide.pdf'], target: ['../pdf-files/'] }

\documentclass[a4paper,12pt]{book}


%\usepackage[nosolutions]{optional}
%\usepackage[justhints]{optional}
%\usepackage[check]{optional}
\usepackage[all-solutions-at-end]{optional}

\usepackage{../common/preamble}
%\newcommand{\sectionbreak}{} %\clearpage}


\setcounter{tocdepth}{1}
\usepackage{a4wide}
\usepackage{../common/abbreviations}

\author{Lecturers PT and PD
}
\date{\today}
\title{Probability Distributions, Week 1}

\opt{all-solutions-at-end}{
\Opensolutionfile{hint}
\Opensolutionfile{ans}
}

\begin{document}
\maketitle
\tableofcontents

\chapter{Week 2}
\label{cha:week-1}



\section{Lectures}
\label{sec:lectures}

\subsection{Lecture 3}
\label{sec:lecture-1}

\begin{enumerate}
\item BH Book: Section 7.1.2 joint, marginal, conditional for continuous rvs
\item BH Book: Section 7.1.3 joint, marginal, conditional for hybrid rvs
\item BH video: -
\item BH Exercises: 7.1, 7.10, 7.11
\end{enumerate}


\subsection{Lecture 4}
\label{sec:lecture-1}
\begin{enumerate}
\item BH Book: Section 7.2, 2d lotus
\item BH Book:  Section 7.3, covariance and correlation
\item BH video: 21
\item BH Exercises: 7.38, 7.53, 7.58
\end{enumerate}


\section{Tutorial}
\label{sec:tutorial}

\begin{enumerate}
\item BH Exercises: 7.15,
\item Problems of the  assignment
\end{enumerate}

\section{Assignment}
\label{sec:assignment}

\subfile{../assignments/bh-7-1.tex}
\subfile{../assignments/bh-7-9.tex}



\section{Homework}
\label{sec:homework}

\begin{enumerate}
\item Our exercises on Section BH.7.1.2
\item Redo the exercises of the lectures and the tutorial
\item BH Exercises: 7.24, 7.58, 7.86
\end{enumerate}

\subfile{../study-guide/bh-7.1.2.tex}

\section{Poleverywhere questions}
\label{sec:polev-quest}


TBD


\opt{nosolutions}{
\Closesolutionfile{hint}
\Closesolutionfile{ans}
\clearpage
\chapter{Hints}
\input{hint}
\clearpage
\chapter{Solutions}
\input{ans}
}

\end{document}
