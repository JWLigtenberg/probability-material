\documentclass[study-guide-sol]{subfiles}
\externaldocument{study-guide}

\opt{solutionfiles,check}{
\Opensolutionfile{hint}
\Opensolutionfile{ans}
}

\begin{document}


\subsection{BH.8.4 Figure}
The challenge is to make Figure BH.8.4 for a specific case.

We throw a coin a number of times. We take some beta distribution as prior for the probability $p$ that Heads appears. Then we update the PDF of $p$ according to outcomes of the throws.

\begin{exercise}
Run the follow code and explain the output.

We define the variables HEADS and TAILS as these outcomes are easier to memorize than 1 and 0. Like this, there is less possibility for confusion, hence less bugs.

\begin{minted}[]{python}
HEADS, TAILS = 1, 0
outcomes = [HEADS, TAILS]

h, t = 1, 1  # this
h += outcomes[0] == HEADS
t += outcomes[0] == TAILS
print(f"{h=}, {t=}")

h += outcomes[1] == HEADS
t += outcomes[1] == TAILS
print(f"{h=}, {t=}")
\end{minted}

\begin{minted}[]{R}
HEADS = 1
TAILS = 0
outcomes = c(HEADS, TAILS)

h = 1 #this
t = 1 #this
h = h + as.numeric(outcomes[1] == HEADS)
t = t + as.numeric(outcomes[1]== TAILS)
print(paste0("h=", h, ", t=", t))

h = h + as.numeric(outcomes[2] == HEADS)
t = t + as.numeric(outcomes[2]== TAILS)
print(paste0("h=", h, ", t=", t))
\end{minted}

\end{exercise}


\begin{exercise}
Explain the code below. Why does the line marked as `this' relate to the  prior distribution? What is that prior distribution?
\begin{minted}[]{python}
import numpy as np
from scipy.stats import beta
import matplotlib.pyplot as plt

support = np.arange(0, 1, 0.03)

H, T = 1, 0
outcomes = [HEADS, HEADS, TAILS, HEADS, HEADS, HEADS, TAILS, HEADS, HEADS]
h, t = 1, 1  # this

fig, axes = plt.subplots(3, 3, sharey="all", figsize=(6, 3))  # this
for i, ax in enumerate(axes.flatten()):  # this
    h += outcomes[i] == HEADS
    t += outcomes[i] == TAILS
    X = beta(h, t).pdf(support)
    ax.plot(support, X)
    ax.set_title(f'h={h-1}, t={t-1}')

plt.tight_layout()
plt.savefig("beta.pdf")
\end{minted}

\begin{minted}[]{R}
library(ggplot2)
library(gridExtra)

support = seq(0, 1, 0.03)

HEADS = 1
TAILS = 0
outcomes = c(HEADS, HEADS, TAILS, HEADS, HEADS, HEADS, TAILS, HEADS, HEADS)
h = 1 #this
t = 1 #this

pdf(file="beta.pdf", width = 8, height = 7,
    bg = "white", colormodel = "cmyk", paper = "A4")

plots <-list()
par(oma= c(0,0,0,0))

i = 1
while (i <= length(outcomes)){    #this
  h = h + (outcomes[i] == H)
  t = t + (outcomes[i] == T)
  X = dbeta(support, h, t)
  df = data.frame(support,X)
  plots[[i]] = ggplot(data=df, aes(x=support, y=X, group=1)) +
               geom_line()+
               ylim(0,3.5)+
               labs(title=paste0("h=", h-1, ", t=", t-1),x=" ", y = " ")
  i = i+1
}

do.call("grid.arrange", c(plots, ncol=3))

dev.off()
\end{minted}


\end{exercise}

\begin{exercise}
What do the PDFs actually model? Explain why the PDFs behave as they do. For instance, why does the location of the peak move from right to left to right?
\end{exercise}

\begin{exercise}
Which conjugacy relation between prior and data do we use here to update the prior to the posterior?
\end{exercise}

\begin{exercise}
Adapt the  code above such that the prior is $a=3.5, b= 1.5$, the outcomes are TAILS, HEADS, TAILS, HEADS, TAILS, HEADS, and the output is a figure with 3 x 2 subfigures. Besides the explanation, include the figure.
\end{exercise}



\end{document}

