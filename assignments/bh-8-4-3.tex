\subsection{BH-8-4-3. Sum of two gamma rvs}
\label{sec:sum-two-gamma}

Let us investigate BH.8.4.3. The rvs $X$ and $Y$ are independent, $X\sim\Exp{\lambda}$ and $Y\sim\Exp{\mu}$. We want to compare $X+Y$ to $\Gamma{2, \mu}$ and $\Gamma{2, \lambda}$, just to see what happens.

\begin{exercise}
Run the  code below. Explain the lines marked as `this'. In particular, why do we simulate outcomes for $X$ and $Y$? Then, why are the shape and scale variables as they are? Then explain the three graphs.
\begin{minted}[]{python}
import numpy as np
from scipy.stats import expon, gamma
import matplotlib.pyplot as plt

labda, mu = 3, 5

n = 1000
X = expon(scale=1 / labda)
Y = expon(scale=1 / mu)
Z = X.rvs(n) + Y.rvs(n)
G1 = gamma(2, scale=1 / labda)
G2 = gamma(2, scale=1 / mu)

support = np.arange(0, 5, 0.03)

plt.hist(Z, label="Z", density=True)
plt.plot(support, G1.pdf(support), label="G1")
plt.plot(support, G2.pdf(support), label="G2")
plt.legend()
plt.tight_layout()
plt.savefig("figures/gamma1.pdf")
\end{minted}

\begin{minted}[]{R}
library(ggplot2)

labda = 3
mu = 5

n = 1000
X = rexp(n, labda)
Y = rexp(n, mu)
Z = X + Y

support = seq(0, 5, 0.03)

G1 = dgamma(support, 2, scale=1 / labda)
G2 = dgamma(support, 2, scale=1 / mu)

df1 = data.frame(Z)
df2 = data.frame(support,G1, G2)

pdf(file="figures/gamma1.pdf", width = 8, height = 7, 
    bg = "white", colormodel = "cmyk", paper = "A4")   

ggplot()+
geom_histogram(df1, mapping = (aes(x = Z, y = ..density.., fill = "Z")), bins = 20)+
geom_line(df2, mapping = (aes(x = support, y = G1, color="G1")))+
geom_line(df2, mapping = (aes(x = support, y = G2, color = "G2")))+
scale_fill_manual(name = " ", values = c("Z" = "blue"))  +
scale_color_manual(name = " ", values = c( "G1" = "orange", "G2" = "green"))+
theme(axis.title.x=element_blank(),axis.title.y = element_blank())

dev.off()
\end{minted}
\end{exercise}

\begin{exercise}
Modify the above code such that $X$ and $Y$ are both $\Exp{\lambda}$. Then make a histogram of a simulation of $X+Y$ and compare it to the PDF of the correct gamma  distribution. Discuss your result, and include the figure of course.
\end{exercise}
