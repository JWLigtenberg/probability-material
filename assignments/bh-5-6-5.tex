\subsection{BH.5.6.5}
Read this example of BH first.
We chop up the exercise in many small exercises..


For the python code below, run it for a small number of samples; here I choose \texttt{samples=2}. Read the print statements, and use that to answer the questions below.

\begin{minted}[]{python}
import numpy as np
from scipy.stats import expon

np.random.seed(10)

labda = 6
num = 3
samples = 2

X = expon(scale=labda).rvs((samples, num))
print(X)
T = np.sort(X, axis=1)
print(T)
print(T.mean(axis=0))

expected = np.array([labda / (num - j) for j in range(num)])
print(expected)
print(expected.cumsum())
\end{minted}


\begin{minted}[]{R}
set.seed(10)

labda = 6
num = 3
samples = 2

X = matrix(rexp(samples * num, rate = 1 / labda), nrow = samples, ncol = num)
print(X)
bigT = X
for (i in 1:samples) {
  bigT[i,] = sort(bigT[i,])
}
print(bigT)
print(colMeans(bigT))

expected = rep(0, num)
for (j in 1:num) {
  expected[j] = labda / (num - (j - 1))
}
print(expected)
print(cumsum(expected))
\end{minted}



\begin{exercise}
In line P.11\footnote{Line P.x refers to line x of the Python code.
  Line R.x refers to line x of the R code.}
we print the value of \texttt{X} in line P.10, R.7 and R.8, respectively.
What is the meaning of \texttt{X}?
\begin{solution}
$X$ is a matrix of i.i.d. draws from an exponential distribution with parameter $\lambda$.
\end{solution}
\end{exercise}

\begin{exercise}
What is the meaning of \texttt{T} in line P.12 (R.11)?
\begin{solution}
$T$ is a sorted version of $X$, where we sort each row increasingly.
\end{solution}
\end{exercise}


\begin{exercise}
What do we print in line P.14, R.14?
\begin{solution}
We print the mean value of each column of $T$.
\end{solution}
\end{exercise}

\begin{exercise}
What is meaning of the variable \texttt{expected}?
\begin{solution}
This is an array with expected values of the $i$th order statistic $X_{(i)}$ (see B.H.5.6.5 afor a proof of this result).
\end{solution}
\end{exercise}

\begin{exercise}
 What is the \texttt{cumsum} of \texttt{expected}?
\begin{solution}
The cumsum is the cumulative sum up to and including the current index. So the final entry indicates the expected value of the sum of all three entries of \verb|expected|.
\end{solution}
\end{exercise}

\begin{exercise}
 Now that you understand what is going on, rerun the simulation for a larger number of samples, e.g., 1000, and discuss the results briefly.
\begin{solution}
The result of \verb|print(T.mean(axis=0))| should be close to that of \verb|print(expected.cumsum())|.
\end{solution}
\end{exercise}
