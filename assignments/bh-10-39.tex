\documentclass[study-guide-sol]{subfiles}
\externaldocument{study-guide}

\opt{solutionfiles,check}{
\Opensolutionfile{hint}
\Opensolutionfile{ans}
}

\begin{document}


\subsection{BH.10.39}

BH.10.39.a asks about the first time such that some exponential rv exceeds a certain threshold.
Part b is about when the sum of a number of r.v.s exceed a threshold.
Such problems are called \emph{hitting times}.

\begin{minted}{python}
import numpy as np
from scipy.stats import expon

np.random.seed(3)

X = expon(scale=1)

# part a

N = 10
res = np.zeros(N)
for i in range(N):
    n = 0
    while X.rvs() < 1:
        n += 1
    res[i] = n

print(res.mean(), res.std())

# part b

N = 100
res = np.zeros(N)
for i in range(N):
    M = X.rvs()
    n = 1
    while M < 10:
        M += X.rvs()
        n += 1
    res[i] = n

print(res.mean(), res.std())
\end{minted}

\begin{minted}[]{R}
set.seed(3)

# part a

N = 10
res = rep(0, N)
for (i in 1:N) {
  n = 0
  while (rexp(1, rate = 1) < 1) {
    n = n + 1
    res[i] = n
  }
}

print(mean(res))
print(sd(res))

# part b

N = 100
res = rep(0, N)
for (i in 1:N) {
  M = rexp(1, rate = 1)
  n = 1
  while(M < 10) {
    M = M + rexp(1, rate = 1)
    n = n + 1
    res[i] = n
  }
}

print(mean(res))
print(sd(res))
\end{minted}


By now you have so much experience with reading code that you must be able to explain all without intermediate steps to guide you.
\begin{exercise}
Explain how the first part of the code simulates BH.10.39.a.
\end{exercise}

\begin{exercise}
Explain how the second part of the code simulates BH.10.39.b.
\end{exercise}

\end{document}
