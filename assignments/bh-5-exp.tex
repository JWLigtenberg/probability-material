\documentclass[study-guide-sol]{subfiles}
\externaldocument{study-guide}

\opt{solutionfiles,check}{
\Opensolutionfile{hint}
\Opensolutionfile{ans}
}

\begin{document}


\subsection{Why is the Exponential Distribution so important?}

At the Paris metro, a train arrives every 3 minutes on a platform.
Suppose that 50 people arrive between the departure of a train and an arrival.
It seems entirely reasonable to me to model the arrival times of the individual people as distributed on the interval \([0,3]\).
What is the distribution of the inter-arrival times of these people?
It turns out to be exponential!

You might want to compare your final result to Figure BH.13.1 (It is not forbidden to read the book beyond what you have to do for this course!).
In this exercise we use simulation to see that clustering of arrival times.


\begin{minted}[]{python}
import numpy as np

np.random.seed(3)


num = 5 # small sample at first, for checking.
start, end = 0, 3
labda = num / (end - start)  # per minute
print(1 / labda)

A = np.sort(np.random.uniform(start, end, size=num))
print(A)
print(A[1:])
print(A[:-1])
X = A[1:] - A[:-1]
print(X)

print(X.mean(), X.std())
\end{minted}


\begin{minted}[]{R}
set.seed(3)


num = 5
start = 0
end = 3
labda = num / (end - start)
print(1 / labda)

A = sort(runif(num, min = start, max = end))
print(A)
print(A[-1])
print(A[-length(A)])
X = A[-1] - A[-length(A)]
print(X)

print(mean(X))
print(sd(X))
\end{minted}

\begin{exercise}
Explain the result of line P.12 (R.13)
\begin{solution}
This are the arrival times of 5 passengers within the time interval of 3 minutes (sorted increasingly).
\end{solution}
\end{exercise}

\begin{exercise}
Compare the result of  line P.13 and P.14 (R.12, R.13);  explain what is \texttt{A[1:]} (\texttt{A[-1]})
\begin{solution}
\verb|A[1:]| is an array of all elements of \verb|A| except the first one.
\end{solution}
\end{exercise}

\begin{exercise}
Compare the result of  line P.12 and P.14 (R.11 and R.13);  explain what is \texttt{A[:-1]} (\texttt{A[-length(A)]}).
\begin{solution}
\verb|A[:-1]| is an array of all elements of \verb|A| except the last one.
\end{solution}
\end{exercise}

\begin{exercise}
 Explain what is \texttt{X} in P.15 (R.14)
\begin{solution}
\verb|X| consists of the interarrival times.
\end{solution}
\end{exercise}

\begin{exercise}
Why do we compare $1/\lambda$ and \texttt{X.mean()}?
\begin{solution}
$1/\lambda$ is the expected interarrival time. $\verb|X.mean()|$ is the sample average of the interarrival times.
\end{solution}
\end{exercise}

\begin{exercise}
Recall that $\E X = \sigma (X)$ when $X\sim \Exp{\lambda}$.
Hence, what do you expect to see for \texttt{X.std()}?
\begin{solution}
For \verb|X.std()| we expect to see $1/\lambda = 0.6$ too (if $X$ is indeed exponentially distributed with parameter $\lambda$).
\end{solution}
\end{exercise}

\begin{exercise}
 Run the code for a larger sample, e.g. 50, and discuss (very briefly) your results.
\begin{solution}
For a sample of size 50, we expect an average interarrival time of $0.06$ and an equal standard deviation if the distribution of the interarrival times is indeed exponential. We indeed observe a sample mean and sample average that are very close to this value.
\end{solution}
\end{exercise}


\end{document}
