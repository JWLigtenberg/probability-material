\documentclass[study-guide-sol]{subfiles}
\externaldocument{study-guide}

\opt{solutionfiles,check}{
\Opensolutionfile{hint}
\Opensolutionfile{ans}
}

\begin{document}


\subsection{Why is the Exponential Distribution so important?}

At the Paris metro, a train arrives every 3 minutes on a platform.
Suppose that 50 people arrive between the departure of a train and an arrival.
It seems entirely reasonable to me to model the arrival times of the individual people as distributed on the interval \([0,3]\).
What is the distribution of the inter-arrival times of these people?
It turns out to be exponential!

You might want to compare your final result to Figure BH.13.1 (It is not forbidden to read the book beyond what you have to do for this course!).
In this exercise, we use simulation to see that clustering of arrival times.


\begin{minted}[]{python}
import numpy as np

np.random.seed(3)


num = 5 # small sample at first, for checking.
start, end = 0, 3
labda = num / (end - start)  # per minute
print(1 / labda)

A = np.sort(np.random.uniform(start, end, size=num))
print(A) #this
print(A[1:])
print(A[:-1])
X = A[1:] - A[:-1]
print(X)

print(X.mean(), X.std())
\end{minted}


\begin{minted}[]{R}
set.seed(3)


num = 5
start = 0
end = 3
labda = num / (end - start)
print(1 / labda)

A = sort(runif(num, min = start, max = end))
print(A) #this
print(A[-1])
print(A[-length(A)])
X = A[-1] - A[-length(A)]
print(X)

print(mean(X))
print(sd(X))
\end{minted}

\begin{exercise}
Explain the result of line P. 12 (R.11) marked as 'this'.
\end{exercise}

\begin{exercise}
Compare the result of  line P.13 and P.14 (R.12, R.13);  explain what is \texttt{A[1:]} (\texttt{A[-1]})
\end{exercise}

\begin{exercise}
Compare the result of  line P.12 and P.14 (R.11 and R.13);  explain what is \texttt{A[:-1]} (\texttt{A[-length(A)]}).
\end{exercise}

\begin{exercise}
 Explain what is \texttt{X} in P.15 (R.14).
\end{exercise}

\begin{exercise}
Why do we compare $1/\lambda$ and \texttt{X.mean()}?
\end{exercise}

\begin{exercise}
Recall that $\E X = \sigma (X)$ when $X\sim \Exp{\lambda}$.
Hence, what do you expect to see for \texttt{X.std()}?
\end{exercise}

\begin{exercise}
 Run the code for a larger sample, e.g. 50, and discuss (very briefly) your results.
\end{exercise}


\end{document}

