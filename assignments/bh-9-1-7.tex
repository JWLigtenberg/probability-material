\subsection{BH.9.1.7, optional}
\label{sec:bh.9.1.7}




\emph{You don't have to do this exercise since we ask in the other exercise to include the hand-written solutions. Hence, you are free to skip this exercise. It will not contribute the grade.}

The aim of this exercise is to use a simulator to analyze the expected profit for the game of BH.9.1.7 but for a different prior than the uniform.
First, however, we build BH.9.1.7 with the uniform prior.

First read and solve BH.9.1.7.

\begin{exercise}
Run the code below. What are the values of \verb|accepted|? What does it represent? How is the gain computed? Why do we take the mean?
\begin{minted}[]{python}
import numpy as np
from scipy.stats import uniform

np.random.seed(3)

alpha = 2 / 3
n = 100
V = uniform(0, 1).rvs(n)

bid = 0.5
accepted = bid > alpha * V
gain = (V - bid) * accepted
print(gain.mean())
\end{minted}

\begin{minted}[]{R}
set.seed(3)

alpha = 2 / 3
n = 100
V = runif(n, 0, 1)

bid = 0.5
accepted =  bid > alpha * V
gain = (V - bid) * accepted
print(mean(gain))
\end{minted}
\end{exercise}


\begin{exercise}
Here is some code to compute the gain for multiple bids. Explain how the function $f$ works. In other words, how does it handles the formatting of floating point numbers? The rest of the code must be clear to you, so you don't have to explain that.

Printing pretty floating points is something you really have to learn. If you unquote the line marked as `unquote this line' and rerun the program, you'll see why people like to see formatted numbers.

Hint, search the web (Python or R documentation) on formatting floating point numbers.
\begin{minted}[]{python}
import numpy as np
from scipy.stats import uniform


def f(x):  # format floating point numbers
    # return f"{x}"  # unquote this line
    return f"{x:3.3f}"


np.random.seed(3)


alpha = 2 / 3
n = 100
V = uniform(0, 1).rvs(n)

for b in np.arange(0.1, 1, 0.1):
    accepted = b > alpha * V
    gain = (V - b) * accepted
    print(f"{f(b)}, {f(gain.mean())}")
\end{minted}

\begin{minted}[]{R}
f = function(x) # format floating point numbers
    # return(x) # unquote this line
    return(signif(x, digits = 3))

set.seed(3)

alpha = 2 / 3
n = 100
V = runif(n, 0, 1)

for (b in seq(0.1, 0.9, 0.1)){
    accepted = b > alpha * V
    gain = (V - b) * accepted
    print(paste0(f(b), ", " ,f(mean(gain))))
}
\end{minted}
\end{exercise}


\begin{exercise}
Use the  code of the previous exercise as a starting point to make a graph of the gain as a function of the bid. Compute the bids for $0.05, 0.1, 0.15$, etc, i.e., in steps of $0.05$. Include your code, a figure, and comment on the graph.  Use your student id as seed.
\end{exercise}


\begin{exercise}
It is very easy to compute the expected gain by means of numerical integration. Run the code below and explain how the \verb|res| variable is computed.  Is the result similar to the simulation? Once again (so that you really remember what an anonymous function is!), explain very briefly why we use an anonymous function here.
\begin{minted}[]{python}
import numpy as np
from scipy.integrate import quad

alpha = 2 / 3
n = 100
b = 0.5

res = quad(lambda v: (v - b) * (b > alpha * v), 0, 1)
print(res)
\end{minted}

\begin{minted}[]{R}
library(cubature)

alpha = 2 / 3
n = 100
b = 0.5

res = adaptIntegrate(function(v){(v - b) * (b > alpha * v)},
      lowerLimit = c(0), upperLimit = c(1))
print(paste(res$integral, res$error))
\end{minted}
\end{exercise}

\begin{exercise}
Modify the numerical integrator  such that the prior on $V$ is $\Beta{8, 2}$. Include your code and give the result for a bid of $1/2$.
\end{exercise}
