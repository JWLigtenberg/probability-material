\documentclass[study-guide-sol]{subfiles}
\externaldocument{study-guide}

\opt{solutionfiles,check}{
\Opensolutionfile{hint}
\Opensolutionfile{ans}
}

\begin{document}


\subsection{BH.10.9}


\begin{exercise}
In the code below we compute or simulate the various expressions of BH.10.9. Check the results and compare it to the analytical results. Are both results in line? If not, do you know why that might be the case? What can you do about that?

I don't know how to build $\E{\V{Y|X}}$ in any sensible way. Since $Y$ is independent of $X$, the conditioning does not have any influence. Thus, this numerical exercise is skipped.
\end{exercise}

\begin{minted}{python}
import numpy as np
from scipy.stats import expon

np.random.seed(3)

X = expon(2)
Y = expon(2)

print(np.exp(-X.mean()), X.expect(lambda x: np.exp(-x)))

n_sample = 1000
X_sample = X.rvs(n_sample)
Y_sample = Y.rvs(n_sample)
p1 = sum(X_sample > Y_sample + 3) / n_sample
p2 = sum(Y_sample > X_sample + 3) / n_sample
print(p1, p2)

p3 = sum(X_sample > Y_sample - 3) / n_sample
print(p1, p3)

EXY = X_sample @ Y_sample / n_sample
EX4 = X.expect(lambda x: x ** 4)
print(EX4, EXY * EXY)

p = sum(np.abs(X_sample + Y_sample) > 3) / n_sample
print(p, X.mean())
\end{minted}

\begin{minted}[]{R}
library(cubature)   
set.seed(3)

expectation = adaptIntegrate(function(x){exp(-(x[1]+2)) * exp(-x[1]) },
              lowerLimit = c(0), upperLimit = c(Inf))
print(paste(exp(-mean(rexp(10000,1)+2)), expectation$integral))

n_sample = 1000
X_sample = rexp(n_sample, 1) + 2
Y_sample = rexp(n_sample, 1) + 2
p1 = sum(X_sample > Y_sample +  3) / n_sample
p2 = sum(Y_sample > X_sample + 3) / n_sample
print(paste(p1, p2))

p3 = sum(X_sample > Y_sample - 3) / n_sample
print(paste(p1, p3))

EXY = X_sample %*% Y_sample / n_sample
expectation2 = adaptIntegrate(function(x){((x[1]+2) ** 4) * exp(-x[1]) }, 
              lowerLimit = c(0), upperLimit = c(Inf))

print(paste(expectation2$integral, EXY * EXY))

p = sum(abs(X_sample + Y_sample) > 3) / n_sample
print(paste(p, mean(X_sample)))
\end{minted}


\begin{exercise}
Now make the plot and  explain what is wrong.
\end{exercise}

\begin{exercise}
The line  marked as `this' should be replaced with
\begin{minted}{python}
plt.plot(supp, pmf * np.sqrt(n))
\end{minted}

\textcolor{red}{XXX Arpan: This two questions below do not look related. It seems to be copied from the next question. XXX}
\begin{minted}[]{R}
TBD
\end{minted}
Explain why.  (You should memorize  that comparing PMFs and PDFs is not straightforward.)
\end{exercise}

\begin{exercise}
Why do I use simulation to estimate $\E{XY}$? Recall an earlier assignment in which we used numerical  integration. How successful was that?
\end{exercise}

\begin{exercise}
What would you do to increase the credibility of the claims. (Note these claims  are based on case checking and simulation. There is no actual proof, however, they can act as counter-examples.)
\end{exercise}


\end{document}
