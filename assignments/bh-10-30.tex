\documentclass[study-guide-sol]{subfiles}
\externaldocument{study-guide}

\opt{solutionfiles,check}{
\Opensolutionfile{hint}
\Opensolutionfile{ans}
}

\begin{document}


\subsection{BH.10.30}

\begin{exercise}
Include your hand written solution for this problem here. Read the instructions of the announcement if you don't know what to do.
\end{exercise}

\begin{exercise}
Explain how the code below implements the solution of parts a and b BH.10.30. (Of course we skip the proof of the convergence to $g(\alpha)$ here.)
\end{exercise}

\begin{minted}{python}
import numpy as np
from scipy.stats import bernoulli
import matplotlib.pyplot as plt

np.random.seed(3)

num = 100
p, alpha = 0.5, 0.5
revenue = 0.5 + 1.2 * bernoulli(p).rvs(num)

Y = np.ones(num)
Y[0] = 1
for i in range(1, num):
    Y[i] = (1 - alpha + alpha * revenue[i]) * Y[i - 1]

print(np.log(Y[-1]) / num)

# plt.plot(Y) # this

def g(alpha):
    return 0.5 * np.log((1 + 0.7 * alpha) * (1 - 0.5 * alpha))


plt.ylim(-10 * g(alpha), 10 * g(alpha)) # this
plt.axhline(g(alpha), c='r', lw=2, label="g")
plt.plot(np.log(Y)[1:] / range(1, num), label="(log Yn)/n")
plt.legend()
plt.savefig("figures/bh-10-30.pdf")
\end{minted}

\begin{minted}[]{R}
library(cubature)
set.seed(3)

num = 100
p = 0.5
alpha = 0.5
revenue = 0.5 + 1.2 * rbinom(num, 1, p)

Y = rep(1, num)
Y[1] = 1
for(i in 2:num){
  Y[i] = (1 - alpha + (alpha * revenue[i])) * Y[i - 1]
}

print(log(Y[length(Y)]) / num)

# plot(Y, type = "l")

g = function(alpha){
  return(0.5*log((1 + 0.7 * alpha)*(1 - 0.5 *alpha)))
}

pdf(file="figures/bh-10-30.pdf",
    width = 8, height = 7, bg = "white",
    colormodel = "cmyk", paper = "A4")

df = data.frame(log(Y)/(1:num))
colnames(df) = c("lnY")
p = ggplot()+
    geom_hline(mapping = aes(yintercept = g(alpha), color = "g"), size = 2)+
    geom_line(df,mapping=aes(x = 1:num, y =lnY, color = '(log Yn)/n')) +
    scale_color_manual( name = " ", values = c("g" = "red", "(log Yn)/n" = "blue"))+
    ylim(-10* g(alpha), 10 * g(alpha)) + #this
    theme(axis.title.x = element_blank(), axis.title.y = element_blank())
p

dev.off()
\end{minted}


\begin{exercise}
Uncomment the relevant line to obtain a plot of $Y_{n}$ as a function $n$. Make the plot and include it in your assigment.
\end{exercise}

\begin{exercise}
Take \mintinline{python}{num} much larger, e.g., 100 000. What is $(\log Y_n)/n$ now?
\end{exercise}


\begin{exercise}
Print $(\log Y_n)/n$  for various values of $\alpha$. For instance, take $\alpha = 1/10, 2/10, 3/10, 4/10$.
\end{exercise}

\begin{exercise}
If you like a simple challenge, include a plot of $(\log Y_n)/n$ as a function of $\alpha$. However, skip this if you don't have time, or interest in this extension of the problem. (Perhaps, if you find programming hard, you should do it to improve your skills :-) )
\end{exercise}


\end{document}
