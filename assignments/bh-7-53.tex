\documentclass[study-guide-sol]{subfiles}
\externaldocument{study-guide}

\opt{solutionfiles,check}{
\Opensolutionfile{hint}
\Opensolutionfile{ans}
}

\begin{document}


\subsection{BH.7.53}
\label{sec:bh.53}

\begin{exercise}
Read and solve BH.7.53.
\end{exercise}


\begin{exercise}
The code in \cref{BH.7.53.p} or~\cref{BH.7.53.r} simulates this exercise.
Run the code. (To improve your understanding, just change some parameters here and there, and check the output.) Explain the lines marked with $\# \, *$. (In particular, why do we do a \verb|runs| number of runs to estimate the covariance?) Finally, explain the output.
\end{exercise}

\begin{listing}[!ht]
\begin{minted}[]{python}
import numpy as np
from scipy.stats import randint

np.random.seed(3)


num = 4
steps = randint(0, 4).rvs(num)  # *
print(steps)
X = (2 * steps - 1) * (steps < 2)  # *
Y = (2 * steps - 5) * (steps >= 2)  # *
print(X)
print(Y)
R2 = (X.sum()) ** 2 + (Y.sum()) ** 2
print(f"{num = }, {R2 = }")


def cov(X, Y):
    return (X * Y).mean() - X.mean() * Y.mean()  # *


num = 10
runs = 300
Ss = np.zeros(runs)  # *
Ts = np.zeros(runs)
R2s = np.zeros(runs)
for i in range(runs):  # *
    steps = randint(0, 4).rvs(num)
    X = (2 * steps - 1) * (steps < 2)
    Y = (2 * steps - 5) * (steps >= 2)
    Ss[i] = X.sum()  # *
    Ts[i] = Y.sum()  # *
    R2s[i] = (X.sum()) ** 2 + (Y.sum()) ** 2   # *

print(f"{cov(Ss,Ts) = }")  # *
print(f"{R2s.mean() = }")  # * 
\end{minted}
\caption{BH.7.53, python code.}
\label{BH.7.53.p}


\end{listing}

\begin{listing}[!ht]
\begin{minted}[]{R}
set.seed(3)

num = 4
steps = sample(0:3, num, replace = TRUE)  # *
print(steps)
X = (2 * steps - 1) * (steps < 2)  # *
Y = (2 * steps - 5) * (steps >= 2)  # *
print(X)
print(Y)
R2 = sum(X)^2 + sum(Y)^2
print(paste0("num = ", num, " R2 = ", R2))

cov = function(X, Y) {
  return (mean(X * Y) - mean(X) * mean(Y))  # *
}

num = 10
runs = 300
Ss = rep(0, runs)  # *
Ts = rep(0, runs)
R2s = rep(0, runs)
for (i in 1:runs) {  # *
  steps = sample(0:3, num, replace = TRUE)
  X = (2 * steps - 1) * (steps < 2)
  Y = (2 * steps - 5) * (steps >= 2)
  Ss[i] = sum(X)  # *
  Ts[i] = sum(Y)  # *
  R2s[i] = sum(X)^2 + sum(Y)^2  # *
}

print(paste0("cov(Ss, Ts) = ", cov(Ss, Ts)))  # *
print(paste0("R2 = ", mean(R2s)))  # *
\end{minted}
\caption{BH.7.53, R code.}
\label{BH.7.53.r}
\end{listing}

\begin{exercise}
Modify the code such that the drunkard makes steps of size $2$ when moving left or right; the stepsizes up or down remain the same. What happens with $\E{R^{2}_n}$?
\end{exercise}

\begin{exercise}
Optional: Add a third dimension.
\end{exercise}


\end{document}

