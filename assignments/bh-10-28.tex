\documentclass[study-guide-sol]{subfiles}
\externaldocument{study-guide}

\opt{solutionfiles,check}{
\Opensolutionfile{hint}
\Opensolutionfile{ans}
}

\begin{document}


\subsection{BH.10.28}

\begin{exercise}
Include your hand written solution for this problem here. Read the instructions of the announcement if you don't know what to do.
\end{exercise}

\begin{exercise}
Let us first plot the PMF of the standardized version of $X_{n}$ for $n=10$. Explain the code below BEFORE making the plot.
\end{exercise}

\begin{minted}{python}
import numpy as np
from scipy.stats import poisson, norm
import matplotlib.pyplot as plt

n = 10
X = poisson(n)
N = norm()

pmf = np.zeros(2 * n)
for i in range(len(pmf)):
    pmf[i] = X.pmf(i)
supp = (np.linspace(0, len(pmf), num=len(pmf)) - n) / np.sqrt(n)


plt.plot(supp, pmf)  # this
plt.plot(supp, norm.pdf(supp))
plt.savefig("figures/bh-10-28.pdf")
\end{minted}

\begin{minted}[]{R}
library(ggplot2)

n = 10

pmf = dpois(0:(2 * n), n)
supp = (seq(0, length(pmf), length.out = length(pmf)) - n) / sqrt(n)

pdf(file="figures/bh-10-28.pdf",
    width = 8, height = 7, bg = "white",
    colormodel = "cmyk", paper = "A4")

df1 = data.frame(supp)
df1$pmf = pmf # this
df1$norm_pdf = dnorm(supp,0,1)

p = ggplot(df1) +
    geom_line(mapping =(aes(x=supp, y = pmf)), color = "blue")+
    geom_line(mapping =(aes(x=supp, y = norm_pdf)), color = "orange")+
    theme(axis.title.x=element_blank(),axis.title.y = element_blank())
p

dev.off()
\end{minted}

\begin{exercise}
Now make the plot and  explain what is wrong.
\end{exercise}

\begin{exercise}
The line  marked as `this' should be replaced with
\begin{minted}{python}
plt.plot(supp, pmf * np.sqrt(n))
\end{minted}

\begin{minted}[]{R}
df1$pmf = pmf * sqrt(n)
\end{minted}
Explain why.  (You should memorize  that comparing PMFs and PDFs is not straightforward.)
\end{exercise}


\begin{exercise}
Make the plot of $n=2$ and $n=20$. Include both plots and explain the results.
\end{exercise}

\begin{exercise}
The code below computes $M_{Y}(s)$ where $Y$ is the standardized version of $X_{n}$. Then it compares $M_Y$ to the MGF of the standard normal distribution.
\begin{enumerate}
\item Explain how it works.
\end{enumerate}
\end{exercise}

\begin{minted}{python}
import numpy as np
from scipy.stats import poisson, norm
import matplotlib.pyplot as plt

n = 10
X = poisson(n)
num = 50
S = np.linspace(-1, 1, num)
M = np.zeros(num)
pmf = X.pmf(range(100)) # this

for i in range(num):
    for j in range(len(pmf)):
        M[i] += np.exp(S[i] * (j - n) / np.sqrt(n)) * pmf[j]


plt.plot(S, M, label="M X")
plt.plot(S, np.exp(S * S / 2), label="M N")
plt.xlabel("s")
plt.legend()
plt.savefig("figures/bh-10-28-mgf1.pdf")
\end{minted}

\begin{minted}[]{R}
library(ggplot2)

n = 10
num = 50
S = seq(-1, 1, length.out = num)
M = rep(0, num)
pmf = dpois(0:99,n) #this

for(i in 1:num){
  for(j in 1:length(pmf)){
    M[i] = M[i] + exp(S[i] * (j - 1 - n)/sqrt(n)) * pmf[j]
  }
}

pdf(file="figures/bh-10-28-mgf1.pdf",
    width = 8, height = 7, bg = "white",
    colormodel = "cmyk", paper = "A4")

df1 = data.frame(S)
df1$M_X = M
df1$M_N = exp(S * S / 2)

p = ggplot(df1) +
    geom_line(mapping =(aes(x=S, y = M_X, color = "M X")))+
    geom_line(mapping =(aes(x=S, y = M_N, color = "M N")))+
    scale_color_manual(name = " ", values = c("M X" = "blue", "M N" = "orange"))+
    theme(axis.title.y = element_blank())
p

dev.off()
\end{minted}

\begin{exercise}
The line marked as `this' computes the PMF upfront. Why is that a good idea?
Why do we stop the computation of the PMF at 100?
\end{exercise}


\begin{exercise}
In the code below, one of the for loops is removed.
\begin{enumerate}
\item What is the change?
\item Will this change have an effect on the speed of the computation?
\item Which of the two alternatives do you find more readable?
\end{enumerate}
\end{exercise}


\begin{minted}{python}
import numpy as np
from scipy.stats import poisson, norm
import matplotlib.pyplot as plt

n = 10
X = poisson(n)
num = 50
S = np.linspace(-1, 1, num)
M = np.ones(num)

for i in range(num):
    M[i] = X.expect(lambda j: np.exp(S[i] * (j - n) / np.sqrt(n)))

plt.plot(S, M, label="M X")
plt.plot(S, np.exp(S * S / 2), label="M N")
plt.xlabel("s")
plt.legend()
plt.savefig("figures/bh-10-28-mgf2.pdf")
\end{minted}

\begin{minted}[]{R}
library(ggplot2)

n = 10
num = 50
S = seq(-1, 1, length.out = num)
M = rep(1, num)

for(i in 1:num){
  j = 0:10000
  X = exp(S[i] * (j-n) / sqrt(n))
  M[i] = weighted.mean(X,dpois(j,n))
}

pdf(file="figures/bh-10-28-mgf2.pdf",
    width = 8, height = 7, bg = "white",
    colormodel = "cmyk", paper = "A4")

df1 = data.frame(S)
df1$M_X = M
df1$M_N = exp(S * S / 2)

p = ggplot(df1) +
    geom_line(mapping =(aes(x=S, y = M_X, color = "M X")))+
    geom_line(mapping =(aes(x=S, y = M_N, color = "M N")))+
    scale_color_manual(name = " ", values = c("M X" = "blue", "M N" = "orange"))+
    theme(axis.title.y = element_blank())
p

dev.off()
\end{minted}

\begin{exercise}
Change \mintinline{python}{n} to 20 and to 50. Does the quality improve? Choose some value for \mintinline{python}{n} that you like, and include the graph.
\end{exercise}


\end{document}

