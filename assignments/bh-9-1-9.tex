\subsection{BH.9.1.9}
\label{sec:not-niet}

This is a nice variation of one-step analysis of the coin throwing example of BH.9.1.9.

We have a mouse that sits on one corner of a cube whose edges are made of wire, and  diagonally opposite the mouse there is some cheese.
The mouse chooses at random (uniform) any edge and moves to the next corner.
How many steps does it take, on average, for the mouse to reach the cheese (assuming the cheese does not move)?

For ease, we label assemble  all edges with the same minimal number of edges into states. Thus, the starting corner will be called state 3, then we have a state 2, and 1, and finally state 0 is the corner with the cheese.

\begin{exercise}
Explain in the code below how the while loop simulates the moves of the mouse.  To understand the line marked as `this', read what is a, so-called, ternary operator.
\begin{minted}[]{python}
import numpy as np
from scipy.stats import uniform
import matplotlib.pyplot as plt

move = uniform()


def do_run():
    state = 3
    n_steps = 0
    while state != 0:
        n_steps += 1
        p = move.rvs()
        if state == 3:
            state = 2
        elif state == 2:
            state = 1 if p < 2 / 3 else 3 # this
        elif state == 1:
            state = 2 if p < 2 / 3 else 0
    return n_steps


n = 200
steps = np.zeros(n)
for i in range(n):
    steps[i] = do_run()

print(steps.mean(), steps.std())
plt.hist(steps, density=True)
plt.tight_layout()
plt.savefig("figures/mouse.pdf")
\end{minted}

\begin{minted}[]{R}
library(ggplot2)

do_run = function(){
  state = 3
  n_steps = 0
  while (state != 0){
    n_steps = n_steps + 1
    p = runif(1, 0, 1)
    if (state == 3){
      state = 2
    }
    else if (state == 2){
      state = ifelse(p < 2 / 3, 1, 3) #this
    }
    else if (state == 1){
      state = ifelse(p < 2 / 3, 3, 0) #this
    }
  }
  return (n_steps)
}

n = 200
steps = rep(0, n)

for (i in 1:n){
  steps[i] = do_run()
}

print(paste(mean(steps),sd(steps)))

pdf(file="figures/mouse.pdf",
    width = 8, height = 7, bg = "white",          
    colormodel = "cmyk", paper = "A4") 

df1 = data.frame(steps)
ggplot()+
geom_histogram(df1, mapping = (aes(x = steps, y = ..density.., fill = "steps")))+
scale_fill_manual(name = " ", values = c("steps" = "blue"))  +
theme(axis.title.x=element_blank(),axis.title.y = element_blank())

dev.off()
\end{minted}

\begin{solution}
\end{solution}
\end{exercise}
