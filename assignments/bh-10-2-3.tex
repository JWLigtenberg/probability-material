\documentclass[study-guide-sol]{subfiles}
\externaldocument{study-guide}

\opt{solutionfiles,check}{
\Opensolutionfile{hint}
\Opensolutionfile{ans}
}

\begin{document}



\subsection{BH.10.2.3}

Let us try to understand the weak law of large numbers by means of simulation. An easy example is to take $X_{i}\sim\Unif{0,1}$, so that is what we do here.

\begin{minted}[]{python}
import numpy as np
from numpy.random import uniform

np.random.seed(3)

n = 10
N = 50 # num samples

mu = 1 / 2
var = 1 / 12
eps = 0.1

X = uniform(size=[num_samples, n])

Y = X.mean(axis=1)

larger = np.abs(Y - mu) > eps
count = larger.sum()
P = count / N
RHS = var / (n * eps * eps)
print(P, RHS)
\end{minted}

\begin{minted}[]{R}
set.seed(3)

n = 10
N = 50 # num samples

mu = 1 / 2
var = 1 / 12
eps = 0.1

X = matrix(runif(N * n), N, n)

Y = rowMeans(X)

larger = abs(Y - mu) > eps
count = sum(larger)
P = count / N
RHS = var / (n * eps * eps)
print(P)
print(RHS)
\end{minted}


\begin{exercise}
Explain \texttt{mu} and \texttt{var}.
\end{exercise}
,
\begin{exercise}
What is \texttt{Y}? What is the symbol that BH use for this?
\end{exercise}

\begin{exercise}
What are the meanings of \texttt{larger} and \texttt{count}?
\end{exercise}

\begin{exercise}
What is \texttt{RHS} in the notation of BH?
\end{exercise}

\begin{exercise}
What inequality of BH do we check by printing \texttt{RHS}  and \texttt{P}?
\end{exercise}

\begin{exercise}
Choose some different values for $n$ and the sample size $N$. Is the inequality always true?
\end{exercise}


\end{document}
