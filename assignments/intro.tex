\section*{Introduction}
\addcontentsline{toc}{section}{Introduction}

Here we just provide the exercises of the assignments.  For information with respect to grading we refer to the  course manual.

Each assignment contains code in Python and R that shows how to implement an example or the solution of an exercises of BH. You have to run the code, read the output, and explain how the code works.

We include python and R code, and leave the choice to you what to use.
In the exam we will also include both languages in the same problem, so you can stay with the language you like.
You should know, however, that many of you will need to learn multiple languages later in life.
For instance, when you have to access databases to obtain data about customers, patients, clients, suppliers, inventory, demand, lifetimes (whatever), you often have to use \texttt{sql}.
Once you have the raw data, you process it with \texttt{R} or python to do statistics or make plots.\footnote{(While I (= NvF) worked at a bank, I used Fortran for numerical work, awk for string parsing and making tables, excel, SAS to access the database, and matlab for other numerical work, all next to each other.
I got extremely tired of this, so I went to using python as this can do all of this stuff, but within one language.}
For your interest, based on the statistics \href{https://www.tiobe.com/tiobe-index/}{here} or \href{https://www.northeastern.edu/graduate/blog/most-popular-programming-languages/}{here}, python scores (much) higher than R in popularity; if you opt for a business career, the probability you have to use python is simply higher than to have to use R.

You should become familiar with looking up documentation on coding on the web, no matter your programming language of choice. Invest time in understanding the, at times, rather technical and terse, explanations.  Once you are used to it, the core documentation is faster to read, i.e., less clutter. In the long run, it pays off.



The rules:
\begin{enumerate}
\item For each assigment you have to turn in a pdf document typeset in \LaTeX{}. Include a title, group number, student names and ids, and date.
\item We expect brief answers, just a few sentences, or a number plus some short explanation. The idea of the assignment is to help you studying and improve your coding skills by showing good code, not to turn you in a writer.
\item When you have to turn in a graph, provide decent labels and a legend, ensure the axes have labels too.
\end{enumerate}
