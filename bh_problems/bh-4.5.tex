\documentclass[study-guide-sol]{subfiles}
\externaldocument{study-guide}

\opt{solutionfiles,check}{
\Opensolutionfile{hint}
\Opensolutionfile{ans}
}

\begin{document}

\setcounter{theorem}{4}
\begin{exercise} [BH.4.5]
%\begin{hint}
%\end{hint}
\begin{solution}
    Check A.8.4 for the sums. 
	\begin{align*}
		&\mathbb{E}X=\frac{1}{n}\sum_{i=1}^{n} i= (n+1)/2\\
		&\mathbb{E}X^2=\frac{1}{n}\sum_{i=1}^{n}i^2 = (n+1)(2n+1)/6 \\
		&\mathbb{V}X= \mathbb{E}X^2 -(\mathbb{E}X)^2 =  (n+1)(n-1)/12
	\end{align*}
	Check what happens when n=1, we will see that $\mathbb{V}X=0$. Is this surprising? No, because when $n=1$, then X is only able to take one value, it is a constant (or in other words one degenerated random variable with all probability mass shrinking to a single point) and thus variance is of course zero (we also know if a random variable has zero variance, it has to be equal to its expectation with probability equal to one, and thus zero-variance random variable is a constant.)
\end{solution}
\end{exercise}
\end{document}

