\documentclass[study-guide-sol]{subfiles}
\externaldocument{study-guide}

\opt{solutionfiles,check}{
\Opensolutionfile{hint}
\Opensolutionfile{ans}
}

\begin{document}


\setcounter{theorem}{36}
\begin{exercise}[BH.5.37] Let $T$ be the time until a radioactive particle decays, and suppose (as is often done in
	physics and chemistry) that $T\sim $   Expo$(\lambda)$.
	\begin{enumerate}
		\item  The half-life of the particle is the time at which there is a 50\% chance that the
		particle has decayed (in statistical terminology, this is the median of the distribution of
		$T$). Find the half-life of the particle.
		\item Show that for $\epsilon$ a small, positive constant, the probability that the particle decays in the time interval $[t, t+\epsilon]$, given that it has survived until time $t$, does not depend on $t$ and is approximately proportional to $\epsilon$. (Use $e^x \approx 1+x$ if $x \approx 0$.)
		\item Now consider $n$ radioactive particles, with i.i.d. times until decay $T_1, \ldots, T_n \sim$ $\operatorname{Expo}(\lambda)$. Let $L$ be the first time at which one of the particles decays. Find the CDF of $L$. Also, find $E(L)$ and $\operatorname{Var}(L)$.
		\item Continuing (c), find the mean and variance of $M=\max \left(T_1, \ldots, T_n\right)$, the last time at which one of the particles decays, without using calculus.
		Hint: Draw a timeline, apply (c), and remember the memoryless property.
	\end{enumerate} 
\begin{solution}
    \begin{enumerate}
        \item  Setting $P(T>t)=e^{-\lambda t}=1 / 2$ and solving for $t$, we get that the half-life is
        $$
        t=\frac{\log 2}{\lambda} \approx \frac{0.693}{\lambda} .
        $$
        \item  
        $$
        P(T \in[t, t+\epsilon] \mid T \geq t)=\frac{P(T \in[t, t+\epsilon], T \geq t)}{P(T \geq t)}=\frac{P(t \leq T \leq t+\epsilon)}{P(T \geq t)}=\frac{e^{-\lambda t}-e^{-\lambda(t+\epsilon)}}{e^{-\lambda t}}=1-e^{-\lambda \epsilon},
        $$
        which does not depend on $t$. Alternatively, using the memoryless property: given that $T \geq t$,  $T-t|\{T \geq t\}$ starting from time $t$ again follows $\operatorname{Expo}(\lambda)$.
        
        For $\epsilon>0$ small, the above probability is approximately equal to $1-(1-\lambda \epsilon)=\lambda \epsilon$, which is proportional to $\epsilon$.
        \item  $L \sim \operatorname{Expo}(n \lambda)$, since
        $$
        P(L>t)=P\left(T_1>t, \ldots, T_n>t\right)=e^{-n \lambda t},
        $$
     The CDF of $L$ is given by $F(t)=1-P(L>t)=1-e^{-n \lambda t}$ for $t>0$ and 0 for $t \leq 0$, which gives (once you derive the PDF from the CDF) $E(L)=\frac{1}{n \lambda}$ and $\operatorname{Var}(L)=\frac{1}{(n \lambda)^2}$.
        \item Let $L_1$ be the first time at which a particle decays, $L_2$ be the additional time until another particle decays, $\ldots$, and $L_n$ be the additional time after $n-1$ particles have decayed until the remaining particle decays. Then $M=L_1+\cdots+L_n$. By the memoryless property and the previous part, $L_1, L_2, \ldots, L_n$ are independent with $L_j \sim$ $\operatorname{Expo}((n-j+1) \lambda)$. Thus,
        $$
        \begin{gathered}
            E(M)=E\left(L_1\right)+\cdots+E\left(L_n\right)=\frac{1}{\lambda} \sum_{j=1}^n \frac{1}{j}, \\
            \operatorname{Var}(M)=\operatorname{Var}\left(L_1\right)+\cdots+\operatorname{Var}\left(L_n\right)=\frac{1}{\lambda^2} \sum_{j=1}^n \frac{1}{j^2} .
        \end{gathered}
        $$
    \end{enumerate}
\end{solution}
\end{exercise}

\end{document}

