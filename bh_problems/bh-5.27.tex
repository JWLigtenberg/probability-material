\documentclass[study-guide-sol]{subfiles}
\externaldocument{study-guide}

\opt{solutionfiles,check}{
\Opensolutionfile{hint}
\Opensolutionfile{ans}
}

\begin{document}


\setcounter{theorem}{26}
\begin{exercise}[BH.5.27] Let $Z \sim \mathcal{N}(0,1)$. We know from the $68-95-99.7 \%$ rule that there is a $68 \%$ chance of $Z$ being in the interval $(-1,1)$. Give a visual explanation of whether or not there is an interval $(a, b)$ that is shorter than the interval $(-1,1)$, yet which has at least as large a chance as $(-1,1)$ of containing $Z$.
\begin{solution}
    The PDF of $Z$ is maximized at 0 , and gets smaller and smaller (monotonically) as one moves away from 0 . So there is more area under the PDF curve from $-1$ to 1 then there is area under the curve for any other interval of length 2. More generally, there is more area under the PDF curve from $-c$ to $c$ then there is area under the curve for any other interval of length $2 c$, for any $c>0$. But there is less area for $(-c, c)$ than there is for $(-1,1)$ for $c<1$, so no such interval $(a, b)$ exists.
\end{solution}
\end{exercise}

\end{document}

