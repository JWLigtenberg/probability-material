\documentclass[study-guide-sol]{subfiles}
\externaldocument{study-guide}

\opt{solutionfiles,check}{
\Opensolutionfile{hint}
\Opensolutionfile{ans}
}

\begin{document}


\setcounter{theorem}{9}
\begin{exercise}[BH.5.10]
\begin{solution}
    \begin{enumerate}
        \item Uniform probability is proportional to the interval length $(2+4)/8$.\\~\\
    	Here we can calculate using the fact that for a continuous r.v. X with given PDF $f(.)$ we can simply integrate over the region $A$ to calculate $\mathbb{P}(X\in A) = \int_A f(a) da$:
    	\begin{align*}
    		\mathbb{P}\left(U\in (0,2) \cup (3,7)  \right) = \int_{(0,2) \cup (3,7)} 1/(8-0) da= \int_{0}^2 1/8 da+ \int_{3}^7 1/8 da =3/4
    	\end{align*}
    	
    	\item Unif(3 , 7). (Which proposition in the book?)
    	~\\~~\\
    	\textit{Take a look of the proposition 5.2.3.}\\~\\
    	\textit{Conditional distribution is just one special case of conditional probability, where we look into the probability of a r.v. being smaller than a arbitrary constant (this part resembles the definition of CDF) conditional on certain events (take a look at page 46, the definition of conditional probability.)}
    	Now we derive the conditional distribution using the definition of the conditional probability:  
    	For an arbitrary constant $u\in (3,7)$
    	\begin{align*}
    		\mathbb{P}\left(U\leq u |U\in (3,7) \right) &= \mathbb{P}\left(\{U\leq u\} \cap \{U\in (3,7)\} \right)/ \mathbb{P}\left(U\in (3,7) \right)\\
    		& = \mathbb{P}\left( U\in (3,u)  \right)/ \mathbb{P}\left(U\in (3,7) \right)\\
    		& = \left(\int_{3}^{u}1/8 da\right) /\left(\int_{3}^{7}1/8 da\right)\\
    		& =(u-3)/4
    	\end{align*}
    	To complete, 
    	for $u\leq 3, 	\mathbb{P}\left(U\leq u |U\in (3,7) \right)=0 $; for $u\geq 7, 	\mathbb{P}\left(U\leq u |U\in (3,7) \right)=1$. \\~\\
    	Note that $	\mathbb{P}\left(U\leq u |U\in (3,7) \right)$ is a function of $u$, and this function is equal to the CDF of Unif(3,7)!
    \end{enumerate}
\end{solution}
\end{exercise}

\end{document}

