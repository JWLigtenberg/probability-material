\documentclass[study-guide-sol]{subfiles}
\externaldocument{study-guide}

\opt{solutionfiles,check}{
\Opensolutionfile{hint}
\Opensolutionfile{ans}
}

\begin{document}


\setcounter{theorem}{17}
\begin{exercise}[BH.2.36]
	\begin{enumerate}
		\item Suppose that in the population of college applicants, being good at baseball is independent of having a good math score on a certain standardized test (with respect to some measure of ``good"). A certain college has a simple admissions procedure: admit an applicant if and only if the applicant is good at baseball or has a good math score on the test.
	
		Give an intuitive explanation of why it makes sense that among students that the college admits, having a good math score is \emph{negatively associated} with being good at baseball, i.e., conditioning on having a good math score decreases the chance of being good at baseball.
		\item Show that if $A$ and $B$ are independent and $C = A \cup B$, then $A$ and $B$ are conditionally dependent given $C$ (as long as $\P{A \cap B} > 0$ and $\P{A \cup B} < 1$), with $\P{A|B, C} < \P{A|C}$. This phenomenon is known as \emph{Berkson's paradox}, especially in the context of admissions to a school, hospital, etc.
	\end{enumerate}
\begin{solution}
	This question will of course be super important for later in the studies when issues like selection bias are being discussed. Perhaps you have a nice alternative example as well. Also, Tom has written some $R$ code to illustrate what's going on.
	\begin{minted}[]{R}
		# Author: Tom Boot
		# Date: 22 October 2020
		# Introduction to Probability Question 2.36
		
		library('ggplot2')
		
		# Take a population of size ng
		n <- 1000
		
		# With math skills and baseball skills some number between 0 and 1
		# [runif() gives continuous uniform random variables, 
		# which will be discussed later in the course]
		# and these skills are independent
		skills <- data.frame("baseball"=runif(n),"math"=runif(n))
		
		# Scatter plot of baseball and math skills for the whole population
		ggplot(skills,aes(x=baseball,y=math)) + geom_point() + 
		geom_smooth(method="lm",formula=y~x)
		
		# Only admit students that are good at at least one of baseball and math
		admit <- ((skills$baseball>0.8) + (skills$math>0.8)>0)
		skills_admit <- skills[admit,]
		
		# Scatter plot of baseball and math skills for admitted students
		ggplot(skills_admit,aes(x=skills_admit$baseball,y=skills_admit$math)) +
		geom_point()+ geom_smooth(method="lm",formula=y~x)
	\end{minted}
	\begin{enumerate}
		\item Within the group admitted to the college, if you are bad at baseball, you must be good at math, otherwise you would not be admitted. This creates a negative association between baseball and math skills. See also the $R$ illustration provided:
		\begin{figure}[htbp!]
			\includegraphics[width=0.5\columnwidth]{Rplot.png}
		\end{figure}
		\item \begin{align*}
			P(A|C) =P(A|A\cup B) = \frac{P(A\cap(A\cup B))}{P(A\cup B)} = \frac{P(A)}{P(A\cup B)}>P(A).
		\end{align*}
		Using this result, we see that
		\begin{align*}
			P(A|B,C) = P(A|B\cap (A\cup B)) =P(A|B) = P(A)<P(A|C)
		\end{align*}
		We conclude that $P(A|B,C)\neq P(A|C)$, hence $A$ and $B$ are conditionally dependent given $C$.
	\end{enumerate}
\end{solution}
\end{exercise}
\end{document}

