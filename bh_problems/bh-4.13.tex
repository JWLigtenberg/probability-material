\documentclass[study-guide-sol]{subfiles}
\externaldocument{study-guide}

\opt{solutionfiles,check}{
\Opensolutionfile{hint}
\Opensolutionfile{ans}
}

\begin{document}

\setcounter{theorem}{12}
\begin{exercise} [BH.4.13]
%\begin{hint}
%\end{hint}
\begin{solution}
    Yes. Consider what happens if we make X usually 0 but on rare occasions, X is extremely large (like the outcome of a lottery); Y , on the other hand, can be more moderate. For a simple example, let X be $10^6$ with probability 1/100 and 0 with probability 99/100, and let Y be the constant 1 (which is a degenerated r.v. taking one value with probability one). 
\end{solution}
\end{exercise}
\end{document}

