\documentclass[study-guide-sol]{subfiles}
\externaldocument{study-guide}

\opt{solutionfiles,check}{
\Opensolutionfile{hint}
\Opensolutionfile{ans}
}

\begin{document}


\setcounter{theorem}{24}
\begin{exercise}[BH.5.25] We will show in the next chapter that if $X_1$ and $X_2$ are independent with $X_i \sim$ $\mathcal{N}\left(\mu_i, \sigma_i^2\right)$, then $X_1+X_2 \sim \mathcal{N}\left(\mu_1+\mu_2, \sigma_1^2+\sigma_2^2\right)$. Use this result to find $P(X<Y)$ for $X \sim \mathcal{N}(a, b), Y \sim \mathcal{N}(c, d)$ with $X$ and $Y$ independent. 
\begin{solution}
    Write $P(X<Y)=P(X-Y<0)$ and then standardize $X-Y$. Check that your answer makes sense in the special case where $X$ and $Y$ are i.i.d.
    Solution: Standardizing $X-Y$, we have
    $$
    P(X<Y)=P(X-Y<0)=P\left(\frac{X-Y-(a-c)}{\sqrt{b+d}}<\frac{-(a-c)}{\sqrt{b+d}}\right)=\Phi\left(\frac{c-a}{\sqrt{b+d}}\right) .
    $$
\end{solution}
\end{exercise}

\end{document}

