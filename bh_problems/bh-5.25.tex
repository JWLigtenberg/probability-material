\documentclass[study-guide-sol]{subfiles}
\externaldocument{study-guide}

\opt{solutionfiles,check}{
\Opensolutionfile{hint}
\Opensolutionfile{ans}
}

\begin{document}


\setcounter{theorem}{24}
\begin{exercise}[BH.5.25] 
\begin{solution}
    Write $P(X<Y)=P(X-Y<0)$ and then standardize $X-Y$. Check that your answer makes sense in the special case where $X$ and $Y$ are i.i.d.
    Solution: Standardizing $X-Y$, we have
    $$
    P(X<Y)=P(X-Y<0)=P\left(\frac{X-Y-(a-c)}{\sqrt{b+d}}<\frac{-(a-c)}{\sqrt{b+d}}\right)=\Phi\left(\frac{c-a}{\sqrt{b+d}}\right) .
    $$
\end{solution}
\end{exercise}

\end{document}

