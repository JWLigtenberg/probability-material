\documentclass[study-guide-sol]{subfiles}
\externaldocument{study-guide}

\opt{solutionfiles,check}{
\Opensolutionfile{hint}
\Opensolutionfile{ans}
}

\begin{document}

\setcounter{theorem}{26}
\begin{exercise} [BH.4.27]
%\begin{hint}
%\end{hint}
\begin{solution}
    Note, $X+Y$ now only takes even numbers, why (because $X+Y=2X$ as $X=Y$)? Is it still Poisson? No, because Poisson random variable would also take odd numbers with positive probability....\\~\\
	Alternatively, check the variance:
	\begin{align*}
		\mathbb{V}(X+Y) =\mathbb{V}(2X) =
		4\mathbb{V}(X) =4\lambda
	\end{align*}
	while if $T\sim Pois(2\lambda)$ we should have
	\begin{align*}
		\mathbb{V}(T) =2\lambda \neq 4\lambda
	\end{align*}
	which also disproves the statement. \\~\\
	\textit{Further comment: to prove the  claim in this exercise is incorrect, we only need to check that some features of the r.v. do not match with the ones of a Poisson distributed r.v. Well, in the lecture/textbook, we have results that \textbf{sum of independent Poisson is still Poisson}, where we prove by looking at the PMF or MGF (note these can determine the distribution, well in most cases only knowing variance/expectation is not enough to determine the distribution unless there is other restrictions). \\~\\This exercise also emphasizes the importance of \textbf{independence}, when it is missing we see that sum of Poisson may no longer be a Poisson distribution.}
\end{solution}
\end{exercise}
\end{document}

