\documentclass[study-guide-sol]{subfiles}
\externaldocument{study-guide}

\opt{solutionfiles,check}{
\Opensolutionfile{hint}
\Opensolutionfile{ans}
}

\begin{document}

\setcounter{theorem}{3}

\begin{exercise}[BH.3.8]
  There are 100 prizes, with one worth \$1, one worth \$2, ..., and one worth \$100. There are 100 boxes, each of which contains one of the prizes. You get 5 prizes by picking random boxes one at a time, without replacement. Find the PMF of how much your most valuable prize is worth (as a simple expression in terms of binomial coefficients).
\begin{solution}
	Let $X$ be equal to $k$ if $k$ is your most valuable prize ($k=5,\ldots,100$). Suppose your most valuable price is \$27. Then you draw 4 out of the first 26 boxes and the 27th box. Using the naive definition of probability
	\begin{align*}
		P(X=27) = \frac{{26\choose 4}}{{100 \choose 5}}.
	\end{align*}
	In general,
	\begin{align*}
		p_{X}(k) = P(X=k) = \frac{{k-1\choose 4}}{{100 \choose 5}}\quad \text{for } k=5,\ldots,100,
	\end{align*}
	and $p_{X}(k)=0$ otherwise.
\end{solution}
\end{exercise}

\end{document}

