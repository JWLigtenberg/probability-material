\documentclass[study-guide-sol]{subfiles}
\externaldocument{study-guide}

\opt{solutionfiles,check}{
\Opensolutionfile{hint}
\Opensolutionfile{ans}
}

\begin{document}

\setcounter{theorem}{7}

\begin{exercise}[BH.1.10]
	To fulfill the requirements for a certain degree, a student can choose to take any 7 out of a list of 20 courses, with the constraint that at least 1 of the 7 courses must be a statistics course. Suppose that 5 of the 20 courses are statistics courses.
	\begin{enumerate}
		\item How many choices are there for which 7 courses to take?
		\item Explain intuitively why the answer to (a) is \emph{not} $\binom{5}{1} \cdot \binom{19}{6}$.
	\end{enumerate}
\begin{solution}~
	 \begin{enumerate}
	 	\item Without restrictions there would be $N_{u}={20\choose 7}$ options. From those, we need to subtract the number of choices that do not contain statistics course, which would amount to choosing 7 courses from the 15 non-statistics courses: $N_{r} = {15\choose 7}$. Then, $N=N_{u}-N_{r}={20\choose 7}-{15 \choose 7}$.
		\item The argument seems to be that we first choose 1 out of 5 statistics courses, so ${5 \choose 1}$ options for those, and then argue that the remaining 6 courses would be free to choose, so ${19 \choose 6}$ options for those. But now we are counting things double. Consider choosing Statistics 1 and Statistics 2, and 5 non-statistics courses. Then we are counting this options twice: once with Statistics 1 as \textit{the} statistics course in the ${5 \choose 1}$ term and once with Statistics 2 as \textit{the} statistics course in the ${5 \choose 1}$ term. So we are overcounting!
	 \end{enumerate}
\end{solution}
\end{exercise}
\end{document}

