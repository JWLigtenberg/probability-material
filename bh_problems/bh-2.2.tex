\documentclass[study-guide-sol]{subfiles}
\externaldocument{study-guide}

\opt{solutionfiles,check}{
\Opensolutionfile{hint}
\Opensolutionfile{ans}
}

\begin{document}


\setcounter{theorem}{1}
\begin{exercise}[BH.2.2]
A woman is pregnant with twin boys. Twins may be either identical or fraternal. Suppose that 1/3 of twins born are identical, that identical twins have a 50\% chance of being both boys and a 50\% chance of being both girls, and that for fraternal twins each twin independently has a 50\% chance of being a boy and a 50\% chance of being a girl. Given the above information, what is the probability that the woman’s twins are identical?
\begin{solution}
	Define the events. $I$: identical twins, $BB$: two boys. We are given that
	\begin{align*}
		P(I) &= \frac{1}{3},\\
		P(BB|I) & = \frac{1}{2},\\
		P(BB|F)& = \frac{1}{2}\cdot\frac{1}{2} = \frac{1}{4}.
	\end{align*}
	What is $P(I|BB)$? Again, we use Bayes' rule:
	\begin{align*}
		P(I|BB)&= \frac{P(BB|I)P(I)}{P(BB|I)P(I) + P(BB|I^{C})P(I^C)}\\
		&=\frac{ \frac{1}{2}\frac{1}{3}}{\frac{1}{2}\frac{1}{3} + \frac{1}{4}\frac{2}{3}}\\
		& = \frac{1}{2}.
	\end{align*}
\end{solution}
\end{exercise}
\end{document}

