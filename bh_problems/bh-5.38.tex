\documentclass[study-guide-sol]{subfiles}
\externaldocument{study-guide}

\opt{solutionfiles,check}{
\Opensolutionfile{hint}
\Opensolutionfile{ans}
}

\begin{document}


\setcounter{theorem}{37}
\begin{exercise}[BH.5.38] Fred wants to sell his car, after moving back to Blissville (where he is happy with
	the bus system). He decides to sell it to the first person to offer at least \$18,000 for
	it. Assume that the offers are independent Exponential random variables with mean
	\$12,000, and that Fred is able to keep getting offers until he obtains one that meets his
	criterion.
	\begin{enumerate}
		\item Find the expected number of offers Fred will have.
		\item  Find the expected amount of money that Fred will get for the car.
	\end{enumerate} 
\begin{solution}
    \begin{enumerate}
        \item Let $X_i \sim \operatorname{i.i.d.~ Expo}(1 / 12000)$ denote the $i$th offer. So the number of offers that are too low is $\operatorname{Geom}(p)$ with $p=P\left(X_i \geq 18000\right)=e^{-1.5}$. The expected number of offers (including the successful offer) is thus $(1-p) / p+1=1 / p=e^{1.5}$ using the properties of the Geometric distribution.
        \item Let $N$ be the number of offers, so the sale price of the car is the $N$th offered price $X_N$ (unlike the $i$th in the above solution, here we have a random index $N$). Note that
        $$
        E\left(X_N\right)=E(X \mid X \geq 18000)
        $$
        for $X \sim \operatorname{Expo}(1 / 12000)$, since the successful offer is an Exponential for which our information is that the value is at least $\$ 18,000$. The memoryless property implies that the distribution of $X-a$ given $X>a$ is itself $\operatorname{Expo}(\lambda)$. So via the linearity of the expectation and the memoryless property of Expo:
        $$
        E(X \mid X \geq 18000)=	18000+		E(X-18000 \mid X \geq 18000)=18000+E(X)=30000,
        $$
        which shows that Fred's expected sale price is $\$ 30,000$.
    \end{enumerate} 
\end{solution}
\end{exercise}

\end{document}

