\documentclass[study-guide-sol]{subfiles}
\externaldocument{study-guide}

\opt{solutionfiles,check}{
\Opensolutionfile{hint}
\Opensolutionfile{ans}
}

\begin{document}


\setcounter{theorem}{37}
\begin{exercise}[BH.5.38]
\begin{solution}
    \begin{enumerate}
        \item Let $X_i \sim \operatorname{i.i.d.~ Expo}(1 / 12000)$ denote the $i$th offer. So the number of offers that are too low is $\operatorname{Geom}(p)$ with $p=P\left(X_i \geq 18000\right)=e^{-1.5}$. The expected number of offers (including the successful offer) is thus $(1-p) / p+1=1 / p=e^{1.5}$ using the properties of the Geometric distribution.
        \item Let $N$ be the number of offers, so the sale price of the car is the $N$th offered price $X_N$ (unlike the $i$th in the above solution, here we have a random index $N$). Note that
        $$
        E\left(X_N\right)=E(X \mid X \geq 18000)
        $$
        for $X \sim \operatorname{Expo}(1 / 12000)$, since the successful offer is an Exponential for which our information is that the value is at least $\$ 18,000$. The memoryless property implies that the distribution of $X-a$ given $X>a$ is itself $\operatorname{Expo}(\lambda)$. So via the linearity of the expectation and the memoryless property of Expo:
        $$
        E(X \mid X \geq 18000)=	18000+		E(X-18000 \mid X \geq 18000)=18000+E(X)=30000,
        $$
        which shows that Fred's expected sale price is $\$ 30,000$.
    \end{enumerate} 
\end{solution}
\end{exercise}

\end{document}

