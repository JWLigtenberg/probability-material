\begin{exercise}[BH.2.7]
Fred is working on a major project. In planning the project, two milestones are set up, with dates by which they should be accomplished. This serves as a way to track Fred’s progress. Let $A_1$ be the event that Fred completes the first milestone on time, $A_2$ be the event that he completes the second milestone on time, and $A_3$ be the event that he completes the project on time.
	
	Suppose that $\P{A_{j+1}|A_j} = 0.8$ but $\P{A_{j+1}|A^c_j} = 0.3$ for $j = 1,2$, since if Fred falls behind on his schedule it will be hard for him to get caught up. Also, assume that the second milestone supersedes the first, in the sense that once we know whether he is on time in completing the second milestone, it no longer matters what happened with the first milestone. We can express this by saying that $A_1$ and $A_3$ are conditionally independent given $A_2$ and they're also conditionally independent given $A^c_2$.
\begin{solution}~
	\begin{enumerate}
		\item Use LOTP with extra conditioning on $A_{2}$, and then the conditional independence of $A_{3}$ and $A_{1}$ given $A_{2}$ or $A_{2}^{C}$ that is given in the question.
		\begin{align*}
			P(A_{3}|A_{1}) &= P(A_{3}|A_{1},A_{2})P(A_{2}|A_{1}) + P(A_{3}|A_{1},A_{2}^{C})P(A_{2}^{C}|A_{1})\\
			&= P(A_{3}|A_{2})P(A_{2}|A_{1}) + P(A_{3}|A_{2}^{C})P(A_{2}^{C}|A_{1})\\
			&= 0.8^2 + 0.3\cdot 0.2\\
			& = 0.7.
		\end{align*}
		And also,
		\begin{align*}
			P(A_{3}|A_{1}^{C}) &= P(A_{3}|A_{1}^{C},A_{2})P(A_{2}|A_{1}^{C}) + P(A_{3}|A_{1}^{C},A_{2}^{C})P(A_{2}^{C}|A_{1}^{C})\\
			&= P(A_{3}|A_{2})P(A_{2}|A_{1}^{C}) + P(A_{3}|A_{2}^{C})P(A_{2}^{C}|A_{1}^{C})\\
			&= 0.8 \cdot 0.3 + 0.3\cdot 0.7\\
			& = 0.45.
		\end{align*}
		\item  Using the law of total probability (with $A_{1}$ as conditioning event) and the results from (a), we find that
		\begin{align*}
			P(A_{3}) &= P(A_{3}|A_{1})P(A_{1}) +P(A_{3}|A_{1}^{C})P(A_{1}^{C})\\
			& = 0.7\cdot 0.75 + 0.45\cdot 0.25\\
			& = 0.6375.
		\end{align*}
	\end{enumerate}
\end{solution}
\end{exercise}
