\documentclass[study-guide-sol]{subfiles}
\externaldocument{study-guide}

\opt{solutionfiles,check}{
\Opensolutionfile{hint}
\Opensolutionfile{ans}
}

\begin{document}

\setcounter{theorem}{16}
\begin{exercise}[BH.1.34]
	A random 5-card poker hand is dealt from a standard deck of cards. Find the probability of each of the following possibilities (in terms of binomial coefficients).
	\begin{enumerate}
		\item A flush (all 5 cards being of the same suit; do not count a royal flush, which is a flush with an ace, king, queen, jack, and 10).
		\item Two pair (e.g., two 3’s, two 7’s, and an ace).
	\end{enumerate}
\begin{solution}~
	\begin{enumerate}
		\item There are ${13 \choose 5}$ ways to get 5 cards with the suit Spades (for example), so ${13 \choose 5}-1$ ways to get 5 cards with the suit Spades excluding the royal flush. A flush can appear for all suits, so there are $N_{F}=4\cdot ({13 \choose 5}-1)$ ways to get a flush. There are $N_{T}={52 \choose 5}$ ways to draw 5 cards from a standard deck. Then, denoting by $F$ the event of a flush, $$P(F) = \frac{N_{F}}{N_{T}} = \frac{4 \cdot ({13 \choose 5}-1)}{{52 \choose 5}}.$$
		\item[34b.] There are ${13 \choose 2}$ ways to choose the ranks forming the pair (say 3s and 7s) and then ${4 \choose 2}$ ways to choose the 3s and ${4 \choose 2}$ ways to choose the 7s. There are then $11\cdot 4$ (rank times choice between 4 cards of this rank) choices for the remaining card. In total, denoting by $2P$ the event of two pair,
		$$P(2P) = \frac{{13 \choose 2}\cdot {4 \choose 2}^2 \cdot 11\cdot 4}{{52 \choose 5}}.$$
	\end{enumerate}
\end{solution}
\end{exercise}

\end{document}

