\documentclass[study-guide-sol]{subfiles}
\externaldocument{study-guide}

\opt{solutionfiles,check}{
\Opensolutionfile{hint}
\Opensolutionfile{ans}
}

\begin{document}


\setcounter{theorem}{29}
\begin{exercise}[BH.5.30] Let $Z \sim N(0,1)$ and let $S$ be a random sign independent of $Z$, i.e., $S$ is 1 with probability 1/2 and -1 with probability 1/2. Show that $SZ \sim N (0, 1)$.
\begin{solution}
    To show $S\sim N(0,1)$ we only need to show its distribution is the same as the one of Z (whose distribution is N(0,1)):
	\begin{align*}
		\mathbb{P}\left(SZ\leq y \right) = \mathbb{P}\left(SZ\leq y|S=1 \right)\mathbb{P}(S=1)+\mathbb{P}\left(SZ\leq y|S=1 \right)\mathbb{P}(S=-1) =\mathbb{P}\left(Z\leq y\right) 
	\end{align*}
	(or use MGF)
	\\~~\\
	Alternatively, we can use MGF to check:
	\begin{align*}
		M_{SZ}(t) =& \mathbb{E}e^{tSZ} \\
		=_{\textit{Independence}}&~ \mathbb{E}e^{-tZ} 1/2 + \mathbb{E}e^{tZ} 1/2 = e^{t^2/2} 
	\end{align*}
	which is a MGF of N(0,1). (The above calculation of MGF is slightly beyond the scope of this course, as it may require the concept of conditional expectation.)
\end{solution}
\end{exercise}

\end{document}

