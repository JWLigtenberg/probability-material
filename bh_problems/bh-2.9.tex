\documentclass[study-guide-sol]{subfiles}
\externaldocument{study-guide}

\opt{solutionfiles,check}{
\Opensolutionfile{hint}
\Opensolutionfile{ans}
}

\begin{document}


\setcounter{theorem}{5}

\begin{exercise} [BH.2.9] 
 \begin{enumerate}
		\item Show that if events $A_1$ and $A_2$ have the same prior probability $\P{A_1} = \P{A_2}$, $A_1$ implies $B$, and $A_2$ implies $B$, then $A_1$ and $A_2$ have the same posterior probability $\P{A_1|B} = \P{A_2|B}$ if it is observed that $B$ occurred.
		\item Explain why (a) makes sense intuitively, and give a concrete example.
	\end{enumerate}
\begin{solution}~
	\begin{enumerate}
		\item The thing to realize is that $A_{1}$ implies $B$ in set notation is $A_{1}\subseteq B$. Therefore, $P(A_{1}\cap B) = P(A_1)$. The same is true for $A_{2}$. Then,
		\begin{align*}
			P(A_{1}|B) &= \frac{P(A_{1}\cap B)}{P(B)} = \frac{P(A_{1})}{P(B)},\\
			P(A_{2}|B) &= \frac{P(A_{2}\cap B)}{P(B)} = \frac{P(A_{2})}{P(B)}.\\
		\end{align*}
		If $P(A_{1})=P(A_{2})$, then the right hand sides are equal, so the left hand sides must also be equal.
		\item Suppose the probability of studying Econometrics $(A_{1})$ is the same as the probability of studying Mathematics $(A_{2})$. Denote by $B$ the probability of going to the university. Of course, $A_{1}$ and $A_{2}$ imply $B$. The problem is completely symmetric in $A_{1}$ and $A_{2}$. If I now observe that someone is going to the university, I have to update the probabilities for $A_{1}$ and $A_{2}$ in the same way. 
	\end{enumerate}
  \end{solution}
\end{exercise}

\end{document}

