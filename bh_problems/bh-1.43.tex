\documentclass[study-guide-sol]{subfiles}
\externaldocument{study-guide}

\opt{solutionfiles,check}{
\Opensolutionfile{hint}
\Opensolutionfile{ans}
}

\begin{document}

\setcounter{theorem}{20}

\begin{exercise}[BH.1.43]
	Show that for any events A and B, $\P{A} + \P{B} - 1 \leq \P{A \cap B} \leq \P{A \cup B} \leq \P{A} + \P{B}$. For each of these three inequalities, give a simple criterion for when the inequality is actually an equality (e.g., give a simple condition such that $\P{A \cap B} = \P{A \cup B}$ if and only if the condition holds).
\begin{solution}
	$P(A\cap B)\leq P(A\cup B)$, since $A\cap B\subseteq A\cup B$. We also know from the inclusion-exclusion principle that $P(A\cup B) = P(A) + P(B)-P(A\cap B)$. Hence, $P(A\cap B) = P(A)+P(B)-P(A\cup B)\geq P(A) + P(B)-1$ (where the inequality uses that probabilities are $\leq 1$). Since $P(A\cap B)\geq 0$, we also have from $P(A\cap B) = P(A)+P(B)-P(A\cup B)$ that $P(A\cup B)\leq P(A) + P(B)$.

	The first inequality ($P(A\cap B)\leq P(A\cup B)$) becomes an equality when $A\cap B= A\cup B$. This means (draw a Venn diagram) that $A\cap B^{C}$ and $A^{C}\cap B$ are of zero probability. E.g., $A=B$.

	For the second inequality $(P(A\cap B) \geq P(A) + P(B)-1)$, we see from the derivation above that it holds with equality if $P(A\cup B)=1$. Also $P(A\cup B)=P(A)+P(B)$ if $P(A\cap B)=0$. E.g., $A=\emptyset$.
\end{solution}
\end{exercise}
\end{document}

