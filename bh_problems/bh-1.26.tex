\documentclass[study-guide-sol]{subfiles}
\externaldocument{study-guide}

\opt{solutionfiles,check}{
\Opensolutionfile{hint}
\Opensolutionfile{ans}
}

\begin{document}

\setcounter{theorem}{14}
\begin{exercise}[BH.1.26]
	A survey is being conducted in a city with 1 million residents. It would be far too expensive to survey all of the residents, so a random sample of size 1000 is chosen (in practice, there are many challenges with sampling, such as obtaining a complete list of everyone in the city, and dealing with people who refuse to participate). The survey is conducted by choosing people one at a time, with replacement and with equal probabilities.
	\begin{enumerate}
		\item Explain how sampling with vs. without replacement here relates to the birthday problem.
		\item Find the probability that at least one person will get chosen more than once.
	\end{enumerate}
\begin{solution}~
	\begin{enumerate}
		\item In the birthday problem, birthdays enter the room with replacement. Here survey respondents enter with replacement.
		\item From the analysis of the birthday problem, we know that it is much easier to look at the probability of no match ($A$).
		$$P(A) = 1\cdot \left(1-\frac{1}{10^6}\right)\left(1-\frac{2}{10^6}\right)\ldots \left(1-\frac{999}{10^6}\right).$$
		The probability that at least one person will be chosen more than once is equal to $1-P(A) \approx 0.393$.
	\end{enumerate}
\end{solution}
\end{exercise}

\end{document}

