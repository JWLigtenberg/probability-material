\documentclass[study-guide-sol]{subfiles}
\externaldocument{study-guide}

\opt{solutionfiles,check}{
\Opensolutionfile{hint}
\Opensolutionfile{ans}
}

\begin{document}

\setcounter{theorem}{29}
\begin{exercise} [BH.4.30]
%\begin{hint}
%\end{hint}
\begin{solution}
    \begin{enumerate}
        \item By LOTUS we know the expectation can be calculated as: 
        $$	\sum_{i=0}^\infty ig(i) e^{-\lambda} \lambda^i/i! $$
        Furthermore,
        \begin{align*}
            \sum_{i=0}^\infty ig(i) e^{-\lambda} \lambda^i/i! = 	\lambda \sum_{i=1}^\infty g(i) e^{-\lambda} \lambda^{i-1}/(i-1)!= 	 \sum_{i=0}^\infty \lambda g(i+1) e^{-\lambda} \lambda^{i}/(i)!  
        \end{align*}
        where the last term $ \sum_{i=0}^\infty \lambda g(i+1) e^{-\lambda} \lambda^{i}/(i)!  $ equals $\lambda \sum_{i=0}^\infty  g(i+1) \mathbb{P}(X=i)$, which is $\lambda\mathbb{E}\left(g(X+1)\right)$.\\~\\
        \item First by the definition of variance  and $\mathbb{E}X=\lambda, \mathbb{V}X=\lambda$ we know:
        \begin{align*}
            \mathbb{E}X^2 = (\mathbb{E}X)^2 +\mathbb{V}X =\lambda^2 +\lambda
        \end{align*}
        Then we choose $g(x)=x^2$, the result from (a) implies that: 
        \begin{align*}
            \mathbb{E}X^3= \mathbb{E}\lambda (X+1)^2 = \lambda\left(\mathbb{E}X^2 +2\mathbb{E}X +1  \right)    
        \end{align*}
        and thus $\mathbb{E}X^3 = \lambda (\lambda^2+3\lambda +1).$\\~~\\
        This is a smart way to calculate $\mathbb{E}X^3$, which is also doable by LOTUS and the computation would be much longer than the above.
    \end{enumerate}
\end{solution}
\end{exercise}
\end{document}

