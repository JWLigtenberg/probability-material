% arara: pdflatex: { shell: yes }
% arara: pythontex: {verbose: yes, rerun: modified }
% arara: pdflatex: { shell: yes }
% arara: move: {  files: ['chapter9.pdf'], target: ['..'] }

\documentclass[study-guide-sol]{subfiles}
\externaldocument{study-guide}

\opt{solutionfiles,check}{
\Opensolutionfile{hint}
\Opensolutionfile{ans}
}

\begin{document}



\chapter{Chapter 9}



\setcounter{theorem}{0}
\begin{exercise}
BH.9.1. It is best to solve this problem with EVE's law.
\begin{solution}
a.
\begin{align*}
\E{T\given R=j} &= \mu_{j} \\
\E{T\given R} &= \mu_{R} = \sum_j \mu_j \1{R=j},\\
\E{T} &= \E{\E{T\given R}} = \E{\sum_j \mu_{j}\1{R=j}} = \sum_j\mu_{j}p_{j} \\
&=(\mu_{1}+\mu_2+\mu_3)/3.
\end{align*}
For b., use a. And then,
\begin{align*}
\V{T|R=j} &= \sigma_j^2 \\
\V{T|R} &= \sum_j\sigma_j^2\1{R=j} \\
\E{V{T|R}} &= (\sigma_1^2+\sigma_2^2+\sigma_3^3)/3,\\
\V{\E{T\given R}}
&= \E{(\E{T\given R})^{2}}  - (\E{\E{T|R}})^2 \\
(\E{T\given R})^{2}
&= \left(\sum_j \mu_j \1{R=j}\right)^{2}
= \sum_j \mu_j^{2} \1{R=j} + 2\mu_1 \mu_2\1{R=1}\1{R=2} + \cdot\\
&= \sum_j \mu_j^{2} \1{R=j}.\\
\E{(\E{T\given R})^{2}} &= \sum_j\mu_j^{2}p_j= (\mu_1^2+\mu_2^2+\mu_{3}^2)/3\\
\V{\E{T\given R}}
&= \E{(\E{T\given R})^{2}}  - (\E{\E{T|R}})^2 \\
&=(\mu_1^2+\mu_2^2+\mu_{3}^2)/3 - (\mu_1+\mu_2+\mu_3)^2/9.  \\
\V T
 &= \V{\E{T|R}} + \E{\V{T|R}} \\
&=(\mu_1^2+\mu_2^2+\mu_{3}^2)/3 - (\mu_1+\mu_2+\mu_3)^2/9 + (\sigma_1^2+\sigma_2^2+\sigma_3^3)/3.
 \end{align*}
\end{solution}
\end{exercise}

\setcounter{theorem}{24}

\begin{exercise}
BH.9.25. We tackle this problem also in an assignment with simulation.  Check out \url{https://en.wikipedia.org/wiki/Kelly_criterion} you're interested.
\begin{hint}
Use Adam's law to express $\E{X_{n+1}}$ in terms of $\E{X_{n}}$, then use recursion.
\end{hint}
\begin{solution}
\begin{align*}
  \E{X_{n+1} |X_n=100} &= 100 + pf 100 - (1-p)f 100  = 100(1-f + 2pf) \\
  \E{X_{n+1} |X_n} &= X_{n}( 1- f + 2pf) \\
  \E{X_{n+1} } &= ( 1- f + 2pf) \E{X_{n}} \\
  \E{X_{n+1} } &= ( 1- f + 2pf)^{2} \E{X_{n-1}} =  ( 1- f + 2pf)^{n+1}X_{0}.
\end{align*}
\end{solution}
\end{exercise}


\setcounter{theorem}{27}
\begin{exercise}
BH.9.28.
\end{exercise}




\setcounter{theorem}{31}
\begin{exercise}
BH.9.32. The results of this exercise are (or should be) used by nearly all software packages to control inventory levels of companies such as supermarkets and bol.com.
\begin{hint}
a. Let $Y$ be the amount purchased by the first customer that comes along, let $P$ be the rv that is 1 if the customer does indeed purchase, and 0 otherwise, and let $X$ be the size of the purchase. Why is $Y=XP$? What is $\E{P}$? What is $\E{Y\given P}$? What is $\E{Y^{2}\given P}$. You might want to use BH.9.1.


b. Let $N\sim\Pois{8\lambda}$ be the number of customers that pass by. Given $N=n$, what is $\E{S\given N}$, where $S=\sum_{i=1}^N X_iP_{i}$ is the total sales. Now use the law of total expectation. What is $\V{S\given N}$? Use Eve's law  to compute $\V S$. Bigger hint, read Example 9.6.1.
\end{hint}
\begin{solution}
Use that $P^{2}=P$ (indicator funtion), Adam and Eve, and that $N\sim \Pois{8\lambda}$,
\begin{align*}
  \E{Y|P}&= \E{PX|P} = \E{X}\E{P|P} = \mu P, &\V{Y|P} &= \V{XP|P} = P^{2} \V{X|P} = P \sigma^2 \\
  \E Y &= \mu p, & \V Y &= \E{\V{Y|P}} + \V{\E{Y|P}} = \sigma^2 p + \mu^{2}p(1-p), \\
  \E{S|N} &= N \E Y, & \V{S|N} &= N \V Y \\
  \E N &= 8 \lambda, & \V N &=  8\lambda.
\end{align*}
Now use BH.9.6.1. It's just a matter of filling in.
\end{solution}
\end{exercise}

\setcounter{theorem}{36}
\begin{exercise}
BH.9.37.

Bootstrapping is used in statistics to, for instance, construct confidence intervals. It is a much used and intuitive technique.

Extra exercise to  help you recall some ideas of Ch 1. How many different bootstrap samples are possible?

I used some extra ideas to save some time. We say that the rvs $\{X_i\}$ are independent and  distributed as the common rv~$X$ when $X_i\sim F_X$ where $F_X$ is the CDF of the rv $X$. Then $\E{X_{i}}=\E X$, and so on.  Next,  I prefer to write $Y_{j}= X_j^{*}$, as this writes (and types) faster. Finally,  it is easy to define
$Y_j = \sum_{i=1}^n X_i\1{S_j = i}$, where $S_{j}\sim \DUnif{\{1, \ldots, n\}}$ is the \(j\)th sample of the $\{X_{i}\}$.
 \begin{solution}

a. Here you should assume that the $X_i$ are not yet known. Thus, the expectation over $X_i$ is taken with respect to the CDF $F_X$. Using the independence of $X_j$ and $S_j$, $\1{S_j=i}\1{S_j=k} = 0$ if $i\neq k$, and that $\E{\1{S_j=k}} = 1/n$,
\begin{align*}
\E{Y_j} &= \sum_i \E{X_i}\E{\1{S_{j}=i}} = \mu, \\
\E{Y_j^2}  &= \E{\sum_k\sum_{l} X_k X_{l} \1{S_j=k}\1{S_{j}=l}}
= \E{\sum_k X_k^{2}  \1{S_j=k}} =\sum_{k} \E{X^{2}} n^{-1} = \E{X^{2}}, \\
\V{Y_j}&= \E{Y_j^{2}}- (\E{Y_{j}})^{2} = \sigma^{2}.
\end{align*}

b. Now we are given the outcomes (samples) $X_i=x_i$ of $n$ experiments.
I prefer to write $D = X_{1}, \ldots, X_{n}$ as it is shorter.
Noting that $S_j$ and $D$ are independent, and that $\E{X_{k}|D} = X_{k}$,
\begin{align*}
  \E{Y_{j}|D} = \sum_k X_k \E{\1{S_j=k}|D} = \frac 1 n \sum_k X_k := \bar X
\end{align*}
Observe that this average  need  not be the same as $\mu$!

The conditional variance. Since $S_j$ and $S_k$ are independent when $j\neq k$, it must be that $Y_j|D$ and $Y_k|D$ are also conditionally independent. Moreover, $\{Y_j|D\}$ are conditionally iid. Therefore,
\begin{align*}
\E{Y_{j}^{2}|D}
&= \E{\sum_k\sum_l X_kX_l\1{S_j=k}\1{S_j=l}|D} \\
&= \E{\sum_kX_k^{2}\1{S_j=k}|D}  = \sum_kX_k^{2}\E{\1{S_j=k}|D} \\
&= \frac 1 n \sum_{k} X_k^{2}, \\
\V{Y_j| D} &= \frac 1 n \sum_{k} X_k^{2} - (\bar X)^{2} = \frac 1 n \sum_{k} (X_k - \bar X)^{2} = \frac{n-1}n \sigma^{2},\\
\V{\bar Y|D} &= \V{\frac 1 n\sum_{j}Y_{j}|D} = \frac 1{n^{2}} \sum_j \V{Y_j|D} = \frac 1 n \V{Y_1|D}.
\end{align*}


c. For $\E{\bar Y}$ use linearity and Adam's law:
\begin{align*}
\E{\bar Y} = \E{\E{\bar Y|D}} = \frac 1 n \sum_{k}\E{X_{k}} = \E X = \mu.
\end{align*}


Here are the details for $\V{\bar Y}$. Using  BH.6.3.3 and BH.6.3.4,
\begin{align*}
\E{\V{\bar Y|D}} &=  \frac 1 n \E{\V{Y_{1}|D}} =
\frac 1 {n^2} \E{ \sum_{i=1}^n (X_i-\bar X)^2} \\
&= \frac{n-1} {n^2} \E{\frac 1 {n-1} \sum_{i=1}^n (X_i-\bar X)^2}
= \frac{n-1} {n^2} \E{S_{n}^2} = \frac{(n-1)\sigma^{2}} {n^2} \\
\V{\E{\bar Y|D}} &= \V{\bar X}= \frac 1{n^{2}} \sum_{i} \V{X_{i}} = \frac 1 n \sigma^{2}.
\end{align*}
Now use Eve's law to add both terms to get $V{\bar Y}$.


d. We add randomness twice, first we draw  samples to get $D$, and then we draw randomly from $D$.

The extra exercise: immediate from Example 1.4.22. We are not interested in the sequence of the bootstrap sample. BTW, the story that goes for me with this example is the `balls and bars story'. I have $n$ balls to distribute over $k$ boxes. Hence, there are $k-1$ bars to separate the boxes. For the bootstrap sample, I have to distribute $n$ bootstrap samples (the $X^*_{i}$) over $n$ boxes (the initial sample $X_i$.)

If $n$ is small, say $n=4$. Does it make sense to take more than 1000 bootstrap samples?
\end{solution}
\end{exercise}

\setcounter{theorem}{38}
\begin{exercise}
BH.9.39. There are numerous examples of rvs with  non-zero kurtosis, for instance,  claim sizes of car accidents, the time patients spend in hospital beds, finance. This exercise helps to understand how a positive kurtosis may originate.
\end{exercise}


\setcounter{theorem}{49}
\begin{exercise} BH.9.50.  We will also simulate this in an assignment.
\begin{hint}
a.
$N | \lambda \sim \Pois{\lambda}$.

b. Analogous to BH.9.6.1

c. and d. See BH.8.4.5.
\end{hint}
\begin{solution}
a. Using the hint gives us $\E{N\given \lambda} = \lambda$ and $\V{N\given \lambda}= \lambda$.

Now use Adam and Eve.

b. Just copy the formulas of BH.9.6.1

c. With the hint, observe that $\Exp(1)=\Gamma{1,1}$. In the relevant formula of BH.8.4.5 ($\P{Y=y}$), take $t=r_0=b_0=1$ and conclude that $\P{N=n}=2^{-n-1}$. Hence, $N\sim\Geo{1/2}$.

d. Same story. The relevant formula is $f_1(\lambda|y)$.
\end{solution}
\end{exercise}

\setcounter{theorem}{51}
\begin{exercise}
BH.9.52
\end{exercise}

\setcounter{theorem}{54}

\begin{exercise}
BH.9.55.  Suppose first you draw just one number per day, what is then the recursion? Then suppose you draw 2 numbers per day.

An interesting variation is to find a recursion for the number of \emph{draws} instead of \emph{days} are needed until all  numbers have been seen.
\end{exercise}

\setcounter{theorem}{55}
\begin{exercise}
BH.9.56.
\begin{hint}
Refresh your  knowledge of the Beta distributions.

a. Since we include the win, the number of games $T|p$ (since we assume $p$ given) must be $\sim\FS{p}$. Hence, $\E{T|p} = 1/p$

To get $\E T$ use Adam's law. Realize that you have to take the integral with respect to $p$!

b. $1+\E G$ is smaller than the expected time as computed in a. Why is this so?

c. The number of wins, conditional on $p$, out of $n$ is $X|p\sim\Bin{n,p}$. Then use Beta-Binomial conjugacy.

BTW, I find it easier to think about $f(p, X=k)$ instead of $f(p|X=k)$, since on the event $(p, X=k)$.
\begin{equation*}
f(p, X=k) \propto p^{a-1} q^{b-1} {n \choose k} p^k q^{n-k} \propto p^{a-1+k}q^{b-1+n-k}.
\end{equation*}
Then, as  $f(p|X=k) = f(p, X=k)/\P{X=k} \sim f(p, X=k)$ (because $\P{X=k}$ is just a constant) we get the same result op to a scaling factor. But we can use the reasoning of BH.8.3.3 to get the correct constant.
\end{hint}
\begin{solution}
a. From the hint,
\begin{align*}
\E{T}
&=\E{\E{T|p}} = \frac{1}{\beta(a,b)}\int_0^{1}\frac 1 p p^{a-1}(1-p)^{b-1} \d p \\
&= \frac{1}{\beta(a,b)}\int_0^{1} p^{a-2}(1-p)^{b-1} \d p
= \frac{\beta(a-1, b)}{\beta(a,b)} \\
&= \frac{a+b-1}{a-1} = 1 + \frac b {a-1}.
\end{align*}
To get the last equation,  use the definition of $\beta(a,b)$ in terms of factorials (see the Bayes' billiards story) to simplify. This is easy, many terms cancel.


b. Take $Y=1+G$, then $Y$ has the first success distribution since $G$ is geometric. Hence, $\E Y = (a+b)/a= 1 + b/a$. Clearly, this is smaller than $1+b/(a-1) = \E T$.

But why is this so?


I must miss something here.
The prior is $\Beta{a, b}$.
Then Beta-Binomial conjugacy story, we assume that Vishy won $a-1$ games, and lost $b-1$ games.
My guess for Vishy winning the next game would be $(a-1)/(a+b-2)$, not $a/(a+b)$.
But I make an error here. Check the BH problem 9.57. You'll see that we should indeed use $a/(a+b)$!  Tricky!

c.  Immediate from BH.8.3.3: $p|X=7 \sim\Beta{a+7, b+3}$.
\end{solution}
\end{exercise}



\setcounter{theorem}{56}
\begin{exercise}
BH.9.57
\begin{hint}
a. The prior of $p$ is uniform on $[0,1]$. But this is equal to $\Beta{1,1}$. Now use  Beta-Binomial conjugacy.

b. Write $S_n=\sum_{i=1}^nX_{i}$. What are $\P{X_{n+1}=1|p}$ and $\P{S_n=k|p}$?
\end{hint}
\begin{solution}
a. By  the hint,
\begin{equation*}
f(p|X_1=x_{1}) \propto f(p, X_1=x_{1}) \propto
p^{a-1} q^{b-1} p^{x_1}q^{1-x_1}
\propto p^{a+x_1-1} q^{b+(1-x_{1})-1}.
\end{equation*}
Hence, $p|X_1=x_1 \sim \Beta{a+x_1, b+(1-x_{1})}$. We can now use this as prior to see that  $p|X_1=x_{a}, X_{2}=x_{2} \sim \Beta{1+x_1+x_2, 1+ (1-x_1) + (1-x_2)}$, and so on. Hence, $p|X_{1}, \ldots X_{n} \sim \Beta{1+S_k, 1+n-S_{k}}$.

b. With the hint, $\P{X_{n+1}=1|p} = p$ and $\P{S_n=k|p} ={n \choose k} p^{k}q^{n-k} \propto p^{k}q^{n-k}$. Also $X_{n_+1}|p$ and $S_n|p$ are conditionally independent.
Therefore,
\begin{equation*}
\P{X_{n+1}=1, S_n=k | p} \propto p p^kq^{n-k} = p^{k+1} q^{n-k},
\end{equation*}
which in turn implies that
\begin{equation*}
\P{X_{n+1}=1| S_n=k, p} \propto = p^{k+1} q^{n-k}.
\end{equation*}
Hence, $X_{n+1}| S_n=k, p \sim \Beta{k+2, n-k+1}$.
Now, $X_{n+1}\in \{0, 1\}$, so that $\P{X_{n+1}=1| S_n=k, p} = \E{X_{n+1}| S_n=k, p} = (k+2)/(n+3)$, since $X_{n+1}| S_n=k, p \sim \Beta{k+2, n-k+1}$.

The last step is to realize that $\E{X_{n+1}|S_{n}=k} = \E{\E{X_{n+1}|S_{n}=k, p}|S_{n=k}}$.

Here is another way to get the same result.
\begin{align*}
\P{S_n=k} &= \frac 1 {n+1},\text{ by Bayes' billiard}, \\
\P{X_{n+1}=1,S_n=k} &=  \int_0^1 \P{X_{n+1}=1,S_n=k\given p} f(p) \d p
= \int_0^1 p {n \choose k} p^{k} (1-p)^{n-k} f(p) \d p \\
&= \frac{k+1}{n+1}\int_0^1  {n+1 \choose k+1} p^{k+1} (1-p)^{n-k} f(p) \d p \\
&= \frac{k+1}{n+1} \frac 1 {n+2}, \text{ again with Bayes' billiard}, \\
\P{X_{n+1}=1 \given S_n=k} &= \P{X_{n+1}=1,S_n=k}/\P{S_{n}=k}.
\end{align*}
Now simplify.
\end{solution}
\end{exercise}

\setcounter{theorem}{57}
\begin{exercise}
BH.9.58. In part c.\/ the prior is the uniform distribution. What would happen if you would take the prior of part b, i.e., $a$ out of $j$ wins?
\begin{hint}
a. Recall that the uniform distribution on $[0,1]$ is $\Beta{a,b}$ with $a=b=1$. I prefer to write $S_n=\sum_{j=1}^n X_j$. First compute $\E{S_n\given p}$. Then compute $\E{\E{S_n| p}}$. Note that the outer expectation is an integral with respect to $p$ and the density of $\Beta{1,1}$.

For the variance, use Eve's law.

b. Use Beta-Binomial conjugacy. Or use the insights of BH.9.56 and BH.9.57.

c. Bayes Billiards.
\end{hint}

\begin{solution}
\begin{align*}
\E{S_n|p} &= np \\
\E{p} &= \frac 1{\beta(a,b)}\int_0^1pp^{a-1}q^{b-1}\d p = \frac{\beta(a+1,b)}{\beta(a,b)} = \frac a{a+b} = 1/2\\
\E{\E{S_n|p}} &= n\E{p} = n/2. \\
\V{S_n|p}  &= np q\\
\E{\V{S_n|p}}  &= n\E{p q} = n \E p - n \E{p^{2}} = n/2 - n\E{p^{2}}\\
\E{p^2} &=  \frac 1{\beta(a,b)}\int_0^1p^{2}p^{a-1}q^{b-1}\d p =
\frac{\beta(a+2,b)}{\beta(a,b)} = \frac{a(a+1)}{(a+b)(a+b+1)} = \frac 2{2\cdot 3} = 1/3\\
\V{\E{S_{n}|p}} &= \V{n p} = n^{2} \V p = n^{2}/12.
\end{align*}
The rest of Eve's law is now trivial.

b. We start with a $\Beta{1,1}$ prior on $p$. After the first win, the prior gets updated to $\Beta{1+1, 1}$, after a loss to $\Beta{1, 1+1}$. Reasoning like this, after $a$ wins and $j-a$ losses, the distribution for a win becomes $\Beta{1+a, a+ j-1}$. Therefore, by using the hint in the book,  $\E{p | S_j=a } = (a+1)/(j+2)$.

c. When somebody doesn't give me any information about what team can win, then any outcome must be equally likely. (What else can it be?)
This is also my way to understand the expression in BH.8.3.2. Hence, $\P{X=k} = 1/(n+1)$. Observe that we use the prior $p\sim\Beta{1,1}$.

When the prior is $\Beta{a, j-a}$, we should get the negative hypergeometric distribution, see the remark  in BH.8.3.3.

d. Shanille scores the first and missed the second. Hence, there are 98 shots left, out which she has to score $49$.  Thus, we ask for $\P{S_{98}=49|p}$, where $p\sim \Beta{a=1,b=1}$ is the prior since she hit $a=1$  out of $a+b=2$ shots. This places us in the situation of part c above, with $n=98$. Hence, $\P{S_{98}=49|p} = 1/99$.
\end{solution}
\end{exercise}




\end{document}


%%% Local Variables:
%%% mode: latex
%%% TeX-master: "study-guide.tex"
%%% End:
