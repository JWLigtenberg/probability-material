\documentclass[study-guide-sol]{subfiles}
\externaldocument{study-guide}

\opt{solutionfiles,check}{
\Opensolutionfile{hint}
\Opensolutionfile{ans}
}

\begin{document}


\setcounter{theorem}{19}

\begin{exercise}[BH.2.48]
  A fair die is rolled repeatedly, and a running total is kept (which is, at each time, the total of all the rolls up until that time). Let $p_n$ be the probability that the running total is ever \emph{exactly} $n$ (assume the die will always be rolled enough times so that the running total will eventually exceed $n$, but it may or may not ever equal $n$).
	\begin{enumerate}
		\item  Write down a recursive equation for $p_n$ (relating $p_n$ to earlier terms $p_k$ in a simple way). Your equation should be true for all positive integers $n$, so give a definition of $p_0$ and $p_k$ for $k < 0$ so that the recursive equation is true for small values of $n$.
		\item Find $p_7$.
	\end{enumerate}
\begin{solution}
	\begin{enumerate}
		\item Suppose, we current have $n-i$ for $i=1,\ldots,6$, then we have a probability of $\frac{1}{6}$ of achieving $p_{n}$. So
		\begin{align*}
			p_{n} = \frac{1}{6}\sum_{i=1}^{6}p_{n-i}
		\end{align*}
		Assuming we start at 0, it is reasonable to set $p_{0}=1$ and $p_{k}=0$ for $k<0$. 
		\item
		\begin{align*}
			p_{1} &= \frac{1}{6},\\
			p_{2} &= \frac{1}{6}(p_{0}+p_{1}) =\frac{1}{6}\bigg(1+\frac{1}{6}\bigg),\\
			p_{3} &= \frac{1}{6}(p_{0}+p_{1}+p_{2}) = p_{2}(1+\frac{1}{6})=\frac{1}{6}\bigg(1+\frac{1}{6}\bigg)^2,\\
			p_{4} & = \frac{1}{6}(p_{0}+p_{1}+p_{2}+p_{3})=p_{3}\bigg(1+\frac{1}{6}\bigg) = \frac{1}{6}\bigg(1+\frac{1}{6}\bigg)^3,\\
			p_{5} & = \frac{1}{6}(p_{0}+p_{1}+p_{2}+p_{3} + p_{4})=p_{4}\bigg(1+\frac{1}{6}\bigg) = \frac{1}{6}\bigg(1+\frac{1}{6}\bigg)^4,\\
			p_{6} & = \frac{1}{6}(p_{0}+p_{1}+p_{2}+p_{3}+p_{4}+p_{5})=p_{5}\bigg(1+\frac{1}{6}\bigg) = \frac{1}{6}\bigg(1+\frac{1}{6}\bigg)^5,\\
			p_{7} & = \frac{1}{6}(p_{1}+p_{2}+p_{3}+p_{4}+p_{5}+p_{6})=p_{6}\bigg(1+\frac{1}{6}\bigg)-\frac{1}{6}p_{0}  = \frac{1}{6}\bigg(1+\frac{1}{6}\bigg)^6-\frac{1}{6}.\\	
		\end{align*}
		In decimals, $p_{7}=0.2536$.
	\end{enumerate}
\end{solution}
\end{exercise}
\end{document}

