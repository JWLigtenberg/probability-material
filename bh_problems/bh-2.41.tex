\documentclass[study-guide-sol]{subfiles}
\externaldocument{study-guide}

\opt{solutionfiles,check}{
\Opensolutionfile{hint}
\Opensolutionfile{ans}
}

\begin{document}


\setcounter{theorem}{18}
\begin{exercise}[BH.2.41]
  The ratings of Monty Hall's show have dropped slightly, and a panicking executive producer complains to Monty that the part of the show where he opens a door lacks suspense: Monty always opens a door with a goat. Monty replies that the reason is so that the game is never spoiled by him revealing the car, but he agrees to update the game as follows.
	
	Before each show, Monty secretly flips a coin with probability $p$ of Heads. If the coin lands Heads, Monty resolves to open a door with a goat (with equal probabilities if there is a choice). Otherwise, Monty resolves to open a random door, with equal probabilities. Of course, Monty will not open the door that the contestant initially chooses. The contestant knows $p$ but does not know the outcome of the coin flip. When the show starts, the contestant chooses a door. Monty (who knows where the car is) then opens a door. If the car is revealed, the game is over; if a goat is revealed, the contestant is offered the option of switching. Now suppose it turns out that the contestant chooses door 1 and then Monty opens door 2, revealing a goat. What is the contestant's probability of success if they switch to door 3?
\begin{solution}
	\begin{enumerate}
		\item It's annoying that we do not know whether Monty flipped Heads $H$ or Tails, so we are going to condition on this event.  
	
		Let $C_{i}$ be the event that the car is behind door $i$ and $G_{i}$ the event there is a goat behind door $i$. Denote by $D_j$ the event that Monty opens door $j$. We now apply the LOTP
		\begin{align*}
			P(C_{3}|D_{2},G_{2})& = P(C_{3}|H,D_{2},G_{2})P(H|D_{2},G_{2}) + P(C_{3}|H^{C},D_{2},G_{2})P(H^{C}|D_{2},G_{2})\\
			& = \frac{2}{3}\cdot P(H|D_{2},G_{2}) + P(C_{3}|H^{C},D_{2},G_{2})P(H^{C}|D_{2},G_{2})
		\end{align*}
		The second line uses the result from the standard Monty Hall problem that the probability of winning the car given the switching strategy is $2/3$. Now, we use the definition of conditional probability
		\begin{align*}
			P(C_{3}|H^{C},D_{2},G_{2}) &=\frac{P(D_{2},G_{2}|C_{3},H^{C})P(C_{3}|H^{C})}{P(D_{2},G_{2}|H^{C})}=\frac{\frac{1}{2}\cdot \frac{1}{3}}{\frac{1}{2}\cdot \frac{2}{3}}=\frac{1}{2}.
		\end{align*}
		Here we used that $P(D_{2},G_{2}|C_{3},H^{C})=\frac{1}{2}$, since (1) if there is a car behind door 3, then there is a goat behind door 2, so $P(D_{2},G_{2}|C_{3},H^{C})=P(D_{2}|C_{3},H^{C})$ and (2) Monty is randomly opening one of the two doors, so $P(D_{2}|C_{3},H^{C})=P(D_{2}|H^{C})=\frac{1}{2}$. Also $P(C_{3}|H^{C})=P(C_{3})=\frac{1}{3}$, since the location of the car is independent of whether Monty opens doors at random or only opens doors with goats. Finally $P(D_{2},G_{2}|H^{C})=P(D_{2}|H^{C})P(G_{2}|H^{C})=\frac{1}{2}\cdot \frac{1}{3}$, since Monty randomly opens a door, this is independent of whether there is a goat there.

		The only unknown quantity that remains is
		\begin{align*}
			P(H|D_{2},G_{2})& = \frac{P(D_{2},G_{2}|H)P(H)}{P(D_{2},G_{2}|H)P(H)+P(D_{2},G_{2}|H^C)P(H^C)} \\
			&= \frac{(P(D_{2},G_{2}|C_{1},H)P(C_{1}|H)+P(D_{2},G_{2}|C_{3},H)P(C_{3}|H))P(H)}{P(D_{2},G_{2}|H)P(H)+P(D_{2},G_{2}|H^C)P(H^C)}\tag*{(LOTP)} \\
			&= \frac{(\frac{1}{2}\cdot \frac{1}{3}+\frac{1}{3})p}{(\frac{1}{2}\cdot \frac{1}{3}+\frac{1}{3})p+(\frac{1}{2}\cdot \frac{2}{3})(1-p)}
		\end{align*}
		where we used that $P(D_{2},G_{2}|C_{1},H)=\frac{1}{2}$ (if there is a car behind door 1, Monty decides between the two remaining doors with probability $\frac{1}{2}$). Also, $P(C_{1}|H)=P(C_{3}|H)=P(C_{1})=\frac{1}{3}$, and $P(D_{2},G_{2}|C_{3},H)=P(D_{2}|C_{3},H)=1$ (note that the term involving $C_{2}$ not appearing when applying the LOTP, why?). 

		Substituting all of this into the first equation of the answer gives
		\begin{align*}
			P(C_{3}|D_2,G_2) = \frac{1+p}{2+p}.
		\end{align*}
	\end{enumerate}
\end{solution}
\end{exercise}
\end{document}

