\documentclass[study-guide-sol]{subfiles}
\externaldocument{study-guide}

\opt{solutionfiles,check}{
\Opensolutionfile{hint}
\Opensolutionfile{ans}
}

\begin{document}


\setcounter{theorem}{8}
\begin{exercise}[BH 7.9.]
We'll develop a simulation for this in the assignments.
\begin{hint}
a. $\P{X=i, Y=j, N=n} = \P{X=i, Y=j} \1{i+j=n}$.

c. $\P{X=i\given N=n} = 1/(n+1)$. Why is this uniform?
\end{hint}
\begin{solution}
a. The hint is the solution.

b. Use the hint for a. Then, for $k=0, 1, \ldots, n$,
\begin{equation*}
\P{X=k, N=n} = \P{X=k, Y=n-k} = pq^k p q^{n-k} = p^2 q^{n}.
\end{equation*}
(Have we used independence somewhere?)

c. Observe that the right hand side does not depend on $k$. This implies that $\P{X=k|N=n}$  also does not depend on $k$. (Why?) But, since $\P{X=k|N=n}$ is a true PMF,  is must be that $\sum_{k=0}^{n}\P{X=k|N=n}$ adds up to $1$. These two ideas put together imply that
$\P{X=k|N=n} = 1/(n+1)$.

With Bayes' expression, and using that  $\P{X=k|N=n} = 1/(n+1)$,
\begin{equation*}
\P{X=k|N=n} = \frac{\P{X=k, N=n}}{\P{N=n}},
\end{equation*}
it follows that
\begin{equation*}
\P{N=n} = \frac{\P{X=k, N=n}}{\P{X=k|N=n}} = \frac{p^2q^n}{1/(n+1)} = (n+1) p^2q^{n}.
\end{equation*}
\end{solution}
\end{exercise}


\end{document}

