\documentclass[study-guide-sol]{subfiles}
\externaldocument{study-guide}

\opt{solutionfiles,check}{
\Opensolutionfile{hint}
\Opensolutionfile{ans}
}

\begin{document}

\setcounter{theorem}{15}
\begin{exercise}[BH.1.31]
	Elk dwell in a certain forest. There are $N$ elk, of which a simple random sample of size $n$ are captured and tagged (``simple random sample" means that all $\binom{N}{n}$ sets of $n$ elk are equally likely). The captured elk are returned to the population, and then a new sample is drawn, this time with size $m$. This is an important method that is widely used in ecology, known as capture-recapture. What is the probability that exactly $k$ of the $m$ elk in the new sample were previously tagged? (Assume that an elk that was captured before doesn’t become more or less likely to be captured again.)
\begin{solution}
	After the first round, the are $n$ elk that have been captured. Suppose these are tagged. In the second round, we capture $m$ elk. Denote by $A_{k}$ the probability that exactly $k$ of these $m$ elk are tagged. The total number of ways to select $m$ elk is ${N \choose m}$. Selecting $k$ tagged elks and $m-k$ non-tagged elks can be done in ${n \choose k}\cdot {N-n \choose m-k}$ ways. Then, $$P(A_k) = \frac{{n \choose k}\cdot {N-n \choose m-k}}{{N \choose m}}.$$
\end{solution}
\end{exercise}

\end{document}

