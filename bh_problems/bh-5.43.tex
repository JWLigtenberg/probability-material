\documentclass[study-guide-sol]{subfiles}
\externaldocument{study-guide}

\opt{solutionfiles,check}{
\Opensolutionfile{hint}
\Opensolutionfile{ans}
}

\begin{document}


\setcounter{theorem}{42}
\begin{exercise}[BH.5.43] The Exponential is the analog of the Geometric in continuous time. This problem explores the connection between Exponential and Geometric in more detail, asking what happens to a Geometric in a limit where the Bernoulli trials are performed faster and faster but with smaller and smaller success probabilities.
	
	Suppose that Bernoulli trials are being performed in continuous time; rather than only thinking about first trial, second trial, etc., imagine that the trials take place at points on a timeline. Assume that the trials are at regularly spaced times $0$, $\Delta t$, $2 \Delta t$, ..., where $\Delta t$ is a small positive number. Let the probability of success of each trial be $\lambda \Delta t$, where $\lambda$ is a positive constant. Let $G$ be the number of failures before the first success (in discrete time), and $T$ be the time of the first success (in continuous time).
	
	\begin{enumerate}
		\item Find a simple equation relating $G$ to $T$.
		\item Find the CDF of $T$.
		\item Show that as $\Delta t \to 0$, the CDF of $T$ converges to the $\text{Expo}(\lambda)$ CDF, evaluating all the CDFs at a fixed $t \geq 0$.
	\end{enumerate}
\begin{solution}
    \begin{enumerate}
	    \item Assume the first trial happens at time position $0$, then the $j$th trial would occur at $(j-1)\Delta t$. Thus
    	\[T=((G+1) - 1)\Delta t\] 
    	since the first success trial would be the $(G+1)$th trial.  
    	\item For $t\geq 0$
    	\begin{align*}
    		\mathbb{P}\left(T>t\right) &= \mathbb{P}\left(G> \lfloor t /(\Delta t)\rfloor \right) = \left( 1-\lambda \Delta t \right)^{\lfloor t /(\Delta t)\rfloor +1 }
    	\end{align*}
    	where $\lfloor x \rfloor = \max\{n \in \mathbb{N}: n\leq x \}$. Thus for $t\geq 0$
    	\begin{align*}
    		\mathbb{P}\left(T\leq t\right) &= 1- \left( 1-\lambda \Delta t \right)^{\lfloor t /(\Delta t)\rfloor +1 }
    	\end{align*}
    	\textbf{For $t<0, \mathbb{P}\left(T\leq t\right)=0$.
    	}
        \item From 
    	$\left(t /(\Delta t) -1 \right)\geq \lfloor t /(\Delta t)\rfloor \geq t /(\Delta t)$
    	we know 
    	\begin{align}
    		\left( 1-\lambda \Delta t \right)^{  t /(\Delta t) }/\left( 1-\lambda \Delta t \right) \geq \left( 1-\lambda \Delta t \right)^{\lfloor t /(\Delta t)\rfloor +1 } \geq \left( 1-\lambda \Delta t \right)^{ t /(\Delta t) +1} \label{eq543}
    	\end{align}
    	Note that $\lim\limits_{\Delta t\rightarrow 0}\left( 1-\lambda \Delta t \right)=1$ and 
    	\begin{align*}
    		\lim\limits_{\Delta t\rightarrow 0}\left[\left( 1-\lambda \Delta t \right)^{ -1 /(\lambda \Delta t) }\right]^{-\lambda t} = e^{-\lambda t}
    	\end{align*}
    	Therefore, taking limits on the both sides of equation (\ref{eq543}) we know $ \left( 1-\lambda \Delta t \right)^{\lfloor t /(\Delta t)\rfloor +1 }$ converges to $e^{-\lambda t}$ and thus $\mathbb{P}(T\leq t)$ converges to $1-e^{-\lambda t}$ (the Expo($\lambda$) CDF). 
	\end{enumerate}
\end{solution}
\end{exercise}

\end{document}

