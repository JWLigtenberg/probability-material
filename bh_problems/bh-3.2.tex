\documentclass[study-guide-sol]{subfiles}
\externaldocument{study-guide}

\opt{solutionfiles,check}{
\Opensolutionfile{hint}
\Opensolutionfile{ans}
}

\begin{document}

\setcounter{theorem}{1}
\begin{exercise}[BH.3.2]
	\begin{enumerate}
		\item Independent Bernoulli trials are performed, with probability 1/2 of success, until there has been at least one success. Find the PMF of the number of trials performed.
		\item Independent Bernoulli trials are performed, with probability 1/2 of success, until there has been at least one success and at least one failure. Find the PMF of the number of trials performed.
	\end{enumerate}
\begin{solution}~
	\begin{enumerate}
		\item Denote by $X$ the random variable that is equal to $i$ if the first success takes place on the $i$th trial. We need to find the PMF of $X$. Furthermore, denote by $F_{k-1}$ the number of failures up to the $k$th trial and by $S_{k}$ the event that there is a success on the $k$th trial. Notice that the trials are independent, so $F_{k-1}$ is independent of $S_{k}$. Since the probability of success is 1/2, a success on the first trial happens with probability
		\begin{align*}
			P(X=1) = \frac{1}{2}.
		\end{align*}
		For the first success to happen on the second trial, it cannot happen on the first trial, so
		\begin{align*}
			P(X=2) = P(F_{1}\cap S_{2}) = P(F_{1})P(S_{2}) = \frac{1}{2^2},
		\end{align*}
		where the second equality uses independence between $F_{1}$ and $S_{2}$.
		Continuing in this fashion, we find
		\begin{align*}
			p_{X}(k)=	P(X=k) = \frac{1}{2^k}\quad k=1,\ldots,\infty,
		\end{align*}
		and $P(X=k)=0$ otherwise. We can quickly check whether this is a valid PMF. It is clear that $p_{X}(k)\geq 0$ for all $k$. Also,
		\begin{align*}
			\sum_{k=-\infty}^{\infty}p_{X}(k) &= \sum_{k=1}^{\infty}\frac{1}{2^k}\\
			&= -1+\sum_{k=0}^{\infty}\frac{1}{2^k}\\
			&= -1+\frac{1}{1-\frac{1}{2}}\\
			&=-1+2 =1.
		\end{align*}
		We conclude that the provided PMF is a valid PMF.
		\item Denote by $Y$ the random variable that is equal to $i$ if the missing event (success of failure) occurs on the $i$th trial. Notice that after the first trial, we are in the situation of a. Suppose the outcome of the first trial is success ($S_{1}$). Then $P(Y=2|S_{1}) = P(X=1), P(Y=3|S_{1}) = P(X=2)$, etcetera. Using the law of total probability,
		\begin{align*}
			P(Y=2) &= P(Y=2|S_{1})P(S_{1}) + P(Y=2|S_{1}^{C})P(S_{1}^{C})\\
			&=\frac{1}{2}(P(X=1) + P(X=1)) = P(X=1) = \frac{1}{2}.
		\end{align*}
		This reasoning actually holds for all subsequent trials as well, so we find that 
		\begin{align*}
			p_Y(k)=P(Y=k) &= P(Y=k|S_{1})P(S_{1}) + P(Y=k|S_{1}^C)P(S_{1}^C)\\
			& = \frac{1}{2}(P(X=k-1) + P(X=k-1))\\
			& = P(X=k-1)\\
			&= \frac{1}{2^{k-1}}\quad k = 2,3,\ldots		
		\end{align*}
		and $p_{Y}(k)=0$ otherwise.
	\end{enumerate}
\end{solution}
\end{exercise}

\end{document}

