\documentclass[study-guide-sol]{subfiles}
\externaldocument{study-guide}

\opt{solutionfiles,check}{
\Opensolutionfile{hint}
\Opensolutionfile{ans}
}

\begin{document}


\setcounter{theorem}{13}
\begin{exercise}[BH.5.14] 
\begin{solution}
    Note that $X \leq x$ holds if and only if all of the $U_j$ 's are at most $x$. So the CDF of $X$ is
		$$
		P(X \leq x)=P\left(U_1 \leq x, U_2 \leq x, \ldots, U_n \leq x\right)=\left(P\left(U_1 \leq x\right)\right)^n=x^n,
		$$
		for $0<x<1$ (and the CDF is 0 for $x \leq 0$ and 1 for $x \geq 1$ ), where we make use of the fact that $U_j$ 's are i.i.d. in the second last equality. The conventional PDF of $X$ thus is
		$$
		f(x)=n x^{n-1},
		$$
		for $0<x<1$ (and 0 otherwise). Then
		$$
		E X=\int_0^1 x\left(n x^{n-1}\right) d x=n \int_0^1 x^n d x=\frac{n}{n+1} .
		$$ 
\end{solution}
\end{exercise}

\end{document}

