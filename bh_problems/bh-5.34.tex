\documentclass[study-guide-sol]{subfiles}
\externaldocument{study-guide}

\opt{solutionfiles,check}{
\Opensolutionfile{hint}
\Opensolutionfile{ans}
}

\begin{document}


\setcounter{theorem}{33}
\begin{exercise}[BH.5.34] Let $Z \sim \mathcal{N}(0,1)$. A measuring device is used to observe $Z$, but the device can only handle positive values, and gives a reading of 0 if $Z \leq 0$; this is an example of censored data. So assume that $X=Z I_{Z>0}$ is observed rather than $Z$, where $I_{Z>0}$ is the indicator of $Z>0$. Find $E(X)$ and $\operatorname{Var}(X)$.
\begin{solution}
    By LOTUS,
		$$
		E(X)=\frac{1}{\sqrt{2 \pi}} \int_{-\infty}^{\infty} I_{z>0} z e^{-z^2 / 2} d z=\frac{1}{\sqrt{2 \pi}} \int_0^{\infty} z e^{-z^2 / 2} d z .
		$$
		Letting $u=z^2 / 2$, we have
		$$
		E(X)=\frac{1}{\sqrt{2 \pi}} \int_0^{\infty} e^{-u} d u=\frac{1}{\sqrt{2 \pi}} .
		$$
		Note that
		$$
		E\left(X^2\right)=\frac{1}{\sqrt{2 \pi}} \int_0^{\infty} z^2 e^{-z^2 / 2} d z=\frac{1}{2} \frac{1}{\sqrt{2 \pi}} \int_{-\infty}^{\infty} z^2 e^{-z^2 / 2} d z=\frac{1}{2},
		$$
		since a $\mathcal{N}(0,1)$ r.v. has variance 1 . Thus, 
		$$
		\operatorname{Var}(X)=E\left(X^2\right)-(E X)^2=\frac{1}{2}-\frac{1}{2 \pi} .
		$$
		\textit{Remark:  Note that $X$ is neither purely discrete nor purely continuous, since $X=0$ with probability $1 / 2$ and $P(X=x)=0$ for $x \neq 0$. So $X$ has neither a PDF nor a PMF; but LOTUS still works (since we are working with $g(X)$ with $X$ being continuous), which actually works under very general conditions, allowing us to work with the PDF of $Z$ to study expected values of functions of $Z$.}
\end{solution}
\end{exercise}

\end{document}

