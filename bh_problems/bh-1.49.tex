\documentclass[study-guide-sol]{subfiles}
\externaldocument{study-guide}

\opt{solutionfiles,check}{
\Opensolutionfile{hint}
\Opensolutionfile{ans}
}

\begin{document}

\setcounter{theorem}{22}
\begin{exercise}[BH.1.49]
	A fair die is rolled $n$ times. What is the probability that at least 1 of the 6 values never appears?
\begin{solution}
	Denote by $A_{j}$ the event that the value $j$ never appears in $n$ throws. The probability of $A_{i}$ is $(5/6)^n$ (since we have 5 out of six options that are allowed to appear). The probability that $i$ and $j$ never appear, i.e. $A_{i}\cap A_{j}$ is $(4/6)^n$ (since there are now only 4 out of six options that are allowed to appear). This reasoning can be continued for the higher order intersections. Now by the inclusion exclusion principle
	$$P(A_{1}\cup\ldots\cup A_{6}) = {6 \choose 1}(5/6)^n - {6 \choose 2}(4/6)^n + {6\choose 3}(3/6)^n\ldots +{6 \choose 5}(1/6)^n.$$ 
 
\end{solution}
\end{exercise}

\end{document}

