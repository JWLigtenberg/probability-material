\documentclass[study-guide-sol]{subfiles}
\externaldocument{study-guide}

\opt{solutionfiles,check}{
\Opensolutionfile{hint}
\Opensolutionfile{ans}
}

\begin{document}

\setcounter{theorem}{8}

\begin{exercise}[BH.12]
	Four players, named $A$, $B$, $C$, and $D$, are playing a card game. A standard, well-shuffled deck of cards is dealt to the players (so each player receives a 13-card hand).
	\begin{enumerate}
		\item How many possibilities are there for the hand that player A will get? (Within a hand, the order in which cards were received doesn’t matter.)
		\item How many possibilities are there overall for what hands everyone will get, assuming that it matters which player gets which hand, but not the order of cards within a hand?
		\item Explain intuitively why the answer to Part (b) is not the fourth power of the answer to Part (a).
	\end{enumerate}
\begin{solution}~
	\begin{enumerate}
		\item There are 52 cards, 13 of which are dealt to Player A. The number of possible hands (when ordering within the hand is not important) is $N_{A}={52 \choose 13}$.
	 	\item First deal to Player A. Following the answer to a., there are $N_{A}={52 \choose 13}$ options. Then deal to Player B. There are 39 cards left, of which this player will receive 13. This can be done in $N_{B} ={39 \choose 13}$ ways. Then deal to Player C. There are 26 cards left, of which this player will receive 13. This can be done in $N_{C}={26\choose 13}$ ways. We give the remaining 13 cards to Player D. In total, we have $$N = N_{A}N_{B}N_{C} = {52 \choose 13}{39 \choose 13}{26 \choose 13}$$ possibilities.
	 	\item If we would put the 13 cards that Player A received back into the deck before dealing to Player B, then this would be the answer.
	\end{enumerate}
\end{solution}
\end{exercise}

\end{document}

