\documentclass[study-guide-sol]{subfiles}
\externaldocument{study-guide}

\opt{solutionfiles,check}{
\Opensolutionfile{hint}
\Opensolutionfile{ans}
}

\begin{document}

\setcounter{theorem}{3}
\begin{exercise} [BH.6.4] Let $X \sim \operatorname{Bin}(n, p)$.
	\begin{enumerate}
		\item For $n=5, p=1 / 3$, find all medians and all modes of $X$. How do they compare to the mean?
	\item For $n=6, p=1 / 3$, find all medians and all modes of $X$. How do they compare to the mean?
	\end{enumerate}

\begin{solution}
    \begin{enumerate}
        \item   From the PMF we can see there are 2 modes:   1 and 2 . We have
        $$
        P(X \leq 2) \approx 0.790, P(X \geq 2)=1-P(X \leq 1) \approx 0.539,
        $$
        so 2 is a median. It is the unique median since for $x<2$,
        $$
        P(X \leq x) \leq P(X \leq 1) \approx 0.461
        $$
        and for $x>2$,
        $$
        P(X \geq x)=1-P(X<x) \leq 1-P(X \leq 2) \approx 0.210 .
        $$
        The mean is $5 / 3$, which is less than the median and in between the two modes.
        \item There is a unique mode at 2 . We have
        $$
        P(X \leq 2) \approx 0.680
        $$
        and
        $$
        P(X \geq 2)=1-P(X \leq 1) \approx 0.649,
        $$
        so 2 is a median. It is the unique median since for $x<2$,
        $$
        P(X \leq x) \leq P(X \leq 1) \approx 0.351
        $$
        and for $x>2$,
        $$
        P(X \geq x)=1-P(X<x) \leq 1-P(X \leq 2) \approx 0.320 .
        $$
        The mean is also $6 / 3=2$. So the mean, median, and mode are all equal to 2 here.
    \end{enumerate}
\end{solution}
\end{exercise}
\end{document}

