\documentclass[study-guide-sol]{subfiles}
\externaldocument{study-guide}

\opt{solutionfiles,check}{
\Opensolutionfile{hint}
\Opensolutionfile{ans}
}

\begin{document}


\setcounter{theorem}{16}
\begin{exercise}[BH.2.35]
  You are going to play 2 games of chess with an opponent whom you have never played against before (for the sake of this problem). Your opponent is equally likely to be a beginner, intermediate, or a master. Depending on which, your chances of winning an individual game are 90\%, 50\%, or 30\%, respectively.
	\begin{enumerate}
		\item What is your probability of winning the first game?
		\item Congratulations: you won the first game! Given this information, what is the probability that you will also win the second game (assume that, given the skill level of your opponent, the outcomes of the games are independent)?
		\item Explain the distinction between assuming that the outcomes of the games are independent and assuming that they are conditionally independent given the opponent's skill level. Which of these assumptions seems more reasonable, and why?
	\end{enumerate}
\begin{solution}~
	\begin{enumerate}
		\item The probability of winning the first game is found by applying the LOTP as follows,
		\begin{align*}
			P(W_{1}) &= P(W_{1}|B)P(B) + P(W_{1}|I)P(I) + P(W_{1}|M)P(M)\\
			& = \frac{9}{10}\cdot\frac{1}{3} + \frac{1}{2}\cdot\frac{1}{3} + \frac{3}{10}\cdot\frac{1}{3}\\
			& = \frac{17}{30}.
		\end{align*}
		\item Denote by $B$, $I$, $M$ the events of having a beginner, intermediate or master opponent, respectively.
		\begin{align*}
			P(W_{2}|W_{1}) &= P(W_{2}|W_{1},B)P(B|W_{1}) + P(W_{2}|W_{1},I)P(I|W_{1}) + P(W_{1}|W_{1},M)P(M|W_{1})\\
			&= P(W_{2}|B)P(B|W_{1}) + P(W_{2}|I)P(I|W_{1}) + P(W_{1}|M)P(M|W_{1})\tag*{(\text{Using that given skill, outcomes are independent})}
		\end{align*}
		Now, using Bayes' rule
		\begin{align*}
			P(B|W_{1})&=\frac{P(W_{1}|B)P(B)}{P(W_{1})}=\frac{9}{17},\\
			P(I|W_{1})&=\frac{P(W_{1}|I)P(I)}{P(W_{1})}=\frac{5}{17},\\
			P(M|W_{1})&=\frac{P(W_{1}|M)P(M)}{P(W_{1})}=\frac{3}{17}.
		\end{align*}
		We can conclude that
		\begin{align*}
			P(W_{2}|W_{1}) &= \frac{9}{10}\frac{9}{17} + \frac{1}{2}\frac{5}{17}+\frac{3}{10}\frac{3}{17}\\
			& = \frac{23}{34}.
		\end{align*}
		Now, look back at your answer to (a). There you found that winning the first game had probability 0.567. Given that you won the first round, you would expect the probability of winning the second round to go up. Indeed, the answer to (b) is 0.676.
		\item Knowing the outcome of one of the matches, we adjust our beliefs on the quality of our opponent. This changes our beliefs on the outcome of the other match. However, if the quality of the opponent is fixed, then outcome of the first match does not affect the outcome of the second match.  
	\end{enumerate}
\end{solution}
\end{exercise}
\end{document}

