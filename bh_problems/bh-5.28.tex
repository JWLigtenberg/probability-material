\documentclass[study-guide-sol]{subfiles}
\externaldocument{study-guide}

\opt{solutionfiles,check}{
\Opensolutionfile{hint}
\Opensolutionfile{ans}
}

\begin{document}


\setcounter{theorem}{27}
\begin{exercise}[BH.5.28] Let $Y \sim \mathcal{N}\left(\mu, \sigma^2\right)$. Use the fact that $P(|Y-\mu|<1.96 \sigma) \approx 0.95$ to construct a random interval $(a(Y), b(Y))$ (that is, an interval whose endpoints are r.v.s), such that the probability that $\mu$ is in the interval is approximately $0.95$. This interval is called a confidence interval for $\mu$; such intervals are often desired in statistics when estimating unknown parameters based on data.
\begin{solution}
    $$P(|Y-\mu|<1.96 \sigma)=P(-1.96 \sigma<Y-\mu<1.96 \sigma)=P(Y-1.96 \sigma<\mu<Y+1.96 \sigma)
    $$
    shows that the random interval $(Y-1.96 \sigma, Y+1.96 \sigma)$ is as desired.
\end{solution}
\end{exercise}

\end{document}

