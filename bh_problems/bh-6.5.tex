\documentclass[study-guide-sol]{subfiles}
\externaldocument{study-guide}

\opt{solutionfiles,check}{
\Opensolutionfile{hint}
\Opensolutionfile{ans}
}

\begin{document}

\setcounter{theorem}{4}
\begin{exercise} [BH.6.5] 
\begin{solution}
    The PMF is equal for all points in $\{1,2,\cdots,n \}$ and thus they are all modes. \\~\\
	For n is odd, we have 
	\begin{align*}
		\mathbb{P}\left(X\leq (n+1)/2 \right) = 1/2 + 1/(2n) \geq 1/2;~
		\mathbb{P}\left(X\geq (n+1)/2 \right) = 1/2 +1/(2n) \geq 1/2;~
	\end{align*}
	For $m<(n+1)/2$, $\mathbb{P}\left(X\leq m \right)\leq 1/2 + 1/(2n) -1/n <1/2 $; for $mm(n+1)/2$, $\mathbb{P}\left(X\geq m \right)\leq 1/2 + 1/(2n) -1/n <1/2 $. Therefore, $ (n+1)/2$ is the unique median. \\~\\
	For n is even, we have 
	\begin{align*}
		\mathbb{P}\left(X\leq n/2 \right) = 1/2  \geq 1/2;~
		\mathbb{P}\left(X\geq (n+2)/2 \right) = 1/2 + 1/(2n)  \geq 1/2;~
	\end{align*}
	which shows that any value between n/ 2 and ( n + 2) / 2 (inclusive) is also a median. \\~\\
	For $m<n/2$, $\mathbb{P}\left(X\leq m \right)  <1/2 $; for $m>(n+2)/2$, $\mathbb{P}\left(X\geq m \right) <1/2 $. Therefore, there are no other medians.
\end{solution}
\end{exercise}
\end{document}

