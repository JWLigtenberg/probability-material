\documentclass[study-guide-sol]{subfiles}
\externaldocument{study-guide}

\opt{solutionfiles,check}{
\Opensolutionfile{hint}
\Opensolutionfile{ans}
}

\begin{document}

\setcounter{theorem}{21}
\begin{exercise} BH.6.22

\begin{hint}
Use that the mgf of the sum of \emph{independent} rvs is the product of the mgfs.  Realize that  $X=\sum_j \1{A_j} = \sum Y_{j}$ when we take $Y_j=\1{A_{j}}$. Then, what is $M_{Y_j}(s)$? Simplify.
\end{hint}

\begin{solution}
Use the hint first. Then use your notes of the first lecture, as the steps are nearly the same.
As $M_{Y_j}(x) = 1-p_j + p_je^{s} = 1 + p_j(e^s-1)$,
\begin{align*}
  M_X(s) &= \prod_j M_{Y_j}(s) = \prod_{j} (1+p_j(e^s-1)).
\end{align*}

For b: use the hint of the book to see that $\prod_{j} (1+p_j(e^s-1)) \approx \prod_j \exp(p_j(e^s-1) = \exp\left(\sum_j p_j (e^s-1)\right)$.
Since $\lambda = \E X = \sum_jp_j$, this simplifies to $M_X(s) \approx \exp(\lambda(e^s-1))$.

\end{solution}


\end{exercise}
\end{document}

