\documentclass[study-guide-sol]{subfiles}
\externaldocument{study-guide}

\opt{solutionfiles,check}{
\Opensolutionfile{hint}
\Opensolutionfile{ans}
}

\begin{document}


\setcounter{theorem}{55}
\begin{exercise}[BH.5.56]
\begin{solution}
    \begin{enumerate}
	    \item $\int_{\mathbb{R}}z^2\Phi(z)\phi(z)dz $~\\~\\
    	Note here we need to calculate the expectation of one r.v. which is a transformation of r.v. $Z$ ($\Phi$ is just one function! Note that location transformation is a specific linear transformation of one r.v., you could consider the location-scale transformation as a transformation of a given r.v. $X$ with mean $\mu$ variance $\sigma^2$ using the function $f(a)=(a-\mu)/\sigma$).	
    	By LOTUS, we only need to know the distribution of Z to calculate the expectation (no need to know the distribution of the transformation $Z^2\Phi(Z)$):
    	\begin{align*}
    		\mathbb{E}Z^2\Phi(Z) = \int_{\mathbb{R}}z^2\Phi(z) \underbrace{\phi(z)}_{\text{The PDF of Z}}dz
    	\end{align*}
    	\item 2/3.\\~\\
    	By Universality of the Uniform, $\Phi(Z)\sim$Unif(0,1):
    	\begin{align*}
    		\mathbb{P}\left(\Phi(Z) < 2/3 \right)=2/3
    	\end{align*}
    	Alternatively, just calculate the probability directly:
    	\begin{align*}
    		\mathbb{P}\left(\Phi(Z) < 2/3 \right)=\mathbb{P}\left(Z < \Phi^{-1}(2/3) \right)=\Phi\left( \Phi^{-1}(2/3) \right) = 2/3
    	\end{align*}
    	which uses the steps in the proof of Universality of the Uniform. 
    	~\\~~\\
    	\textit{Remark: Universality of the Uniform simply tells us the CDFs of certain transformed r.v. For many transformations, we do not know the distributions of the transformations, but for some we know. For example, we know standardized transformation of one arbitrary normal distributed r.v. will have standard normal distribution, and  Universality of the Uniform provides two more cases.}
    	\item symmetry of continuous r.v.s. (Result: 1/3!).\\~\\
    	This is essentially proposition 5.7.1. 
    	Let's order three r.v.s, there would be 3! different orderings (e.g., one of them could be $Y<Z<X$), and by symmetry, all orderings of X, Y, Z are with equal probability. 
    	Further more we know the probability of any of these two r.v.s are equal would be zero, as they are all independent \textbf{continuous} r.v.s. 
    	
    	Therefore, we know all orderings of X, Y, Z are with equal probability and their sum is equal to one:   
    	\begin{align*}
    		\mathbb{P}(X<Y<Z) =1/3!
    	\end{align*} 
	\end{enumerate}
\end{solution}
\end{exercise}

\end{document}

