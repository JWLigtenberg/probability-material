\documentclass[study-guide-sol]{subfiles}
\externaldocument{study-guide}

\opt{solutionfiles,check}{
\Opensolutionfile{hint}
\Opensolutionfile{ans}
}

\begin{document}


\setcounter{theorem}{49}
\begin{exercise}[BH.5.50]
\begin{solution}
    If $X$ and $Y$ are i.i.d., then $P(X<Y)=P(Y<X)$ by symmetry since the problem of finding $P(X < Y )$ has exactly the same structure as that of finding $P(Y < X)$. E.g., two fair coins, which one lands head should be equally likely.

	If $X$ and $Y$ are dependent, then we also can't conclude that $P(X<Y)=P(Y<X)$, since then the structure of $P(X<Y)$ is different from the structure of $P(Y<X)$. For example, the dependence could be rigged so that $X$ is usually (or more likely) less than $Y$, even though they have the same distribution. See BH.42 in Chapter 3, $X,Y\sim$DUnif$\{1,2,3,4,5,6,7\}$, $Y=(X+1) \textit{mod} ~7$.  
	
	Furthermore,  $X$ and $Y$ are not interchangeable if they have different distributions, since then the symmetry is broken and it may be the case, for example, consider two unfair coins, coin $X$ always lands head and $Y$ tail.
\end{solution}
\end{exercise}

\end{document}

