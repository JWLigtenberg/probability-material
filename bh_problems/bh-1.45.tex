\documentclass[study-guide-sol]{subfiles}
\externaldocument{study-guide}

\opt{solutionfiles,check}{
\Opensolutionfile{hint}
\Opensolutionfile{ans}
}

\begin{document}

\setcounter{theorem}{22}
\begin{exercise}[BH.1.45]
	Let $A$ and $B$ be events. The symmetric difference $A \Delta B$ is defined to be the set of all elements that are in A or B but not both. In logic and engineering, this event is also called the XOR (exclusive or) of A and B. Show that $\P{A \Delta B} = \P{A} + \P{B} - 2 \P{A \cap B}$, directly using the axioms of probability.
\begin{solution}
	Use a Venn diagram. Note that 
	$$P(A \triangle B)=P\left(A \cap B^c\right)+P\left(A^c \cap B\right),$$
	since $A \triangle B$ is the union of the disjoint events $A \cap B^c$ and $A^c \cap B$. Similarly, we have
	$$
	\begin{aligned}
		&P(A)=P\left(A \cap B^c\right)+P(A \cap B) \\
		&P(B)=P\left(B \cap A^c\right)+P(B \cap A)
	\end{aligned}
	$$
	Hence,
	$$
	P(A)+P(B)=P\left(A \cap B^c\right)+P\left(A^c \cap B\right)+2 P(A \cap B) .
	$$
	Thus,
	$$
	P(A \triangle B)=P\left(A \cap B^c\right)+P\left(A^c \cap B\right)=P(A)+P(B)-2 P(A \cap B)
	$$
\end{solution}
\end{exercise}

\end{document}

