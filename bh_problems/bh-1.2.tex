\documentclass[study-guide-sol]{subfiles}
\externaldocument{study-guide}

\opt{solutionfiles,check}{
\Opensolutionfile{hint}
\Opensolutionfile{ans}
}

\begin{document}


\setcounter{theorem}{1}
\begin{exercise}[BH.1.2]

\begin{enumerate}
		\item[(a)]  How many 7-digit phone numbers are possible, assuming that the first digit can't
		be a 0 or a 1?
		\item[(b)] Re-solve (a), except now assume also that the phone number is not allowed to start
		with 911 (since this is reserved for emergency use, and it would not be desirable for the
		system to wait to see whether more digits were going to be dialed after someone has
		dialed 911).
	\end{enumerate} 
	\begin{solution}
	\begin{enumerate}
	\item[(a)]  For the first digit of the phone number we have 8 options. For the remaining $7-1$ digits we have 10 options. Hence in total there are $8\cdot 10^{(7-1)}$ phone numbers.  
	\item[(b)] This is the answer in a minus the phone numbers starting with 911. There are $10^4$ phone numbers starting with 911. Therefore the total number of phone numbers is $8\cdot 10^{(7-1)}-10^4$. 
\end{enumerate} 
	\end{solution}
\end{exercise}


\end{document}

