\documentclass[study-guide-sol]{subfiles}
\externaldocument{study-guide}

\opt{solutionfiles,check}{
\Opensolutionfile{hint}
\Opensolutionfile{ans}
}

\begin{document}

\setcounter{theorem}{4}
\begin{exercise}[BH.5]
	There are 20 people at a chess club on a certain day. They each find opponents and start playing. How many possibilities are there for how they are matched up, assuming that in each game it does matter who has the white pieces (in a chess game, one player has the white pieces and the other player has the black pieces)?
\begin{solution}
	Different solutions possible. The first player has 19 possible opponents and he can play white or black pieces, so $19\cdot 2$ options. The next player has 17 possible opponents and can again play with white or black pieces, so $17\cdot 2$ options. Continuing all the way down, we see that there are $N=(19\cdot 17\cdot 15\cdot\ldots\cdot 3\cdot 1)\cdot 2^{10}$ options.

	Alternatively, picture 10 boards with pieces already up. There are 20 seats for the players. There are therefore $20!$ arrangement possible. However, once seated, we can change the order of the boards + players. So we overcount by $10!$ and the total number of options is $N=\frac{20!}{10!}$.
\end{solution}
\end{exercise}

\end{document}

