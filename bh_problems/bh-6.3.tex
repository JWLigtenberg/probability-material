\documentclass[study-guide-sol]{subfiles}
\externaldocument{study-guide}

\opt{solutionfiles,check}{
\Opensolutionfile{hint}
\Opensolutionfile{ans}
}

\begin{document}

\setcounter{theorem}{2}
\begin{exercise} [BH.6.3]
\begin{solution}
    Median: $2^{1/\alpha}$ (CDF: $1-x^{-1\alpha}$). Mode: 1.\\~\\
	\\~\\
	Denote $F_X$ the CDF of continuous r.v. $X$, and if we check the definition of median, we know we need to find one value $x$ such that 
	\begin{align}
		\mathbb{P}(X\leq x)= F_X(x) \geq 1/2;~~ \mathbb{P}(X\geq x)=\mathbb{P}(X > x) =1-F_X(x) \geq 1/2
	\end{align}
	which implies that $F_X(x)=1/2$ \textit{(note this only holds for continuous r.v.s, as for discrete r.v.s $X'$, $\mathbb{P}(X'\geq x)=\mathbb{P}(X' > x)$ may not hold true if there is non-zero probability mass at $x$. See exercise 6.05 for example.)}. For one continuous r.v., to find its median, we only need to locate possible values of $x$ such that $$F_X(x)=1/2.$$  
	Now using PDF, we can derive $F_X$ by integration (for $x>1$):
	\begin{align*}
		F_X(x)=\mathbb{P}\left(X\leq x\right) = \int_{1}^{x} a/x^{a+1} dx =1-x^{-a}
	\end{align*}
	Solve $F_X(x)=1-x^{-a}=1/2$ we have a unique solution (\textit{in some cases, if CDF is flat at 1/2 over some intervals, solutions are not unique}) $2^{1/\alpha}$ which is the median. \\~\\
	The PDF given in the exercise is strictly decreasing over $[1,\infty)$, we know the mode is 1.
\end{solution}
\end{exercise}
\end{document}

