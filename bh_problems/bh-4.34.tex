\documentclass[study-guide-sol]{subfiles}
\externaldocument{study-guide}

\opt{solutionfiles,check}{
\Opensolutionfile{hint}
\Opensolutionfile{ans}
}

\begin{document}

\setcounter{theorem}{33}
\begin{exercise} [BH.4.34]
%\begin{hint}
%\end{hint}
\begin{solution}
    Let the indicator $I_i=1$ being the box $i$ is empty, then no balls can choose box $i$ (all balls need to choose the other $n-1$ boxes) and thus 
	\begin{align*}
		\mathbb{P}(I_i=1) = (1-1/n)^k
	\end{align*}
	Therefore, let N denote the total expected empty box number which would be equal to $\sum_{i=1}^n I_i$, and thus   
	\begin{align*}
		\mathbb{E}N=\mathbb{E} \sum_{i=1}^n I_i =\sum_{i=1}^n \mathbb{E} I_i=\sum_{i=1}^n   (1-1/n)^k
	\end{align*}
	Therefore, the final result is 
	$ n(1-1/n)^k$.
\end{solution}
\end{exercise}
\end{document}

