\documentclass[study-guide-sol]{subfiles}
\externaldocument{study-guide}

\opt{solutionfiles,check}{
\Opensolutionfile{hint}
\Opensolutionfile{ans}
}

\begin{document}

\setcounter{theorem}{9}
\begin{exercise}[BH.18]
	\begin{enumerate}
		\item In the World Series of baseball, two teams (call them $A$ and $B$) play a sequence of games against each other, and the first team to win four games wins the series. Let $p$ be the probability that A wins an individual game, and assume that the games are independent. What is the probability that team A wins the series?
		\item Give a clear intuitive explanation of whether the answer to (a) depends on whether the teams always play 7 games (and whoever wins the majority wins the series), or the teams stop playing more games as soon as one team has won 4 games (as is actually the case in practice: once the match is decided, the two teams do not keep playing more games).
	\end{enumerate}
\begin{solution}~
	\begin{enumerate}
	    \item Suppose that $n$ games are played with $n = 4,5,6,7$. If $A$ wins, it of course wins the final match and has won three of the preceding $n-1$ matches. Denote by $W_{i}$ the number of matches won out of $i$ matches. Since the outcomes of the matches are independent, the probability that $W_{n-1}=3$ is
        \begin{align*}
        	P(W_{n-1}=3) = {n-1 \choose 3}p^{3}(1-p)^{n-4}.
        \end{align*}
        The probability of winning is then
        \begin{align*}
        	P(A)& =\sum_{n=4}^{7}P(A|W_{n-1}=3)P(W_{n-1}=3)\\
        	&=p\sum_{n=4}^{7}P(W_{n-1}=3)\\
        	&=p^4\cdot \sum_{n=4}^{7}{n-1 \choose 3}(1-p)^{n-4}.
        \end{align*}
        \item If the teams keep playing, then you can rephrase the question to: what is the probability that team $A$ wins at least 4 out of seven matches? Notice that they can never first lose 4 (and therefore, lose the series) and then win 4, so we don't include scenarios in which team $A$ actually loses the series. Denote by $X$ the number of matches won by team $A$, then the probability of winning the series is,
        \begin{align*}
        	P(A) = \sum_{x=4}^{7}P(X=x) = \sum_{x=4}^{7}{7 \choose x}p^{x}(1-p)^{7-x}.
        \end{align*}
	\end{enumerate}
\end{solution}
\end{exercise}

\end{document}

