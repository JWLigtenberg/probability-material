\documentclass[lectures]{subfiles}
\begin{document}

\section{Lecture 11}
\label{sec:lecture-11}

% Main topics:
% \begin{enumerate}
% \item Conditional expectation
% \item Adam's and Eve's law.
% \end{enumerate}
% \clearpage


\begin{exercise}
As a continuation of BH.9.2.4, we now ask you to use Eve's law to compute $\V Y$, the variance of the second break point.
\begin{solution}
Recall that $\V X = 1/12$. More generally, if the stick has length $l$, then $\V X = l^{2}/12$. With this idea and using the solution of BH.9.2.4,
\begin{align}
\E{Y\given X=x} &= x/2,\\
\V{Y\given X=x}
&= \E{Y^{2}\given X=x} - (\E{Y\given X=x})^{2} \\
&= \frac{1}{x}\int_{0}^{x} y^{2} \d y - \left(\frac1x\int_{0}^{x} y \d y\right)^{2}  = x^{3}/3x - (x^{2}/2x)^{2} = x^{2}/12\\
\intertext{Replacing $x$ by $X$:}
\E{Y\given X} &= X/2,\\
\V{Y\given X} &= X^{2}/12, \\
\intertext{Adam's law:}
\E{Y\given X} &= X/2 \implies \E{Y} = \E{\E{Y\given X}} = \E{X/2} = 1/4.
\intertext{Eve's law:}
\V Y &= \E{\V{Y\given X}} + \V{\E{Y\given X}} \\
\V{Y\given X} &= X^{2}/12 \implies \E{\V{Y\given X}} = \E{X^2}/12 = \frac{1}{12} \int_0^1x^2 \d x = \frac{1}{12\cdot 3}\\
\E{Y\given X} &= X/2 \implies \V{\E{Y\given X}} = \V{X/2} = \frac{1}{4} \frac{1}{12}, \\
\V Y &= 1/12 (1/3 + 1/4) = 7/144.
\end{align}
\end{solution}
\end{exercise}


\begin{exercise}
A server (e.g., a machine, a mechanic, a doctor) spends a random amount $T$ on a job.
While the server works on the job, it sometimes gets interrupted to do other tasks $\{R_{i}\}$.
Such tasks can be failures of the machine, and they need to be repaired before the machine can continue working again.
(In the case of a mechanic, interruptions occur when the mechanic has to check some other machine or help other mechanics.)
It is clear that such interruptions can only occur when the server is working.
Assume that the interruptions arrive according to a Poisson process with rate $\lambda$, i.e., the number of failures $N| T \sim\Pois{\lambda T}$.
The interruptions $\{R_{i}\}$ are independent of $T$ and form an iid sequence with common mean $\E R$ and variance $\V R$.
(Typically, we get estimates for $\E T$ and $\V T$ from measurements.)

Use Adam and Eve's laws to express the expectation and variance of the total time $S$ to complete a job in terms of $\E T$ and $\V R$.
\begin{solution}

Below we need some properties of the Poisson distribution.
\begin{align}
\E{N(T)} &= \E{\E{N(T)\given T}}\\
\E{\E{N(T)\given T=t}}&= \E{N(t)} = \lambda t \\
\E{\E{N(T)\given T}}&= \lambda T \\
\E{N(T)} &= \E{\E{N(T)\given T}} = \E{\lambda T} = \lambda \E T\\
\V{N(t)} &= \lambda t
% \V N &= \E{\V{N\given T}} + \V{\E{N\given T}} \\
% \V{N\given T} &= \lambda T  \implies \E{\V{N\given T}} = \lambda \E{T} \\
% \E{N\given T} &= \lambda T \implies \V{\E{N\given T}} = \V{\lambda T} = \lambda^{2} \V T \\
% \V N &= \lambda \E T + \lambda^{2} \V T.
\end{align}

The  duration $S$ of a job  is its own service time $T$ plus all interruptions, hence $S = T+ \sum_{i=1}^{N(T)} R_{i}$, hence
\begin{align}
\E S &= \E{\E{S\given T}}   = \E{\E{T + \sum_{i=1}^{N(T)} R_{i}\given T}} \\
\E{S\given T = t} &= \E{T + \sum_{i=1}^{N(T)} R_{i}\given T=t} = \E{t + \sum_{i=1}^{N(t)} R_{i}} = t + \E{\sum_{i=1}^{N(t)} R_{i}} \\
\E{\sum_{i=1}^{N(t)} R_{i}}  &= \E{\E{\sum_{i=1}^{N(t)} R_{i}\given N(t)}}\\
\E{\sum_{i=1}^{N(t)} R_{i}\given N(t)= n} &= \E{\sum_{i=1}^{n} R_{i}} = n \E R  \implies \E{\sum_{i=1}^{N(t)} R_{i}\given N(t)} = N(t) \E R \\
\E{\sum_{i=1}^{N(t)} R_{i}} &= \E{N(t) \E R} = \E R \E{N(t)} = \lambda t \E R\\
\E{S \given T=t} &= t + \lambda \E R t = (1+\lambda \E R) t \\
\E{S \given T} &= (1+\lambda \E R) T \\
\E S  &= \E{(1 + \lambda \E R )T} = (1+\lambda \E R) \E T.
\end{align}
(When rereading this exercise, explain per step which assumption we use to get to the next step.)

For the variance,
\begin{align}
\V S &= \E{\V{S\given T}} + \V{\E{S\given T}}.
\end{align}
From the above we immediately have
\begin{align}
\V{\E{S\given T}} &= \V{(1+\lambda \E R) T} = (1+\lambda\E R)^{2} \V T.
\end{align}
For $\E{\V{S\given T}}$ we go step  by step.
\begin{align}
\V{S\given T=t} &= \V{T + \sum_{i=1}^{N(T)} R_{i}\given T=t} = \V{\sum_{i=1}^{N(t)} R_{i}},
\end{align}
as $\V t = 0$. But $N(t)$ is a Poisson rv. So we need to use Eve's law again, but now condition on $N(t)$.
\begin{align}
\V{\sum_{i=1}^{N(t)} R_{i}} &= \E{\V{\sum_{i=1}^{N(t)} R_{i}\given N(t)}} + \V{\E{\sum_{i=1}^{N(t)} R_{i}\given N(t)}}.
\end{align}
For the middle part:
\begin{align}
\V{\sum_{i=1}^{N(t)} R_{i}\given N(t)=n} &= \V{\sum_{i=1}^{n} R_{i}}  = n \V R \\
\V{\sum_{i=1}^{N(t)} R_{i}\given N(t)} &= N(t) \V R \\
\E{\V{\sum_{i=1}^{N(t)} R_{i}\given N(t)}} &= \E{N(t) \V R}=  \lambda t \V R.
\end{align}
For the right part:
\begin{align}
\E{\sum_{i=1}^{N(t)} R_{i}\given N(t)=n} &= n \E R\\
\E{\sum_{i=1}^{N(t)} R_{i}\given N(t)} &= N(t) \E R\\
\V{\E{\sum_{i=1}^{N(t)} R_{i}\given N(t)}} &= \V{N(t)\E R} = (\E R)^{2} \V{N(t)} = (\E R)^{2} \lambda t.
\end{align}
And so,
\begin{align}
\V{\sum_{i=1}^{N(t)} R_{i}} &= \lambda t \V R  + \lambda t (\E R)^{2}  = \lambda t \E{R^{2}}.
\end{align}
Putting the things together:
\begin{align}
\V{S\given T=t} &= \V{\sum_{i=1}^{N(t)} R_{i}} = \lambda t \E{R^{2}}\\
\V{S\given T} &= \lambda T \E{R^{2}}\\
\E{\V{S\given T}} &= \lambda \E{R^{2}} \E T.
\end{align}
So, finally,
\begin{align}
\V S &= \lambda \E{R^{2}} \E T + (1+\lambda\E R)^{2} \V T.
\end{align}

Special cases for you to sort out:
\begin{enumerate}
\item  $R$ and $T$ constant. Why is still $\V S > 0$?
\item Take $R$ and $T$ exponential; then the expressions for $\V T$ etc are still easy to manage.
\end{enumerate}
\end{solution}
\end{exercise}


This and the next exercise illustrate how to think about conditional expectation when the rv on which we condition is discrete.

\begin{exercise}
We have a standard fair die and we paint red the sides with $1, 2, 3$ and blue the other sides.
Let $X$ be the color after a throw and $Y$ the number.
Use the definitions of conditional expectation, conditional variance and Adam's and Eve's law to compute $\E X$ and $\V X$.
Compare that to the results you would obtain from using the standard methods; you should get the same result.

BTW, it's easy to make some variations on this exercise by painting other combinations of sides, e.g., only 1 and 2 are red.
\begin{solution}
The first step is easy:
\begin{align}
\E{Y\given X=r} &= \sum_{y} y \P{Y=y\given X=r} = \sum_{y} y \frac{\P{Y=y,  X=r}}{\P{X=r}} = \frac{(1+2+3)/6}{1/2}  = 2, \\
\E{Y\given X=b} &= \frac{(4 + 5+ 6)/6}{1/2}  = 5.
\end{align}

Recall that in the stick breaking exercise, we first computed $\E{Y\given X = x} = x/2$, and then we replaced $x$ by $X$ to get $\E{Y\given X} = X/2$. In the present case (with the die), where is the $x$ at the RHS? In other words, how to turn the above into a function $g(x)$ so that we can replace $x$ by $X$ to get a rv?


Here is the solution. Define $g$ as
\begin{equation}
  \label{eq:111}
  g(x) =
  \begin{cases}
    \E{Y\given X=r}, & \text{if } x = r \\
    \E{Y\given X=b}, & \text{if } x = b.
  \end{cases}
\end{equation}
But this is equal to
\begin{equation}
  \label{eq:112}
  g(x) = \E{Y\given X=r} \1{x=r} + \E{Y\given X=b} \1{x=b}.
\end{equation}

With this insight:
\begin{align*}
\E{Y\given x} &=  g(x) = \E{Y\given X=r} \1{x=r} + \E{Y\given X=b} \1{x=b}, \\
\E{Y\given X} &=  g(X) = \E{Y\given X=r} \1{X=r} + \E{Y\given X=b} \1{X=b}.
\end{align*}
Here, the indicators are the rvs!

And now,
\begin{align}
\E Y &= \E{\E{Y\given X}} = \E{g(X)} \\
&=\E{\E{Y\given X=r} \1{X=r} + \E{Y\given X=b} \1{X=b}}\\
&=\E{\E{Y\given X=r} \1{X=r}} + \E{\E{Y\given X=b} \1{X=b}}\\
&=\E{Y\given X=r} \E{\1{X=r}} + \E{Y\given X=b} \E{\1{X=b}}\\
&=\E{Y\given X=r} \P{X=r} + \E{Y\given X=b} \P{X=b}\\
&=2 \frac{1}{2}+ 5 \frac{1}{2}.
\end{align}

In view of this, let's consider LOTE again. Observing that
\begin{align}
A_{1} = \{s \in S : \textrm{paint of $s$ is red}\} = \{1, 2,3 \} \\
A_{2} = \{s \in S : \textrm{paint of $s$ is blue}\} = \{4, 5, 6\},
\end{align}
we have
\begin{align}
  \E{Y\given X=r} &= \E{Y\given A_{1}}, \\
  \E{Y\given X=b} &= \E{Y\given A_{2}} \\
  g(x) &= \E{Y\given X=r} \1{x=r} + \E{Y\given X=b} \1{x=b} \\
   &= \E{Y\given A_{1}} \1{x\in A_{1}} + \E{Y\given A_{2}} \1{x=A_{2}} \\
 &= \sum_{i=1,2}  \E{Y\given A_{i}} \1{x\in A_{i}}
\end{align}
The last is simple to generalize!
\begin{align}
  g(x)  &= \sum_{i}  \E{Y\given A_{i}} \1{x\in A_{i}}  \\
  g(X)  &= \sum_{i}  \E{Y\given A_{i}} \1{X\in A_{i}}.
\end{align}
And now we define $\E{Y\given X} = g(X)$. Remember these general equations!


Now observe that Adam's law and LOTE are the same for discrete rvs:
\begin{align*}
\E{Y}
&= \E{\E{Y\given X}} \\
&= \E{g(X)} \\
&=\E{\sum_{i} \E{Y\given A_{i}} \1{X\in A_{i}}} \\
&=\sum_{i}  \E{Y\given A_{i}} \E{\1{X\in A_{i}}} \\
&=\sum_{i}  \E{Y\given A_{i}} \P{A_{i}}.
\end{align*}
\end{solution}
\end{exercise}





\begin{exercise}
We have a continuous r.v. $X$ with PDF
\begin{align}
  \label{eq:1123}
  f(x) =
  \begin{cases}
    1/2, & 0\leq x < 1, \\
    1/4, & 1\leq x \leq 3,
  \end{cases}
\end{align}
and $0$ elsewhere. Compute $\E X$ and $\V X$.

Then define the r.v.
$Y$ such that $Y=\1{X< 1} + 2\1{X\geq 1}$.
Explain what information you obtain about $X$ when somebody tells that $Y=1$, or that $Y=2$.
Use conditioning on $Y$ and Adam's and Eve's laws to recompute $\E X$ and $\V X$.

Again, with this exercise we can check whether we are applying all concepts in the correct way.
\begin{solution}

Using the properties of the uniform distribution,
\begin{align}
\E X &= \frac{1}{2} \int_0^1 x \d x + \frac{1}{4} \int_1^3 x \d x = \frac{1}{4} + \frac{9-1}{8} = \frac{5}{4} \\
\E{X^2} &=  \frac{1}{2}\int_0^{1} x^{2}\d x + \frac{1}{4}\int_1^{3} x^{2}\d x = \frac{1}{6} + \frac{27-1}{12} = \frac{7}{3} \\
\V X &=  \frac{7}{3} - \frac{25}{16} = \frac{37}{48}.
\end{align}

$Y=1 \implies 0\leq X< 1$,  $Y=2 \implies 2\leq X \leq 4$.

\begin{align}
\E{X\given Y=1} &= 1/2, \\
\E{X\given Y=2} &= 2, \\
g(y) &= \E{X\given Y=1}\1{y=1} + \E{X\given Y=2}\1{y=2}\\
g(Y) &= \E{X \given Y} = \frac{1}{2} \1{Y=1} + 2 \1{Y=2},\\
\E{X} &= \E{\E{X\given Y}} = \E{g(Y)} \\
&=\E{\frac{1}{2} \1{Y=1} + 2 \1{Y=2}} \\
&= \frac{1}{2} \E{\1{Y=1}} + 2 \E{\1{Y=2}} \\
&= \frac{1}{2} \P{Y=1} + 2 \P{Y=2} \\
\P{Y=1} &= \E{\1{Y=1}} = \E{\1{X<1}} = \frac{1}{2} \\
\P{Y=2} &= \E{\1{Y=2}} = \E{\1{X\geq 1}} = \frac{1}{2} \\
\E X &= \frac{1}{2} \frac{1}{2} + 2 \frac{1}{2} = \frac{5}{4}.
\end{align}


\begin{align}
\V{X\given Y} &= \V{X\given Y=1}\1{Y=1} + \V{X\given Y=2}\1{Y=2} \\
&= \frac{1}{12} \1{Y=1} + \frac{4}{12} \1{Y=2}, \\
\E{\V{X\given Y}} &= \frac{1}{12} \frac{1}{2} + \frac{4}{12} \frac{1}{2} = \frac{5}{24}, \\
\V{\E{X \given Y}} &= \frac{1}{4}\V{\1{Y=1}} + 4\V{\1{Y=2}} + 2\cov{\frac{1}{2}\1{Y=1}, 2\1{Y=2}} \\
\V{\1{Y=1}} &= \V{\1{Y=2}}  = \frac{1}{4}, \\
2\cov{\frac{1}{2}\1{Y=1}, 2\1{Y=2}} &= 2\cov{\1{Y=1}, \1{Y=2}}=-2\E{\1{Y=1}}\E{\1{Y=2}} = -\frac{1}{2}\\
\V{\E{X \given Y}} &= \frac{1}{16} + 1 - \frac{1}{2} = \frac{9}{16}\\
\V X &= \E{\V{X\given Y}} + \V{\E{X\given Y}} = \frac{5}{24} + \frac{9}{16}.
\end{align}
I forgot the covariance at first :-(
\end{solution}
\end{exercise}
\end{document}
