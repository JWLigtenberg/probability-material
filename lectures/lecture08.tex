\documentclass[lectures-questions]{subfiles}
\begin{document}
\section{Lecture 8}

\begin{exercise}
Let $X$ be a continuous random variable with a pdf
\begin{align}
    f_X(x) = \begin{cases}
    c, &\text{if } 0 \leq x \leq 4, \\
    0, &\text{otherwise}.
    \end{cases}
\end{align}
\begin{enumerate}
    \item What is the value of $c$?
    \item What is the distribution of $X$?
    \item Do we need to know the value of $c$ to determine the distribution of $X$?
\end{enumerate}

\begin{solution}
We need that the pdf $f_X$ integrates to one. Hence, we need
\begin{align}
    \int_{-\infty}^\infty f_X(x) dx &= 1 \qquad \iff \\
    \int_{0}^4 c dx &= 1 \qquad \iff \\
    4c &= 1 \qquad \iff \\
    c &= 1/4.
\end{align}
Clearly, $X$ is uniformly distributed on $[0,4]$. In fact, we do not need to know the value of $c$ to determine this. It is sufficient to know that the pdf of $X$ is constant on the interval $[0,4]$.
\end{solution}
\end{exercise}

\begin{exercise}
Let $X$ be a continuous random variable with a pdf
\begin{align}
    f_X(x) = c \cdot e^{-\frac{(x - 4)^2}{8}}, \quad x \in \R.
\end{align}
\begin{enumerate}
    \item What is the value of $c$?
    \item What is the distribution of $X$?
    \item Do we need to know the value of $c$ to determine the distribution of $X$?
\end{enumerate}
\begin{solution}
We need that the pdf $f_X$ integrates to one. Hence, we need
\begin{align}
    \int_{-\infty}^\infty f_X(x) dx &= 1 \qquad \iff \\
    \int_{-\infty}^\infty c \cdot e^{-\frac{(x - 4)^2}{8}} dx &= 1 \qquad \iff \\
    c \cdot \sqrt{2\pi}\cdot 2 \int_{-\infty}^\infty  \frac{1}{\sqrt{2\pi}\cdot 2} e^{-\frac{1}{2}\frac{(x - 4)^2}{2^2}} dx &= 1 \qquad \iff \\
    c \cdot \sqrt{2\pi}\cdot 2 \cdot 1 &= 1 \qquad \iff \\
    c &= \frac{1}{\sqrt{2\pi}\cdot 2}.
\end{align}
Clearly, $X$ is $N(4,2)$ distributed. In fact, we do not need to know the value of $c$ to determine this. It is sufficient to observe the structure of the pdf as a function of $x$.
\end{solution}
\end{exercise}


(BH.8.4.5) Fred is waiting at a bus stop. He knows that buses arrive according to a Poisson process with rate $\lambda$ buses per hour. Fred does not know the value of $\lambda$, though. Fred quantifies his uncertainty of $\lambda$ by the \textit{prior distribution} $\lambda \sim \text{Gamma}(r_0, b_0)$, where $r_0$ and $b_0$ are known, positive constants with $r_0$ an integer.

Fred has waited for $t$ hours at the bus stop. Let $Y$ denote the number of buses that arrive during this time interval. Suppose that Fred has observed that $Y=y$.

\begin{exercise}
Find Fred's (hybrid) joint distribution for $Y$ and $\lambda$.
\begin{solution}
See the book.
\end{solution}
\end{exercise}

\begin{exercise}
Find Fred's marginal distribution for $Y$. Use this to compute $\E{Y}$. Interpret the result.
\begin{solution}
See the book for the solution to the first part. From the result, it follows that $\E{Y} = \frac{r_0}{b_0} t$ (just plug in the parameters of the Negative Binomial distribution into the expression for the expected value on page 161).\\
Note that the expected number of buses is linear in time $t$ and has a rate of $r_0/b_0$ per hour, which is the expected value of the rate $\lambda$ of our Poisson process.
\end{solution}
\end{exercise}

\begin{exercise}
Find Fred's posterior distribution for $\lambda$, i.e., his conditional distribution of $\lambda$ given the data $y$.
\begin{solution}
See the book.
\end{solution}
\end{exercise}

\begin{exercise}
Find Fred's posterior mean $\E{\lambda | Y = y}$ and variance $\V{\lambda|Y=y}$.
\begin{solution}
See the book.
\end{solution}
\end{exercise}


%\end{comment}
\end{document}
