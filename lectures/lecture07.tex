\documentclass[lectures-questions]{subfiles}
\begin{document}

\section{Lecture 7}
\label{sec:lecture-7}

\begin{exercise}
At the end of Story 2 of Bayes' billiards (BH.8.3.2) there is the expression
\begin{equation}
\label{eq:72}
\beta(a,b) = \frac{(a-1)!(b-1)!}{(a+b-1)!}.
\end{equation}
Derive this equation.
\begin{solution}
We have by the Bayes' billiard stories that
\begin{align}
{n \choose k} \int_0^1 p^k(1-p)^{n-k}\d p = \frac{1}{n+1},
\end{align}
but the integral is equal to $\beta(k+1, n-k+1)$ by the definition of the $\beta$ function. Hence,
\begin{align*}
\beta(k+1, n-k+1)
&= \frac{1}{(n+1) {n \choose k}} = \frac{(n-k)!k!}{(n+1)!}.
\end{align*}
Taking $a-1 = k$, $b-1 =n -k$, then $n=b-1+k = a+b-2$, so that
\begin{align*}
\beta(a, b) = \frac{(b-1)!(a-1)!}{(a+b-1)!}.
\end{align*}
\end{solution}
\end{exercise}

\begin{exercise}
In the Beta-Binomial conjugacy story, BH take as prior $f(p) = \1{p\in [0,1]}$, and then they remark that when $f(p) \sim \beta(a,b)$ for general $a, b \in \N$, we must obtain the negative hypergeometric distribution.
I found this pretty intriguing, so my question is: Relate the Bayes' billiards story to the story of the Negative Hypergeometric distribution, and, in passing, provide an interpretation of $a$ and $b$ in terms of white and black balls.
Before trying to answer this question, look up the details of the negative hypergeometric distribution.
(In other words, this exercise is meant to help you sort out the details of the remark of BH about the negative hypergeometric distribution.)
\begin{solution}
  Let's follow the Bayes' billiards story no 1, but in the context of the negative hypergeometric distribution.
  We have $w$ white balls, and $b$ black balls.
  We place, arbitrarily all balls on a line, and mark the $r$th, $r\leq w$, white ball.
  Call this ball the pivot.
  Then we ask:  what is the probability that the number of black balls $X$ lying at the left of the pivot is equal to $k$, i.e, $\P{X=k}$?

Conditional on the position of the pivot being $p$, we must have that
\begin{equation}
\label{eq:73}
\P{X=k} = w \int_0^1{b \choose k} p^k(1-p)^{b-k} \times {w -1 \choose r-1}p^{r-1} (1-p)^{w-r} \d p.
\end{equation}
To see this, reason as follows.
There are $w$ ways to choose one of the white balls as the pivot.
(I initially forgot this factor.)
Then, given $p$, $k$ black balls lie to the left of the pivot; the rest lies to the right.
Then, of the $w-1$ remaining white balls, $r-1$ also have to lie the left of the pivot, and the others to the right (observe that $w-1 - (r-1) = w -r$, hence $(1-p)^{w-r}$.)

With the definition of the  $\beta$ function, we can simplify to this:
\begin{align}
\P{X=k}
&= w{b \choose k} {w -1 \choose r-1}  \int_0^1 p^{k+r-1}(1-p)^{b-k+w-r} \d p \\
&= w{b \choose k} {w -1 \choose r-1}  \beta(k+r, b-k+w-r + 1)\\
&= w{b \choose k} {w -1 \choose r-1} \frac{(r+k-1)! (b-k+w-r)!}{(w+b)!}.
\end{align}

Now, from the story of the negative hypergeometric distribution, we have that
\begin{align}
\P{X=k}
&= \frac{{w \choose r-1}{b \choose k}}{{w+b \choose r+k -1}}\frac{w-(r-1)}{w -(r-1) +b-k} \\
\end{align}
To see this, note that when $X=k$, the $r+k$th ball is white (hence we stop).
The probability to select a white ball then $w-(r-1)$ out of the remaining $w-(r-1) + b-k$.
(For $X$ to be equal to $k$ we have removed $k$ black balls and $r-1$ white balls).
And we have to multiply by the number of combinations to select $r-1$ of the $w$ balls and $k$ of the black balls.


Next, with a some algebra---here we do the work, but the books just say `after a bit of algebra we find', and then it is left to you to do the algebra---,
\begin{align}
\label{eq:717}
\P{X=k}
&= \frac{{w \choose r-1}{b \choose k}}{{w+b \choose r+k -1}}\frac{w-r+1}{w+b-r-k+1} \\
&= {w \choose r-1}{b \choose k} \frac{(r+k-1)!(w+b-r-k+1)!}{(w+b)!}\frac{w-r+1}{w+b-r-k+1} \\
&= {w \choose r-1}{b \choose k} \frac{(r+k-1)!(w+b-r-k)!}{(w+b)!}(w-r+1) \\
&= w {w -1 \choose r-1}{b \choose k} \frac{(r+k-1)!(w+b-r-k)!}{(w+b)!},
\end{align}
where we use that
\begin{align}
  w{w-1 \choose r-1} &= w\frac{ (w-1)!}{(w-r)! (r-1)!} = \frac{ w!}{(w-r)! (r-1)!} \\
&=\frac{ w!}{(w-r+1)! (r-1)!} (w-r+1)= {w\choose r-1} (w-r+1).
\end{align}

The results for the Bayes' billiards match with those of the negative hypergeometric distribution!

So, inspecting~\cref{eq:73} we see that the $a$ and $b$ parameters in the $\beta$ distribution for the prior $f(p)$ are such that $a=r$ and $b=w-r+1$.
(Don't confuse the $b$ here with the number of black balls.
I took the $b$ to stick to the notation of the book.)
In other words, the $a$ corresponds to the white ball we mark as the pivot, and $b$ to the remaining number of white balls (plus 1).
\end{solution}
\end{exercise}

The next real exercise is about recursion applied to the negative hypergeometric distribution. But to get in the mood, here is short fun question on how to use recursion.

\begin{exercise}
We have a chocolate bar consisting of $n$ small squares.
The bar is in any shape you like, square, rectangular, whatever.
What is the number of times you have to break the bar such that you end up with the $n$ single pieces?
%Recursive arguments are very powerful, and I often find them very insightful (and I also like this type of reasoning).
\begin{solution}
  Write $T(n)$ for the number of times you have to break the bar when there are $n$ squares.
  Clearly, $T(1) =0$.
  In general, suppose $n=m+k$, $1 \leq k < n$, then $T(n) = T(m) + T(k) + 1$, because we need $T(k)$ to break the part with $k$ pieces, and $T(m)$ for the part with $m$ pieces, and we need 1 to break the big bar into the two parts. But then, with the \emph{boundary condition} $T(1)=0$,
  \begin{align*}
    T(2) &= T(1) + T(1) + 1 = 1, \\
T(3) &= T(1) + T(2) + 1 = 1 + 1 = 2,
  \end{align*}
and so on. So we guess that $T(n)=n-1$ for any $n$. Let's check. It certainly holds for $n=1$.  Generally,
\begin{align*}
n-1 = T(n) = T(k) + T(m) + 1 = k-1 + m-1 + 1 = k+m-1 = n-1.
\end{align*}
\end{solution}
\end{exercise}

\begin{exercise}
Use recursion to find the expected number $X$ of black balls drawn without replacement at random from an urn containing $w\geq 1$ white balls and $b$ black balls before we draw 1 white ball.
In other words, I ask to use recursion to compute $\E X$ for $X$ a negative hypergeometric distribution with parameters $w,b, r=1$ and show that
\begin{align}
  \label{eq:727}
\E X = \frac{b}{w+1}
\end{align}
\begin{hint}
For ease, write $N(w,b)= \E X$ for an  urn with $w\geq r = 1$ white balls and $b$ black balls. Then explain that
\begin{align}
N(w, 0) &= 0\quad \text{ for all $w$},\\
  N(w,b) &= \frac{b}{w+b} (1+N(w, b-1)).
\end{align}
Then show that this implies that $N(w,b) = b/ (w+1)$.
\end{hint}
\begin{solution}
If there are no black balls left, then we cannot pick a black ball which implies that $N(w, 0) = 0$.
As soon as we pick a white ball, we have to stop.

When $w, b \geq 1$ two things can happen.
Suppose we pick a white ball, then we stop right away and we cannot increase the number of black balls picked.
Suppose we pick a black ball, which happens with probability $b/(b+w)$, we obtain one black ball, and we continue with one black ball less.
Hence, with LOTE,
\begin{align}
  N(w,b) = \frac{b}{w+b} (1+N(w, b-1)).
\end{align}
You should realize that here we use conditioning; we condition on the color of the ball picked.

Now we need a real tiny bit of inspiration.
Let's \emph{guess} that $N(w,b)=\alpha b$ for some $\alpha>0$.
Guessing is always allowed; if it works, we are done (since the recursion, together with the boundary condition, has a unique solution: just repeatedly applying it allows us to calculate all values).
In general, you can make any guess whatsoever, but mind that only the good guesses work. So, trying the form $N(w,b) = \alpha b$, we find that
\begin{equation*}
\alpha b = \frac{b}{w+b} (1+ \alpha (b-1)) \stackrel{\textrm{\tiny algebra}}{\implies} \alpha = \frac{1}{w+1} \implies  N(w, b) = \frac{b }{w+1}.
\end{equation*}
This is consistent with the boundary condition $N(w,0)=0$. Hence, we can move the case $b=0$ to the case with $b=1$, and so on, thereby proving the validity of the formula in general.
\end{solution}
\end{exercise}

\begin{exercise}
Extend the previous exercise to cope with the case $r\geq 2$.
For this, write $N_{r}(w,b)$ for an urn with $w$ white balls and $b$ black balls, and $r$ white balls to go.
\begin{hint}
Explain that $N_{0}(w,b) = 0$ and
\begin{align}
  N_r(w,b) &= \frac{w}{w+b} N_{r-1}(w-1, b) +  \frac{b}{w+b} (1+N_{r}(w, b-1)).
\end{align}
Then show that this implies that $N_{r}(w,b) = r b/ (w+1)$.
\end{hint}
\begin{solution}
The recursion follows right away by noticing that we pick a white ball with probability $w/(w+b)$, but then we remove one white ball \emph{and} we have one white ball less to go.
With probability $b/(w+b)$ we pick a black ball, in which case the number of black balls drawn increases by one, but there is one black ball less in the urn.

The boundary condition is also clear: $N_{0}(w,b)=0$ because we stop.
Note that this is consistent with the previous exercise.

If $r=2$, we use from the previous exercise that
\begin{equation*}
N_{r-1}(w-1, b) = N_{1}(w-1, b) = \frac{b }{w-1 + 1} = \frac{b }{w}.
\end{equation*}
Using this in the recursion,
\begin{align*}
  N_2(w,b)
&= \frac{w}{w+b} N_{1}(w-1, b) +  \frac{b}{w+b} (1+N_{2}(w, b-1)) \\
&= \frac{w}{w+b} \frac{b }{w} +  \frac{b}{w+b} (1+N_{2}(w, b-1)) \\
&=  \frac{b}{w+b} (2+N_{2}(w, b-1)).
\end{align*}
But this has the same form as~\cref{eq:727}, except that the 1 has been replaced by a 2. But, now we are dealing with the case $r=2$ instead of $r=1$. Hence, by the same token,
\begin{equation*}
  N_2(w,b) = 2 \frac{b}{w+1}
\end{equation*}
Knowing the result for $r=2$, we fill this for $r=3$, and so on, to get the general result.

What an elegant procedure;  we pull ourselves out of the swamp, just like Baron Munchhausen.
\end{solution}
\end{exercise}
\end{document}
