\documentclass[study-guide-sol]{subfiles}
\externaldocument{study-guide}

\opt{solutionfiles,check}{
\Opensolutionfile{hint}
\Opensolutionfile{ans}
}

\begin{document}


\section{Our exercises on Section BH.7.1.2}


\begin{exercise}
Let $X, Y$ be two discrete rvs with CDF $F_{X,Y}$.  Can we compute the PDF as $\partial_{x}\partial_{y} F_{X,Y}(x,y)$?
\begin{solution}
This claim is incorrect, because $X, Y$ are discrete, hence they have a PMF, not a PDF.

Mistake: Someone said that $\partial_{x}\partial_{y}$ is not correct notation; however, it is correct! It's a (much used) abbreviation of the much heaver $\partial^{2}/\partial x \partial y$. Next, the derivative of the PMF is not well-defined (at least, not within this course. If you object, ok, but then show that you passed a decent course on measure theory.)
\end{solution}
\end{exercise}



\begin{exercise}
We have the random vector $(X, Y) \in [0,1]^{2}$ (here $[0,1]^{2} = [0,1]\times [0,1]$) consisting of the rvs X and Y with the joint PDF $f_{X,Y}(x,y) = 2 \1{x\leq y}$.
\begin{enumerate}
\item Are $X$ and $Y$ independent?
\item Compute $F_{X,Y}(x,y)$.
\end{enumerate}
\begin{solution}
\begin{align}
f_{X}(x) &= \int_{0}^{1} f_{X,Y}(x,y) \d y = 2\int_{0}^{1} \1{x\leq y} \d y = 2\int_{x}^{1} \d y = 2(1-x) \\
f_{Y}(y) &= \int_{0}^{1} f_{X,Y}(x,y) \d x = 2\int_{0}^{1} \1{x\leq y} \d x = 2\int_{0}^{y} \d y = 2 y.
\end{align}
But $f_{X,Y}(x,y) \neq f_{X}(x)f_{Y}(y)$, hence $X,Y$ are dependent.

\begin{align}
F_{X,Y}(x,y)
&= \int_{0}^{x}\int_{0}^{y} f_{X,Y}(u,v) \d v \d u \\
&= 2\int_{0}^{x}\int_{0}^{y} \1{u\leq v} \d v \d u \\
&= 2\int_{0}^{x}\int \1{u\leq v} \1{0\leq v \leq y}\d v \d u \\
&= 2\int_{0}^{x}\int  \1{u\leq v \leq y}\d v \d u \\
&= 2\int_{0}^{x} [y-u]^{+} \d u,
\end{align}
because $u > y \implies \1{u\leq v \leq y} = 0$. Now, if $y>x$,
\begin{align}
  2\int_{0}^{x} [y-u]^{+} \d u &=
  2\int_{0}^{x} (y-u) \d u = 2 y x - x^{2},
\end{align}
while if $y\leq x$,
\begin{align}
  2\int_{0}^{x} [y-u]^{+} \d u &=
  2\int_{0}^{y} (y-u) \d u = 2 y^{2} - y^{2} = y^{2}
\end{align}

Make a drawing of the support of $f_{X,Y}$ to help to understand this better.

\end{solution}
\end{exercise}

\begin{exercise}
We have two continuous rvs $X, Y$.
Suppose the joint CDF factors into the product of the marginals, i.e., $F_{X,Y}(x,y) = F_X(x)F_Y(y)$. Can it still be possible in general that the joint PDF does not factor into a product of marginals PDFs of $X$ and $Y$, i.e., $f_{X,Y}(x,y) \neq f_X(x) f_Y(y)$?
\begin{solution}
\begin{align*}
\partial_{x}\partial_{y}F_{X,Y}(x,y)
=\partial_{x}\partial_{y}F_{X}(x) F_{Y}(y)
=\partial_{x}F_{X}(x) \partial_{y} F_{Y}(y) = f_{X}(x) f_{Y}(y).
\end{align*}
\end{solution}
\end{exercise}

\begin{exercise}
BH define the conditional CDF given an event $A$ on page 416 as $F(y|A)$.
Use this definition to write $F_{X,Y}(x,y)/F_{X}(x)$ as a conditional CDF.
Is this equal to the conditional CDF of $X$ and $Y$?
\begin{solution}
\begin{align}
\frac{F_{X,Y}(x,y)}{F_{X}(x)} = \frac{\P{X\leq x, Y\leq y}}{\P{X\leq x}}
  = \P{Y\leq y, X\leq x|X\leq x} = \P{Y\leq y|X\leq x}.
\end{align}
It is a big mistake to write $F_{X,Y}(x,y) = \P{X=x, Y=y}$. If you wrote this, recheck the definitions of BH.
\end{solution}
\end{exercise}


\begin{exercise}
We have two rvs $X$ and $Y$ on $\R^{+}$. It is given that $F_{X,Y}(x,y) = F_X(x)F_Y(y)$ for $x,y \leq 1/3$. It is true that then  $X$ and $Y$ are necessarily independent.
\begin{solution}
For $X, Y$ to be independent, it is necessary that  $F_{X,Y}(x,y) = F_X(x)F_Y(y)$ for all $x,y$, not just one particular choice. (This is an example that satisfying a necessary condition is not necessarily sufficient.)
\end{solution}
\end{exercise}

\begin{exercise}
I select a random guy from the street, his height $X\sim\Norm{1.8, 0.1}$, and I select a random woman from the street, her height is $Y\sim\Norm{1.7, 0.08}$.
I claim that since I selected the man and the woman independently, their heights are independent.
Briefly comment on this claim.


\begin{solution}
  Many answers are possible here, depending on extra assumptions you make.
  Here is one.
  Suppose, just by change, the fraction of taller guys in the street is a bit higher than the population fraction.
  Assuming that taller (shorter) people prefer taller (shorter) spouses, there must be a dependence between the height of the men and the women. This is because when selecting a man, I can also select his wife.

From this exercise you should memorize that \emph{independence is a property of the joint CDF, not of the rvs}.

Mistake:   $\P{Y}$ is wrong notation wrong because we can only compute the probability of an event, such as $\{Y\leq y\}$. But $Y$ itself is not an event. \end{solution}
\end{exercise}


\begin{exercise}
For any two rvs $X$ and $Y$ on $\R^{+}$ with marginals $F_{X}$ and $F_{Y}$, can  it hold that $\P{X\leq x, Y\leq y} = F_{X}(x) F_{Y}(y)$?
\begin{solution}
Only when $X, Y$ are independent.

Mistake:  independence of $X$ and $Y$ is not the same as the linear independence. Don't confuse these two types of dependene.
\end{solution}

\end{exercise}

\begin{exercise} Redo BH.7.1.24 with indicator functions and the  fundamental bridge (recall, $\P{A} = \E{\1{A}}$ for an event $A$).
(Indicators are often  easy to use, and prevent many mistakes, as is demonstrated with this example.)
\begin{solution}
\begin{align*}
\P{T_1 < T_2 } = \E{\1{T_1<T_2}} =
&= \int_0^\infty \int_0^\infty \1{t_1<t_2} f_{T_1, T_2}(t_1, t_2)\d t_{1} \d{t_2} \\
&= \int_0^\infty \int_{t_1}^{\infty}  \lambda_1 e^{-\lambda_1 t_1} \lambda_2 e^{-\lambda_2 t_2}  \d{t_2} \d{t_1}\\
&= \int_0^\infty   \lambda_1 e^{-\lambda_1 t_1} \lambda_2 \int_{t_1}^{\infty} e^{-\lambda_2 t_2}  \d{t_2} \d{t_1}\\
&= \int_0^\infty   \lambda_1 e^{-\lambda_1 t_1} e^{-\lambda_2 t_1}  \d{t_1}\\
&= \int_0^\infty   \lambda_1 e^{-\lambda_1 t_1 - \lambda_2t_{1}} \d{t_1}\\
&= \frac{\lambda_{1}}{\lambda_{1}+\lambda_2}.
\end{align*}
\end{solution}
\end{exercise}


\begin{exercise}
We have a continuous r.v. $X\geq 0$ with finite expectation. Use 2D integration and indicators to prove that
\begin{align}
\E X = \int_{0}^{\infty} x f(x) \d x = \int_{0} G(x) \d x,
\end{align}
where $G(x)$ is the survival function.
\begin{hint}
  Check the proof of BH.4.4.8
\end{hint}
\begin{solution}
The trick is to realize that $x = \int_0^{\infty} \1{y\leq x} \d y$. Using this,
\begin{align}
\E X
&= \int_{0}^{\infty} x f(x) \d x \\
&= \int_{0}^{\infty} \int_{0}^{\infty} \1{y \leq x} f(x) \d y \d x \\
&= \int_{0}^{\infty} \int_{0}^{\infty} \1{y \leq x} f(x) \d x \d y \\
&= \int_{0}^{\infty} \int_{0}^{\infty} \1{x \geq y} f(x) \d x \d y \\
&= \int_{0}^{\infty} \int_{y}^{\infty} f(x) \d x \d y \\
&= \int_{0}^{\infty} G(y) \d y.
\end{align}
\end{solution}
\end{exercise}


\begin{exercise}
A variation on Exercise BH.7.1. Alice is prepared to wait 20 minutes for Bob, while Bob doesn't want to wait longer than 10 minutes. What is the probability that they meet?

Use the fundamental bridge and indicator functions to write this probability as a 2D integral. Then use repeated integration to solve the 2D integral.
\begin{solution}
Let $A$, $B$ be the arrival times of Alice and Bob. They meet if $\1{A<B+1/3}\1{B<A+1/6}$ is true, i.e., is equal to 1. Therefore, by letting $M$ be the event that they meet:

\begin{equation*}
\P{M} = \E{\1{A<B+1/3}\1{B<A+1/6}} = \int_0^1\int_0^1 \1{x<y+1/3}\1{y<x+1/6} \d y \d x.
\end{equation*}
We can solve this integral by first integrating along $y$, and then along $x$. Let's focus on the integral over $y$ first.
\begin{align*}
\int_0^1 \1{x<y+1/3}\1{y<x+1/6} \d y
&=\int_0^1 \1{x-1/3<y<x+1/6} \d y \\
&=\int_0^1 \1{\max\{0, x-1/3\} <y \min\{1,x+1/6\} } \d y \\
&= \min\{1,x+1/6\} - \max\{0, x-1/3\}
\end{align*}
Now the integral over $x$:
\begin{align*}
  \int_0^1 (\min\{1,x+1/6\} - \max\{0, x-1/3\}) \d x
&=  \int_0^1 \min\{1,x+1/6\}\d x - \int_0^{1}\max\{0, x-1/3\} \d x  \\
&=  \int_0^{5/6}(x+1/6)\d x + \int_{5/6}^{1}1 \d x
 - \int_{1/3}^{1}(x-1/3) \d x \\
  &=0.5 x^2\Big|_{0}^{5/6} + 1/6\cdot 5/6 - 0.5x^2\Big|_{1/3}^1 + 1/3\cdot 2/3
\end{align*}

Of course, we can find the probability with some simple geometric arguments (compute the area of two triangles). However, this does not work any longer if the density is not uniform.  Then we have to do the integration, and that is the reason why I show above how to handle the general case.
\end{solution}
\end{exercise}


\end{document}

