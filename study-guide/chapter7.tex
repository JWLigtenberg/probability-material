% arara: pdflatex: { shell: yes }
% arara: pythontex: {verbose: yes, rerun: modified }
% arara: pdflatex: { shell: yes }
% arara: move: {  files: ['chapter7.pdf'], target: ['..'] }


\documentclass[study-guide-sol]{subfiles}
\externaldocument{study-guide}

\opt{solutionfiles,check}{
\Opensolutionfile{hint}
\Opensolutionfile{ans}
}

\begin{document}



\chapter{Chapter 7: Exercises and remarks}
\label{cha:questions-chapter-7}



\section{7.1}


\begin{exercise}
In your own words, explain what is
\begin{enumerate}
\item a joint PMF, PDF, CDF;
\item a conditional PMF, PDF, CDF;
\item a marginal PMF, PDF, CDF.
\end{enumerate}
\begin{solution}
Check the definitions of the book.

Mistake: To say that $\P{X=x}$ is the PMF for a continuous random variable is wrong, because $\P{X=x}=0$ when $X$ is continuous.

Why is $\P{1 < x \leq 4}$ wrong notation?
hint: $X$ should be a capital.
What is the difference between $X$ and $x$?

\end{solution}
\end{exercise}

\begin{exercise}
Suppose the probability of obtaining a head twice out of two coin flips is $\P{X_1=H, X_2=H}$.
What has this to do with joint PMFs? Can you generalize this idea to other examples?
\begin{solution}
This example shows why joint distributions are important!
In any experiment that involves a sequence of measurements, such as multiple throws of a coin, or the weighing of a bunch of chimpanzees, we have to deal with joint CDFs and PMFs.
\end{solution}
\end{exercise}


\begin{exercise}
In the previous exercise, suppose the outcome of the second throw is always equal to that of the first. Specify the joint PMF.
\begin{solution}
Here, we deal with two rvs, and we have to specify how they depend. In the present case $\P{X_1= H, X_2=H} = \P{X_1=H}$ and $\P{X_1= T, X_2=T} = \P{X_1=T}$, $\P{X_1= H, X_2=T} = \P{X_1= T, X_2=H} = 0$. Note that with this, we specified the joint PMF on all possible outcomes.
\end{solution}
\end{exercise}


\begin{exercise}
We have the random vector $(X, Y) \in [0,1]^{2}$ (here $[0,1]^{2} = [0,1]\times [0,1]$) consisting of the rvs X and Y with the joint PDF $f_{X,Y}(x,y) = 2 \1{x\leq y}$.
\begin{enumerate}
\item Are $X$ and $Y$ independent?
\item Compute $F_{X,Y}(x,y)$.
\end{enumerate}
\begin{solution}
\begin{align}
f_{X}(x) &= \int_{0}^{1} f_{X,Y}(x,y) \d y = 2\int_{0}^{1} \1{x\leq y} \d y = 2\int_{x}^{1} \d y = 2(1-x) \\
f_{Y}(y) &= \int_{0}^{1} f_{X,Y}(x,y) \d x = 2\int_{0}^{1} \1{x\leq y} \d x = 2\int_{0}^{y} \d y = 2 y.
\end{align}
But $f_{X,Y}(x,y) \neq f_{X}(x)f_{Y}(y)$, hence $X,Y$ are dependent.

\begin{align}
F_{X,Y}(x,y)
&= \int_{0}^{x}\int_{0}^{y} f_{X,Y}(u,v) \d v \d u \\
&= 2\int_{0}^{x}\int_{0}^{y} \1{u\leq v} \d v \d u \\
&= 2\int_{0}^{x}\int \1{u\leq v} \1{0\leq v \leq y}\d v \d u \\
&= 2\int_{0}^{x}\int  \1{u\leq v \leq y}\d v \d u \\
&= 2\int_{0}^{x} [y-u]^{+} \d u,
\end{align}
because $u > y \implies \1{u\leq v \leq y} = 0$. Now, if $y>x$,
\begin{align}
  2\int_{0}^{x} [y-u]^{+} \d u &=
  2\int_{0}^{x} (y-u) \d u = 2 y x - x^{2},
\end{align}
while if $y\leq x$,
\begin{align}
  2\int_{0}^{x} [y-u]^{+} \d u &=
  2\int_{0}^{y} (y-u) \d u = 2 y^{2} - y^{2} = y^{2}
\end{align}

Make a drawing of the support of $f_{X,Y}$ to help to understand this better.

\end{solution}
\end{exercise}

\begin{exercise}
We have two continuous rvs $X, Y$.
Suppose the joint CDF factors into the product of the marginals, i.e., $F_{X,Y}(x,y) = F_X(x)F_Y(y)$. Can it still be possible in general that the joint PDF does not factor into a product of marginals PDFs of $X$ and $Y$, i.e., $f_{X,Y}(x,y) \neq f_X(x) f_Y(y)$?
\begin{solution}
\begin{align*}
\partial_{x}\partial_{y}F_{X,Y}(x,y)
=\partial_{x}\partial_{y}F_{X}(x) F_{Y}(y)
=\partial_{x}F_{X}(x) \partial_{y} F_{Y}(y) = f_{X}(x) f_{Y}(y).
\end{align*}
\end{solution}
\end{exercise}

\begin{exercise}
BH define the conditional CDF given an event $A$ on page 416 as $F(y|A)$.
Use this definition to write $F_{X,Y}(x,y)/F_{X}(x)$ as a conditional CDF.
Is this equal to the conditional CDF of $X$ and $Y$?
\begin{solution}
\begin{align}
\frac{F_{X,Y}(x,y)}{F_{X}(x)} = \frac{\P{X\leq x, Y\leq y}}{\P{X\leq x}}
  = \P{Y\leq y, X\leq x|X\leq x} = \P{Y\leq y|X\leq x}.
\end{align}
It is a big mistake to write $F_{X,Y}(x,y) = \P{X=x, Y=y}$. If you wrote this, recheck the definitions of BH.
\end{solution}
\end{exercise}

\begin{exercise}
Let $X$ be uniformly distributed on the set $\{0,1,2\}$ and let $Y \sim \Bern{1/4}$; $X$ and $Y$ are independent.
\begin{enumerate}
\item Present a contingency table for $X$ and $Y$.
\item What is the interpretation of the column sums of the table?
\item What is the interpretation of the row sums of the table?
\item Suppose you would change some of the entries in the table. Are $X$ and $Y$ still independent?
\end{enumerate}
\begin{solution}
$\P{X=0, Y=0} = 1/3 \cdot 3/4$,
$\P{X=0, Y=1} = 1/3 \cdot 1/4$, and so on.

If we have one column with $Y=0$ and the other with $Y=1$, then the sum over the columns are $\P{Y=0}$ and $\P{Y=1}$. The row sum for row $i$ are  $\P{X=i}$.

Changing the values will (most of the time) make $X$ and $Y$ dependent. But, what if we changes the values such that  $\P{X=0, Y=0} =1$? Are $X$ and $Y$ then again independent? Check the conditions again.
\end{solution}
\end{exercise}

\begin{exercise}
A machine makes items on a day.
Some items, independent of the other items, are failed (i.e., do not meet the quality requirements).
What are $N$ and  $p$ in the context of the chicken-egg story of BH? What are the `eggs' in this context, and what is the meaning of `hatching'?
What type of `hatching' do we have here?
\begin{solution}
  The number of produced items (laid eggs) is $N$. The probability of hatching is $p$, that is, an item is ok. The hatched eggs are the good items.
\end{solution}
\end{exercise}

% \begin{exercise}
% Apply the chicken-egg story. Families enter a zoo in a given hour. Some families have one child, other two, and so on.
% What are the `eggs' in this context, and what is the meaning of `hatching'?
% \end{exercise}


\begin{exercise}
We have two rvs $X$ and $Y$ on $\R^{+}$. It is given that $F_{X,Y}(x,y) = F_X(x)F_Y(y)$ for $x,y \leq 1/3$. It is true that then  $X$ and $Y$ are necessarily independent.
\begin{solution}
For $X, Y$ to be independent, it is necessary that  $F_{X,Y}(x,y) = F_X(x)F_Y(y)$ for all $x,y$, not just one particular choice. (This is an example that satisfying a necessary condition is not necessarily sufficient.)
\end{solution}
\end{exercise}

\begin{exercise}
I select a random guy from the street, his height $X\sim\Norm{1.8, 0.1}$, and I select a random woman from the street, her height is $Y\sim\Norm{1.7, 0.08}$.
I claim that since I selected the man and the woman independently, their heights are independent.
Briefly comment on this claim.


\begin{solution}
  Many answers are possible here, depending on extra assumptions you make.
  Here is one.
  Suppose, just by change, the fraction of taller guys in the street is a bit higher than the population fraction.
  Assuming that taller (shorter) people prefer taller (shorter) spouses, there must be a dependence between the height of the men and the women. This is because when selecting a man, I can also select his wife.

From this exercise you should memorize that \emph{independence is a property of the joint CDF, not of the rvs}.

Mistake:   $\P{Y}$ is wrong notation wrong because we can only compute the probability of an event, such as $\{Y\leq y\}$. But $Y$ itself is not an event. \end{solution}
\end{exercise}


\begin{exercise}
For any two rvs $X$ and $Y$ on $\R^{+}$ with marginals $F_{X}$ and $F_{Y}$, can  it hold that $\P{X\leq x, Y\leq y} = F_{X}(x) F_{Y}(y)$?
\begin{solution}
Only when $X, Y$ are independent.

Mistake:  independence of $X$ and $Y$ is not the same as the linear independence. Don't confuse these two types of dependene.
\end{solution}

\end{exercise}

\begin{exercise}
Theorem 7.1.11. What is the meaning of the notation $X|N=n$?
\begin{solution}
  Given $N=n$, the random variable $X$ has a certain distribution, here binomial.
\end{solution}
\end{exercise}

\begin{exercise}
Let $X, Y$ be two discrete rvs with CDF $F_{X,Y}$.  Can we compute the PDF as $\partial_{x}\partial_{y} F_{X,Y}(x,y)$?
\begin{solution}
This claim is incorrect, because $X, Y$ are discrete, hence they have a PMF, not a PDF.

Mistake: Someone said that $\partial_{x}\partial_{y}$ is not correct notation; however, it is correct! It's a (much used) abbreviation of the much heaver $\partial^{2}/\partial x \partial y$. Next, the derivative of the PMF is not well-defined (at least, not within this course. If you object, ok, but then show that you passed a decent course on measure theory.)
\end{solution}
\end{exercise}

\begin{exercise} Redo BH.7.1.24 with indicator functions and the  fundamental bridge (recall, $\P{A} = \E{\1{A}}$ for an event $A$).
(Indicators are often  easy to use, and prevent many mistakes, as is demonstrated with this example.)
\begin{solution}
\begin{align*}
\P{T_1 < T_2 } = \E{\1{T_1<T_2}} =
&= \int_0^\infty \int_0^\infty \1{t_1<t_2} f_{T_1, T_2}(t_1, t_2)\d t_{1} \d{t_2} \\
&= \int_0^\infty \int_{t_1}^{\infty}  \lambda_1 e^{-\lambda_1 t_1} \lambda_2 e^{-\lambda_2 t_2}  \d{t_2} \d{t_1}\\
&= \int_0^\infty   \lambda_1 e^{-\lambda_1 t_1} \lambda_2 \int_{t_1}^{\infty} e^{-\lambda_2 t_2}  \d{t_2} \d{t_1}\\
&= \int_0^\infty   \lambda_1 e^{-\lambda_1 t_1} e^{-\lambda_2 t_1}  \d{t_1}\\
&= \int_0^\infty   \lambda_1 e^{-\lambda_1 t_1 - \lambda_2t_{1}} \d{t_1}\\
&= \frac{\lambda_{1}}{\lambda_{1}+\lambda_2}.
\end{align*}
\end{solution}
\end{exercise}


\section{Section 7.2}
\label{sec:section-7.2}


\begin{exercise}
BH.7.2.2. Write down the integral to compute $\E{(X-Y)^{2}}$, and solve it.
\begin{solution}
We have
\begin{align}
    \E{(X-Y)^2} &= \int_{-\infty}^\infty \int_{-\infty}^\infty (x - y)^2 f_{X,Y}(x,y) \d x \d y \\
    &=\int_0^1 \int_0^1 (x - y)^2 \d x \d y \\
    &=\int_0^1 \int_0^1 (x^{2} - 2 x y +  y^2) \d x \d y \\
    &=\int_0^1 \int_0^1 x^{2} \d x \d y
    -2\int_0^1 \int_0^1  x y  \d x \d y
    +\int_0^1 \int_0^1  y^2 \d x \d y  \\
    &=\int_0^1 x^{2} \d x
    -2\int_0^1 \int_0^1 x y  \d x \d y
    + \int_0^1  y^2  \d y \\
    &=1/3  -2 \cdot 1/ 2 \cdot 1/ 2 + 1/3.
\end{align}
\end{solution}
\end{exercise}


\begin{exercise}
Explain that for a continuous r.v. $X$ with CDF $F$ and $a$ and $b$ (so it might be that $a>b$),
\begin{equation}
  \label{eq:87}
\P{a< X < b} = [F(b) - F(a)]^{+}.
\end{equation}
\begin{hint}
  Recall that $F\in [0, 1]$.
\end{hint}
\begin{solution}
\begin{align}
a<b & \implies \P{a< X < b} = F(b) - F(a) = [F(b) - F(a)]^{+} \\
a\geq b & \implies \P{a< X < b} =  0 = [F(b) - F(a)]^{+},
\end{align}
where the last equality follows from the fact that $F$ is increasing.
\end{solution}
\end{exercise}

\begin{remark}
If you are like me, you underestimate at first the importance of using indicator functions. In fact, they are extremely useful for several reasons.
\begin{enumerate}
\item  They help to keep your formulas clean.
\item You can use them in computer code as logical conditions, or to help counting relevant events, something you need when numerically estimating multi-D integrals,  for machine learning for instance.
\item  Even though figures give geometrical insight into how to integrate over an 2D area, when it comes to reversing the sequence of integration, indicators are often easier to use.
\item In fact, \emph{expectation is the fundamental concept in probability theory}, and the \emph{probability of an event is defined} as
\begin{equation}
  \label{eq:84}
  \P{A} := \E{\1{A}}.
\end{equation}
Thus, the fundamental bridge is actually an application of LOTUS to indicator functions. Hence, reread BH.4.4!
\end{enumerate}
\end{remark}

\begin{exercise}
What is $\int_{-\infty}^{\infty} \1{0\leq x \leq 3} \d x$?
\begin{solution}
\begin{align*}
\int_{-\infty}^{\infty} \1{0\leq x \leq 3} \d x =\int_{0}^{3}  \d x  = 3.
\end{align*}
\end{solution}
\end{exercise}

\begin{exercise}
What is
\begin{equation}
\label{eq:85}
\int x \1{0\leq x \leq 4} \d x?
\end{equation}
\begin{solution}
\begin{align*}
\int x \1{0\leq x \leq 4} \d x  = \int_{0}^{4} x \d x = 16/2 = 8.
\end{align*}
\end{solution}
\end{exercise}

When we do an integral over a 2D surface we can first integrate over the $x$ and then over the $y$, or the other way around, whatever is the most convenient.
(There are conditions about how to handle multi-D integral, but for this course these are irrelevant.)

\begin{exercise}
What is
\begin{equation}
\label{eq:185}
\iint x y \1{0\leq x \leq 3}\1{0\leq y \leq 4} \d x \d y?
\end{equation}
\begin{solution}
\begin{align*}
\iint xy \1{0\leq x \leq 3}\1{0\leq y \leq 4} \d x \d y
&=\int_{0}^{3} x \int_{0}^{4} y \d y \d x\\
&=\int_{0}^{3} x \frac{y^{2}} 2 \biggr|_{0}^{4} \d x\\
&= \int_{0}^{3} x\cdot 8 \d x = 8\cdot 9/2 = 4\cdot 9.
\end{align*}
\end{solution}
\end{exercise}

\begin{exercise}
What is
\begin{align}
\label{eq:285}
\iint \1{0\leq x \leq 3} \1{0\leq y \leq 4}\1{x\leq y}\d x \d y?
\end{align}
\begin{solution}
Two solutions. First we integrate over $y$.
\begin{align}
\label{eq:385}
\iint \1{0\leq x \leq 3} \1{0\leq y \leq 4}\1{x\leq y}\d x \d y
&=\int \1{0\leq x \leq 3} \int \1{0\leq y \leq 4}\1{x\leq y}\d y \d x\\
&=\int \1{0\leq x \leq 3} \int \1{\max\{x, 0\} \leq y \leq 4}\d y \d x\\
&=\int_{0}^{3} \int_{\max\{x, 0\}}^{4}\d y \d x\\
&=\int_{0}^{3} y\biggr|_{\max\{x, 0\}}^{4} \d x\\
&=\int_{0}^{3}  (4-\max\{x, 0\}) \d x\\
&=12 - \int_{0}^{3} \max\{x, 0\} \d x\\
&=12 - \int_{0}^{3} x  \d x\\
&=12 - 9/2.
\end{align}

Let's now instead first integrate over $x$.
\begin{align}
\label{eq:485}
\iint \1{0\leq x \leq 3} \1{0\leq y \leq 4}\1{x\leq y}\d x \d y
&= \int \1{0\leq y \leq 4} \int \1{0\leq x \leq 3} \1{x\leq y}\d x \d y\\
&= \int_{0}^{4} \int \1{0\leq x \leq \min\{3, y\}}\d x \d y\\
&= \int_{0}^{4} \int_{0}^{\min\{3, y\}} \d x \d y\\
&= \int_{0}^{4} \min\{3, y\}\d y\\
&= \int_{0}^{3} \min\{3, y\}\d y + \int_{3}^{4} \min\{3, y\}\d y\\
&= \int_{0}^{3} y \d y + \int_{3}^{4}  3\d y\\
&= 9/2 + 3.
\end{align}
\end{solution}
\end{exercise}


\begin{exercise}\label{ex:2a}
Take $X\sim\Unif{[1,3]}, Y\sim\Unif{[2,4]}$ and independent. Compute
\begin{equation}
  \label{eq:824}
\P{Y\leq 2X}.
\end{equation}
\begin{solution}
Take $c$ the normalization constant (why is $c=1/4$), then using the previous exercise
\begin{align}
\P{Y\leq 2X}
&=\E{\1{Y\leq 2X}} \\
&=c \int_{1}^{3}\int_{2}^{4} \1{y\leq 2x} \d y \d x \\
&=c \int_{1}^{3}\int \1{2\leq y\leq \min\{4,2x\}}  \d y \d x \\
&=c \int_{1}^{3} (\min\{4, 2x\} -2) \d x
\end{align}
Now make a drawing of the function $(\min\{4, 2x\} - 2)$ on the interval $[1,3]$ to see that
\begin{equation}
\int_{1}^{3} (min\{4, 2x\} -2) \d x = \int_{1}^{2} (2x -2) \d x + \int_{2}^{3} (4 -2) \d x.
\end{equation}
I leave the rest of the computation to you.
\end{solution}
\end{exercise}



\section{Section 7.3}
\label{sec:section-7.3}



\begin{exercise}
Give a brief example of a situation where it might be more convenient to employ the correlation than the covariance.  Explain why.
\begin{solution}
The covariance might be a large number, which may  suggest that the rvs $X$ and $Y$ are `very' dependent. However, when $\V X$ and $\V Y$ are also large, the correlation can be small. Thus, correlation is a scaled type of covariance.
\end{solution}
\end{exercise}

\begin{exercise}
In queueing theory  the concept of squared coefficient of variance $SCV$ of a rv $X$ is very important. It is defined as $C = \V{X}/(\E X)^{2}$. Is the SCV of $X$ equal to $\text{Corr}(X,X)$? Can it happen that $C>1$?
\begin{solution}
Answers: no and yes.

We have
\begin{align}
    C = \frac{\V{X}}{(\E{X})^2},
\end{align}
which does not equal
\begin{align}
    \text{Corr}(X,X) = \frac{\cov{X,X}}{\sqrt{\V{X}\V{X}}} = 1
\end{align}
in general (for instance, consider a degenerate random variable $X \equiv 1$). Next, consider a $N(1,100)$ random variable. Then,
\begin{align}
    C = 100/(1^2) = 100 > 1.
\end{align}
\end{solution}
\end{exercise}


\begin{exercise}
Prove the key properties 1--5 of the covariance below BH.7.3.2.
\begin{solution}
\begin{enumerate}
    \item We have
    \begin{align}
        \cov{X,X} = \E{XX} - \E{X}\E{X} = \E{X^2} - \E{X}^2 = \V{X}.
    \end{align}
    \item We have
    \begin{align}
        \cov{X,Y} = \E{XY} - \E{X}\E{Y} = \E{YX} - \E{Y} \E{X} = \cov{Y, X}.
    \end{align}
    \item We have
    \begin{align}
        \cov{X,c} = \E{Xc} - \E{X}\E{c} = c\E{X} - c\E{X} = 0.
    \end{align}
    \item We have
    \begin{align}
        \cov{aX,Y} = \E{aXY} - \E{aX}\E{Y} = a\big(\E{XY} - \E{X}\E{Y}\big) = a\cov{X,Y}.
    \end{align}
    \item We have
    \begin{align}
        \cov{X+Y, Z} &= \E{(X + Y)Z} - \E{X+Y}\E{Z} \\
        &= \E{XZ + YZ} - \big(\E{X} + \E{Y}\big) \E{Z}\\
        &= \E{XZ} - \E{X}\E{Z} + \E{YZ} - \E{Y}\E{Z} \\
        &= \cov{X,Z} + \cov{Y,Z}.
    \end{align}
\end{enumerate}
\end{solution}
\end{exercise}

\begin{exercise}
Using the definition of Covariance (BH.7.3.1) derive the expression $\cov{X,Y}=\E{XY}-\E{X}\E{Y}$. Use this to show why independence of X and Y implies their uncorrelatedness (Note that the converse does not hold).
\begin{solution}
We have
\begin{align}
    \cov{X,Y} &= \E{(X - \E{X})(Y - \E{Y})} \\
    &= \E{XY - X\E{Y} - Y\E{X} + \E{X}\E{Y}} \\
    &= \E{XY} - \E{X}\E{Y} - \E{Y} \E{X} + \E{X}\E{Y}\\
    &= \E{XY}-\E{X}\E{Y}.
\end{align}
When $X$ and $Y$ are independent, then $\E{XY} = \E X \E Y$, and then $\cov{X,Y}=0$.
\end{solution}
\end{exercise}

\begin{exercise}
Let $U, V$ be two rvs and let $a,b\in \R$.
Use the previous question to express $\cov{a(U+V), b(U-V)}$ in terms of $\V{U}$, $\V{V}$ and $\cov{U,V}$.
\begin{solution}
By linearity of the covariance wea have
\begin{align}
    \cov{a(U+V), b(U-V)} &= a \Big( \cov{U, b(U-V)} + \cov{V, b(U-V)} \Big) \\
    &= a \Big( b\big( \cov{U, U} - \cov{U, V} \big)  + b\big( \cov{V, U} - \cov{V,V} \Big) \\
    &= a \Big( b\big( \cov{U, U} - \cov{U, V} \big)  + b\big( \cov{V, U} - \cov{V,V} \Big) \\
    &= a b \Big( \V{U} - \cov{U, V} + \cov{V, U} - \V{V} \Big) \\
    &= a b \Big( \V{U} - \V{V} \Big).
\end{align}
Alternatively one can also use the result from BH.7.1.26, according to which $\cov{X,Y} = \E{XY}-\E{X}\E{Y}$.
\end{solution}
\end{exercise}


\begin{exercise}\label{ex:3a}
The solution of BH.7.3.6 is a somewhat tricky; I would have not  found this trick myself. Here is an approach that is trick free.

Neglecting the event $\{X=Y\}$ as this has zero probability, we know that $M=X, L=Y$ or $M=Y, L=X$. Use this idea and the formula $\cov{M,L} = \E{ML} - \E M \E L$ to derive the result of this example.
\begin{hint}
Realize that $\E{ML} = \E{XY}$.
\end{hint}
\begin{solution}
With the hint: $\E{XY}=1/\lambda^{2}$, when $X, Y \sim\Exp{\lambda}$. Then, $L\sim \Exp{2\lambda}$, since $f_{L}(x) = 2 f_X(x) (1-F_Y(x)) = 2 \lambda e^{-2\lambda x}$. Therefore, $\E L = 1/2\lambda$. Also, by memoryless, $\E{M} = \E L + \E X = 3/2\lambda$. Hence, $\E{M}\E L = 3/4\lambda^{2}$. Hence, $\E{ML}- \E M \E L = 1/\lambda^{2-} 3/4\lambda^2 = 1/4\lambda^{2}$.
\end{solution}
\end{exercise}



\section{Section 7.4}
\label{sec:section-7.4}


\begin{exercise}
Come up with a short illustrative example in which the random vector $\mathbf{X} = (X_1, \ldots, X_6)$ follows a Multinomial Distribution with parameters  $n=10$ and $\mathbf{p}=(\frac{1}{6}, ..., \frac{1}{6}) \in \R^{6}$.
\begin{solution}
We throw 10 fair dice. $X_i$ denotes the number of dice that show the number $i$, $i=1,\ldots,6$.
\end{solution}
\end{exercise}


\section{Section 7.5}



\begin{exercise}
Is the following claim correct? If the rvs $X, Y$ are both normally distributed, then $(X, Y)$ follows a Bivariate Normal distribution.
\begin{solution}
No, this does not always hold, see BH.7.5.2. However, it does hold when $X$ and $Y$ are independent.
\end{solution}
\end{exercise}
\begin{exercise}
Let $X, Y, Z$ be iid $\mathcal{N}(0,1).$ Determine whether or not the random vector
\begin{align*}
    \mathbf{W} = (X+2Y, 3X+4Z, 5Y+6Z, 2X-4Y+Z, X-9Z, 12X+\sqrt{3}Y -\pi Z + 18)
\end{align*}
is Multivariate Normal. (Explain in words, don't do a lot of tedious math here!)
\begin{solution}
Since $X,Y,Z$ are independent normally distributed variables, $(X,Y,Z)$ is multivariate normally distributed. Hence, every linear combination of $X,Y,Z$ is normally distributed. Note that every linear combination of the elements of $W$ can be written as a linear combination of $X,Y,Z$. Hence, every linear combination of the elements of $W$ is normally distributed. Hence, $W$ is multivariate normally distributed.
\end{solution}
\end{exercise}



\end{document}



%%% Local Variables:
%%% mode: latex
%%% TeX-master: "study-guide.tex"
%%% End:
