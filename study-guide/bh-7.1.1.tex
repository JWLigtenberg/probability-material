\documentclass[study-guide-sol]{subfiles}
\externaldocument{study-guide}

\opt{solutionfiles,check}{
\Opensolutionfile{hint}
\Opensolutionfile{ans}
}

\begin{document}


\section{Our exercises on Section BH.7.1.1}

\begin{exercise}
In your own words, explain what is
\begin{enumerate}
\item a joint PMF, PDF, CDF;
\item a conditional PMF, PDF, CDF;
\item a marginal PMF, PDF, CDF.
\end{enumerate}
\begin{solution}
Check the definitions of the book.

Mistake: To say that $\P{X=x}$ is the PMF for a continuous random variable is wrong, because $\P{X=x}=0$ when $X$ is continuous.

Why is $\P{1 < x \leq 4}$ wrong notation?
hint: $X$ should be a capital.
What is the difference between $X$ and $x$?

\end{solution}
\end{exercise}



\begin{exercise}
Suppose the probability of obtaining a head twice out of two coin flips is $\P{X_1=H, X_2=H}$.
What has this to do with joint PMFs? Can you generalize this idea to other examples?
\begin{solution}
This example shows why joint distributions are important!
In any experiment that involves a sequence of measurements, such as multiple throws of a coin, or the weighing of a bunch of chimpanzees, we have to deal with joint CDFs and PMFs.
\end{solution}
\end{exercise}


\begin{exercise}
In the previous exercise, suppose the outcome of the second throw is always equal to that of the first. Specify the joint PMF.
\begin{solution}
Here, we deal with two rvs, and we have to specify how they depend. In the present case $\P{X_1= H, X_2=H} = \P{X_1=H}$ and $\P{X_1= T, X_2=T} = \P{X_1=T}$, $\P{X_1= H, X_2=T} = \P{X_1= T, X_2=H} = 0$. Note that with this, we specified the joint PMF on all possible outcomes.
\end{solution}
\end{exercise}


\begin{exercise}
Let $X$ be uniformly distributed on the set $\{0,1,2\}$ and let $Y \sim \Bern{1/4}$; $X$ and $Y$ are independent.
\begin{enumerate}
\item Present a contingency table for $X$ and $Y$.
\item What is the interpretation of the column sums of the table?
\item What is the interpretation of the row sums of the table?
\item Suppose you would change some of the entries in the table. Are $X$ and $Y$ still independent?
\end{enumerate}
\begin{solution}
$\P{X=0, Y=0} = 1/3 \cdot 3/4$,
$\P{X=0, Y=1} = 1/3 \cdot 1/4$, and so on.

If we have one column with $Y=0$ and the other with $Y=1$, then the sum over the columns are $\P{Y=0}$ and $\P{Y=1}$. The row sum for row $i$ are  $\P{X=i}$.

Changing the values will (most of the time) make $X$ and $Y$ dependent. But, what if we changes the values such that  $\P{X=0, Y=0} =1$? Are $X$ and $Y$ then again independent? Check the conditions again.
\end{solution}
\end{exercise}


\begin{exercise}
A machine makes items on a day.
Some items, independent of the other items, are failed (i.e., do not meet the quality requirements).
What are $N$ and  $p$ in the context of the chicken-egg story of BH? What are the `eggs' in this context, and what is the meaning of `hatching'?
What type of `hatching' do we have here?
\begin{solution}
  The number of produced items (laid eggs) is $N$. The probability of hatching is $p$, that is, an item is ok. The hatched eggs are the good items.
\end{solution}
\end{exercise}

% \begin{exercise}
% Apply the chicken-egg story. Families enter a zoo in a given hour. Some families have one child, other two, and so on.
% What are the `eggs' in this context, and what is the meaning of `hatching'?
% \end{exercise}


\begin{exercise}
Theorem 7.1.11. What is the meaning of the notation $X|N=n$?
\begin{solution}
  Given $N=n$, the random variable $X$ has a certain distribution, here binomial.
\end{solution}
\end{exercise}

\end{document}

