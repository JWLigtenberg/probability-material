\section{Question}
\begin{exercise}[1.5]
Let $U\sim\text{Unif}(-1,1)$. Find the PDF of $B = \left|U\right|$. What is its distribution? What is $\E{B}$?
\begin{solution}
Notice that the function $\abs{x}$ is not one-to-one on $(-1,1)$, hence we cannot use the transformation theorem here. We see that since $U\in(-1,1)$, $B\in[0,1)$. Then we see that
\begin{align*}
    F_B(y) = \P{B\leq y} = \P{\abs{X}\leq y} = \P{-y\leq X \leq y} = F_X(y) - F_X(-y)
\end{align*}
For $y\in[0,1]$. We know $F_X(y) = \frac{y+1}{2}$, so then $F_X(y) - F_X(-y) = \frac{y+1}{2} - \frac{-y+1}{2} = y$, and we conclude that $B\sim\text{Unif}(0,1)$. Then we  know $\E{B} = \frac12$.
\\\\
Grading scheme:
\begin{itemize}
    \item Correct derivation with CDF 0.5pt
    \item No mistakes in the above 0.5pt.
    \item Expectation 0.5pt.
\end{itemize}
\end{solution}
\end{exercise}

\begin{exercise}[1.5]
Let $X$ be a continuous random variable such that $M_X(t) = e^t M_X(-t)$. What is $\E{X}$? Can you conclude that $X$ is distributed in the same manner as $B$?
\begin{solution}
Using the given relation of $M_X$, we see that
\begin{align*}
    M_X(t) = e^t M_X(-t) = e^t\E{e^{-t X}} = \E{e^{t(1-X)}} = M_{1-X}(t)
\end{align*}
and since the MGF determines the distribution we can immediately say that $X\sim 1-X$. Then it must be that $\E{X} = \E{1-X}$ and then by linearity we have that $\E{X} = \frac12$. This is not enough information to conclude what distribution $X$ has, we can only see that it must be symmetric around $\frac12$.
\\\\
Grading scheme:
\begin{itemize}
    \item Note $X\sim1-X$ 0.5pt.
    \item Correct expectation 0.5pt.
    \item Cannot conclude the same distribution 0.5pt.
    \item \textbf{OR:} Correctly solved the differential equation/took the derivative and concluded the result 1pt.
    \item Cannot conclude the same distribution 0.5pt.
\end{itemize}
\end{solution}
\end{exercise}

\begin{exercise}[1]
Let $B$ be as you found it in part (a). Find the CDF of $X = \kappa + \lambda\ln{\left(\frac{B}{1-B}\right)}$.
\begin{solution}
As usual, we start with the CDF of $B$, this is known to be $F_B(y) = y$ for $y\in[0,1]$ (and 1 for $y>1$, 0 for $y<0$). Then we have that
\begin{align*}
    F_X(y) &= \P{X\leq y}\\
    &= \P{\kappa + \lambda\ln{\left(\frac{B}{1-B}\right)}\leq y}\\
    &= \P{\ln{\left(\frac{B}{1-B}\right)} \leq \frac{y-\kappa}{\lambda}}\\
    &= \P{B\leq \frac{\exp{\frac{y-\kappa}{\lambda}}}{1+\exp{\frac{y-\kappa}{\lambda}}}}\\
    &= F_B\left(\frac{\exp{\frac{y-\kappa}{\lambda}}}{1+\exp{\frac{y-\kappa}{\lambda}}}\right)\\
    &= \frac{\exp{\frac{y-\kappa}{\lambda}}}{1+\exp{\frac{y-\kappa}{\lambda}}}
\end{align*}
for $y\in\bf{R}$.
\\\\
Grading scheme:
\begin{itemize}
    \item CDF technique derivation 0.5pt.
    \item No mistakes 0.5pt.
\end{itemize}
\end{solution}
\end{exercise}

\begin{exercise}[1]
Let $\kappa = 0$, $\lambda = 1$. The quantile function $Q_X(\cdot)$ is defined to be the function such that $Q_X(F_X(x)) = x$. Find $Q_X(\cdot)$ for the random variable $X$ as in part (c). You may assume $F_X(x)$ to be strictly increasing without proof. This function $Q_X$ is known as the `log-odds', or `logit' function, and is used often in regression analysis to model a binary random variable.
\begin{solution}
Notice that $F_X$ maps the real line onto the interval $(0,1)$. Then, $Q_X$ must map the interval $(0,1)$ onto $\bf{R}$. We find the inverse of the CDF of $X$ as follows:
\begin{align*}
    &z = F_X(y)\implies\\
    &z = \frac{e^{y}}{1+e^{y}}\implies\\
    &z = \frac{1}{1+e^{-y}}\implies\\
    &e^{-y}z = 1-z\implies\\
    &e^y = \frac{z}{1-z}\implies\\
    &y = \ln{\frac{z}{1-z}} = Q_X(z)
\end{align*}
for $z\in(0,1)$.
\\\\
Grading scheme:
\begin{itemize}
    \item Mention the idea to invert the CDF 0.5pt. (of course this includes the people who did so)
    \item Correct inversion 0.5pt.
    \item Correct bounds for the quantile function 0.5pt bonus.
\end{itemize}
\end{solution}
\end{exercise}
