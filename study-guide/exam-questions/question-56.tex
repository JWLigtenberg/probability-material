\section*{Question}

Amy is playing a game. She throws a basketball at a hoop and counts the number of times she successfully throws the ball through the hoop. She keeps counting until she has missed $r$ times, at which moment the current round of the game stops. Her score for the round is the total number of successful throws in the round. Amy plays $n$ rounds in total. We assume that all throws are independent and have the same (unknown) success probability $p$. Amy is interested in finding out her skill level. That is, she is interested in the value of $p$.

Given the value of $p$, Amy's score $X_i$ for the $i$th round of the game follows a negative binomial distribution with parameters $r$ and $p$. That is, for every $i=1,\ldots,n$, we have that $X_i|p \sim \text{NB}(r,p)$, with a corresponding pmf defined by
\begin{align}
    \P{X_i = x_i | p} &= {x_i + r - 1 \choose x_i} (1 - p)^r p^{x_i},
\end{align}
for $x_i = 0,1,2,\ldots$. Amy's prior belief about the distribution of $p$ is that it follows a $\text{Beta}(a,b)$ distribution with given values for $a$ and $b$ (the exact values of $a$ and $b$ are not relevant for this question).

\begin{exercise}[2.5]
In the first round, Amy gets a score of $X_1 = x_1$. Find Amy's \textit{posterior} distribution of $p$, given this observation.
\begin{solution}
Using Bayes' rule we have
\begin{align}
    f_1(p|X_1 = x_1) &= \frac{\P{X_1 = x_1 | p} f_0(p)}{\P{X_1 = x_1}} \\
    &\propto \P{X_1 = x_1 | p} f_0(p) \\
    &= {x_1 + r - 1 \choose x_1} (1 - p)^r p^{x_1} \frac{1}{B(a,b)} p^{a-1} (1-p)^{b-1} \\
    &\propto (1 - p)^{r+b-1} p^{a+x_1-1},
\end{align}
in which we recognize the pdf of a $\text{Beta}(a+x_1, b + r)$ distribution (up to a constant factor). Hence, the posterior distribution of $p$ given $X_1 = x_1$ is a $\text{Beta}(a+x_1, b + r)$ distribution.
\end{solution}
\end{exercise}

\begin{exercise}[1]
Is Amy's prior distribution a \textit{conjugate} prior?
\begin{solution}
Yes. The posterior distribution is in the same family of distributions (Beta) as the prior. Hence, Amy has a conjugate prior.
\end{solution}
\end{exercise}

\begin{exercise}[1.5]
Suppose Amy plays $n$ rounds and observes the scores $X_1 = x_1, \ldots, X_n = x_n$. What is Amy's posterior distribution after these observations? \\
\textit{Hint: you don't need to do a lot of math here! Instead, use the result of the first question above and a logical argument.}
\begin{solution}
The posterior after observing $X_1 = x_1$ becomes our new prior. Hence, our new prior is a $\text{Beta}(a+x_1,b+r)$ distribution. From question 1 it follows that the prior after observing $X_2 = x_2$ then is a $\text{Beta}(a+x_1 + x_2,b+ 2r)$ distribution. Hence, iterating this process, we find that the posterior distribution of $p$ after observing $X_1 = x_1, \ldots, X_n = x_n$ is a $\text{Beta}(a+\sum_{i=1}^n x_i,b+ rn)$ distribution.
\end{solution}
\end{exercise}