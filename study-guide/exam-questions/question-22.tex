\section{Question}

Let $X$ and $Y$ be i.i.d and Unif(1,3) distributed.
\begin{exercise}[0.5]
Find the joint PDF $f(x,y)$ of $X$ and $Y$.
\begin{solution}
    Since $X$ and $Y$ are independent, their joint PDF is the product of their marginal PDFs. This gives:
    \begin{align*}
        f(x,y) = \Big(\frac{1}{(3-1)}\Big) \Big(\frac{1}{(3-1)}\Big) = \frac{1}{4}
    \end{align*}
    For $1<x<3$ and $1<y<3$ and 0 otherwise. \\
    \textit{One mistake, zero points.}
\end{solution}
\end{exercise}

\begin{exercise}[0.5]
Show that $\int_{-\infty}^\infty \int_{-\infty}^\infty f(x,y) \; dx \; dy = 1$
\begin{solution}
    Calculating this integral gives:
    \begin{align*}
      \int_{-\infty}^\infty \int_{-\infty}^\infty f(x,y) \; dx \; dy =& \int_1^3 \int_1^3
      \frac{1}{4} \; dx \; dy \\
      =& 1
    \end{align*} \\
    \textit{One mistake, zero points.}
\end{solution}
\end{exercise}

\begin{exercise}[3]
Find $SD(|X-Y|)$, the standard deviation of the distance between $X$ and $Y$.
\begin{solution}
Step 1. Find the expectation $E(|X-Y|)$. Using LOTUS.
\begin{align*}
    E(|X-Y|) =& \int_1^3 \int_1^3 |x-y| \; \Big(\frac{1}{4}\Big) \; dx \; dy \\
      =& \int_1^3 \int_y^3 (x-y) \; \Big(\frac{1}{4}\Big) \; dx \; dy + \int_1^3 \int_1^y (y-x) \; \Big(\frac{1}{4}\Big) \; dx \; dy \\
      =& \frac{1}{3} + \frac{1}{3} \\
      =& \frac{2}{3}
\end{align*}
Step 2. Find the squared expectation $|X-Y|^2$. Using LOTUS.
\begin{align*}
    E(|X-Y|^2) =& \int_1^3 \int_1^3 |x-y|^2 \; \Big(\frac{1}{4}\Big) \; dx \; dy \\
    =& \int_1^3 \int_1^3 (x-y)^2 \; \Big(\frac{1}{4}\Big) \; dx \; dy \\
    =& \int_1^3 \int_1^3 (x^2- 2xy + y^2) \; \Big(\frac{1}{4}\Big) \; dx \; dy \\
    =& \frac{2}{3}
\end{align*}
Step 3. Find the variance of $|X-Y|$.
\begin{align*}
    \V{|X-Y|} =& E(|X-Y|^2) - E(|X-Y|)^2 \\
        =& \frac{2}{3} - \Big(\frac{2}{3}\Big)^2 \\
        =& \frac{2}{9}
\end{align*}
Step 4. Find the standard deviation of $|X-Y|$.
\begin{align*}
    SD(|X-Y|) =& \sqrt{\V{|X-Y|}} \\
    =& \sqrt{\frac{2}{9}}\\
    =& 0.4714
\end{align*} \\
\textit{One point for writing down the integral for $E|X-Y|$ and splitting it up correctly. One point for $E|X-Y|^2$. One point for finding $SD(|X-Y|)$ in the correct way.}
\end{solution}
\end{exercise}
\noindent Consider the following code:
\begin{minted}{python}
import numpy as np
np.random.seed(3)

num = 100000

x = np.random.normal(loc = 50, scale = 200, size = num)

result1 = np.zeros(num)
for i in range(0,num):
    result1[i] = abs(x[i]-50)<2*200)

print(np.sum(result1)/num)
\end{minted}

\begin{exercise}[0.5]
What does the code above do?
\begin{solution}
It loads the required packages and creates one sample with 100000 observations from a $\mathcal{N}(50, 200)$-distribution. Then for all observations it subtracts its mean and tests if the new value is within 2 standard deviations of the mean. \\
\textit{0.5 points for mentioning the mean is subtracted and it is checked if the value found is smaller than 2 times the s.d.}
\end{solution}
\end{exercise}
\begin{exercise}[0.5]
The code gives as output 0.95429.
Explain why you would expect to get this output from the code. \textit{Hint:} use Theorem 5.4.5 in the book.
\begin{solution}
By Theorem 5.4.5 we get that $P(|X-\mu|<2\sigma) \approx 0.95$, this is also shown in the code. \\
\textit{0.5 points for making a comparison between the theorem and the answer in the code. Conclusion should be that they give similar results.}
\end{solution}
\end{exercise}