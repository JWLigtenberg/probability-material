\section{Question}

Let $Y \sim \Norm{0,1}$. 
In this exercise, we find an upper bound for $\P{|Y| > 3}$.

\begin{exercise}[1.5]
 Let $g$ be a positive and increasing function, and let $Z$ be a r.v. 

Consider the following inequality:
$$\P{Z \geq a} = \P{g(Z) \geq g(a)}   \leq \frac{\E{g(Z)}}{g(a)}.$$
\begin{enumerate}
\item[(i)] Explain why $\P{Z \geq a} = \P{g(Z) \geq g(a)}$ holds. 
\item[(ii)] Explain why $\P{g(Z) \geq g(a)}   \leq \frac{\E{g(Z)}}{g(a)}$ holds. 
\end{enumerate}
Make sure to clearly indicate where you use that  $g$ is positive and increasing. \\
\begin{solution}
(i). Since $g$ is increasing, we have $Z \geq a$ if and only if $g(Z) \geq g(a)$, so $\{Z \geq a\}$ and $\{g(Z) \geq g(a)\}$ are the same event. Hence,  $\P{Z \geq a} = \P{g(Z) \geq g(a)}$.  \\
(ii). Since $g$ is positive, we have $|g(Z)| = g(Z)$ and $g(a) > 0$. Hence, the inequality follows directly from Markov's inequality with r.v. $g(Z)$ and constant $g(a) > 0$. \\ \\
\noindent Remarks and grading scheme:
\begin{itemize}
\item In part (ii), it is not sufficient to just say that it holds by Markov's inequality, also explain how you apply Markov's inequality. 
\item Since it was asked to indicate \textbf{clearly} where you use that  $g$ is positive and increasing, don't just say "since $g$ is positive and increasing" twice. For part (i) the positivity is not needed, it is only needed that $g$ is increasing. For part (ii), it is not needed that $g$ is increasing, only the positivity is needed. Moreover, some explanation should be given how it is used. 
\item $g$ is not necessarily a linear transformation, as some of you claimed.
\item Not every positive and increasing function is convex or concave. \\ 
And using Jensen's inequality certainly doesn't work in (i): you are asked to prove an equality of probabilities, not an inequality of expectations. 
\item In part (i), the \textbf{if and only if} is really important. If you just mention that $g(Z) \geq g(a)$ if $Z \geq a$, then you are only proving that $\P{Z \geq a} \leq \P{g(X) \geq g(a)}$, because if you just say "$g(Z) \geq g(a)$ if $Z \geq a$",  $g(Z) \geq g(a)$ could still be true in cases where $Z \geq a$ is not, and hence  $g(Z) \geq g(a)$  can still have a larger probability. (I've not been strict on this when grading.)
\item Grading: 0.5 for a sufficient explanation for (i), 0.5 for a sufficient explanation for (ii) and 0.5 for clearly indicating where it is used that $g$ is positive and increasing and writing a clear answer overall. 
\end{itemize}
\end{solution}
\end{exercise}




\begin{exercise}[1]
Prove that $\P{|Y| > 3} \leq e^{-9t}\E{e^{t Y^2}}$ for $t>0$. \\
\begin{solution}
Note that $g(x) = e^{t x^2}$ is positive and increasing on $(0,\infty)$ for $t>0$. By applying the inequality of the first question with $a=3$ we find 
$$\P{|Y| > 3} \leq e^{-9t}\E{e^{t |Y|^2}} = e^{-9t}\E{e^{t Y^2}}.$$
\noindent Remarks and grading scheme:
\begin{itemize}
\item Most of you did not apply the inequality from the previous part to solve this question, but instead used Chernoff's inequality as follows:
$$\P{|Z| > 3} = \P{Z^2 > 9} \leq e^{-9t}\E{e^{t |Z|^2}} = e^{-9t}\E{e^{t Z^2}},$$
where the first equality holds since $|Z| > 3$ if and only if $\P{Z^2 > 9}$. This is also correct, but takes a bit more time. 
\item Don't write nonsense like $e^{-3t}\E{e^{t |Z|}} = e^{-9t}\E{e^{t Z^2}}$, just to make it look like you solved the exercise although you didn't. 
\item Grading scheme: 1 point for solving the exercise. Partial credit (0.5) for a correct and relevant application of Chernoff's inequality.
\end{itemize}
\end{solution}
\end{exercise}


\begin{exercise}[2.5]
For which $t$ do we find the best upper bound for  $\P{|Y| > 3}$? Also calculate the upper bound for this value of $t$.

\textit{Hint 1}. You may use that if $X \sim \chi^2_1$, then the MGF of $X$ is given by $M_X(t) = (1-2t)^{-1/2}$ for $t < 1/2$. However, you should explain clearly how you use this fact.

\textit{Hint 2}. Do not forget to check the second order condition of minimization. \\
\begin{solution}
Since $Y^2 \sim \chi^2_1$, we have $\E{e^{t Y^2}}  = \E{e^{t X}} = M_X(t) =  (1-2t)^{-1/2}$. 

So we minimize $ e^{-9t}\E{e^{t Y^2}} = e^{-9t}(1-2t)^{-1/2}$. It is easier if we take the logarithm first and minimize $-9t - \tfrac12 \log(1-2t)$. Its derivative to $t$ is $-9 + \frac1{1-2t}$, so setting the derivative to 0 yields $t = 4/9$. The second derivative to $t$ is  $\frac{2}{(1-2t)^2} > 0$ (the value at $t=4/9$ is $162$), so the second order condition holds.

%Without logarithm, the first derivative is $ e^{-9t} \left(-9(1-2t)^{-1/2} + (1-2t)^{-3/2} \right)$ and the second derivative is $ e^{-9t} \left(81(1-2t)^{-1/2} -18(1-2t)^{-3/2} + 3(1-2t)^{-5/2} \right)$; the value of the second derivative at $t=4/9$ is $e^{-4} \cdot 486 \approx 8.9014 > 0$.

This yields $\P{|Y| > 3} \leq e^{-4}(1-8/9)^{-1/2} \approx 0.0549$. \\

\noindent Remarks and grading scheme:
\begin{itemize}
\item Unless explicitly noted, you have to give an analytical answer and derive everything using just pen and paper. 
\item It is essential that you include some derivations for computing the (second) derivative. 
\item It is not necessary to take the logarithm first, but it makes the derivation easier, and hence it is less likely that you make errors. 
\item Grading scheme: 0.5 for arguing that $\E{e^{t Z^2}}  =  (1-2t)^{-1/2}$ with sufficient explanation; \\ 1 for taking the derivative and calculating that it is 0 at $t=4/9$; (0.5 if small mistake is made but resulting $t$ satisfies $0 < t < 1/2$, or if it is remarked that this should be the case); \\ 
0.5 for calculating the second derivative and checking the second order condition; \\ 0.5 for filling in $t=4/9$ to provide the upper bound (if an incorrect value of $t$ is found, this point can be given only if the resulting bound is between 0.0001 and 1, or if it is explicitly noted that the answer does not make sense). 
\end{itemize}
\end{solution}
\end{exercise}


