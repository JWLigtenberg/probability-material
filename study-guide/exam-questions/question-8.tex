\section{Question}


A server  spends a random amount $T$ on a job.
While the server works on the job, it sometimes gets interrupted to do other tasks $\{R_{i}\}$.
Such tasks need to be repaired before the machine can continue working again.
It is clear that such interruptions can only occur when the server is working.
Assume that the interruptions arrive according to a Poisson process with rate $\lambda$, i.e., the number of failures $N(t) \sim\Pois{\lambda t}$.
The interruptions $\{R_{i}\}$ are independent of $N$ and of $T$ and form an iid sequence with common mean $\E R$ and variance $\V R$.
The  duration $S$ of a job  is its own service time $T$ plus all interruptions, hence $S = T+ \sum_{i=1}^{N(T)} R_{i}$.

\begin{exercise}[1.5]
The computation below consists of a number of steps, a, b, \ldots. Explain for each step which property is used to ensure the step is true.
\begin{align}
\E{\sum_{i=1}^{N(t)} R_{i}\given N(t)= n} &\is{a} \E{\sum_{i=1}^{n} R_{i}} \\
  &\is{b} n \E R \\
\E{\sum_{i=1}^{N(t)} R_{i}\given N(t)} &\is{c} N(t) \E R
\end{align}
\begin{solution}
  We can use the result of part 1.
  \begin{enumerate}[a.]
  \item  On the set $\{N(t)=n\}$ $N(t)=n$. Hence we can replate  $N(t)$ by $n$.
  \item  Linearity of expectation
  \item  definition of conditional expectation.
  \end{enumerate}
Each  property missed, e.g, linearity of expectation, minus 0.5.
\end{solution}
\end{exercise}

\begin{exercise}[2]
  Suppose $R$ is equal to the constant $r$ and $T\sim\Exp{\mu}$, compute $\E S$.
\begin{solution}
    \begin{align*}
      \E S &= \E T  + \E{N(T)}\E R= 1/\mu+ r \lambda \E T = 1/\mu + r \lambda /\mu.
    \end{align*}
Typically, 0.5 point for each correct line, but I subtracted points If you say that $\E{N(T) \E R} = \E R$, or write $n\E R$ as final answer (apparently you did not get the idea that $N$ is an rv.)

\end{solution}
\end{exercise}


\begin{exercise}[1.5]
Explain what  this code computes.
  \begin{minted}{python}
import numpy as np

labda = 0.5
size = 10
num_runs = 50


def do_run():
    T = np.random.uniform(0, 20)
    N = np.random.poisson(labda * T)
    R = np.random.uniform(1, 5, size=N)
    S = T + R.sum()
    return S


samples = np.zeros(num_runs)
for i in range(num_runs):
    samples[i] = do_run()


print(samples[samples > 4].var())
  \end{minted}

 \begin{minted}{R}
labda <- 0.5
size <- 10
num_runs <- 50

do_run <- function() {
  bigT <- runif(n = 1, min = 0, max = 20)
  N <- rpois(n = 1, labda * bigT)
  R <- runif(n = N, min = 1, max = 5)
  S <- bigT + sum(R)
  return(S)
}

samples <- rep(0, num_runs)
for (i in 1:num_runs) {
  samples[i] <- do_run()
}

print(var(samples[samples > 4]))
\end{minted}


Hint, you should know that in P.21 (R18) the string \verb|samples > 4| collects only the samples with value larger than 4.

\begin{solution}
Here is an EXPLANATION:
  \begin{enumerate}
  \item T is a simulated service time of a job without interruptions.
  \item N is is the simulated number of failures that occur during the service of the job
  \item R is then a vector of simulated durations of each of the interruptions
  \item Hence, S is the total time the job spends at the server
  \item We sample the total service time a number of times, and count how often (mean or variance) this exceeds a threshold.
  \end{enumerate}
The points earned depends on how clear you explanation is.


Quite a few of you don't seem to understand what it means to \emph{explain} something. Here are some answers that I saw that certainly don't explain what the code does:
  \begin{enumerate}
  \item `The code does what's stated in the exercise.'.
    What's the explanation here?
    The question is also not: do you understood what the code does?
  \item  `T is uniform rv, N is a Poisson rv, R is a also a uniform rv. We repeat this a few times.' Like this you just read the code, but I know you can read, so this type of answer is quite useless.
  \item `We collect  a subset of the samples and print that.' This answer is just a repition of the hint.
  \end{enumerate}

  So, next time, when you are asked to explain something, explain it.
  In your professional carreer, if you read out loud to your customer what s/he can read for him/herself, you'll be fired pretty soon.
\end{solution}

\end{exercise}
