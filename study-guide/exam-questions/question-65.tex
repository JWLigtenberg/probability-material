\section{Question}

Consider the following code:

\begin{minted}{python}
import numpy as np
from scipy.stats import expon
np.random.seed(42)

n = 100
N = 1000

X = expon(scale = 1/2).rvs([N, n])
Y = X.mean(axis = 1)

mu = 1/2
sigma = 1/2
Z = np.sqrt(n) * (Y - mu)/sigma

print((Z ** 37).mean())
\end{minted}

\begin{minted}{R}
set.seed(42)

n <- 100
N <- 1000

X <- matrix(rexp(N * n, rate = 2), nrow = N, ncol = n)
Y <- rowMeans(X)

mu <- 1/2
sigma <- 1/2
Z <- sqrt(n) * (Y - mu)/sigma

print(mean(Z^37))
\end{minted}

\vspace*{20pt}


\begin{exercise}[1.5]
\texttt{Y} is a vector of i.i.d. random variables.

\begin{itemize}
\item[(i)] What is the length $\ell$ of \texttt{Y}?
\item[(ii)] Each element of \texttt{Y} is a mean of $k$ i.i.d. $\Exp{\lambda}$ r.v.s.
What are $k$ and $\lambda$?  
\item[(iii)] What are the (theoretical) expectation and variance of an element of \texttt{Y}? 
\end{itemize} 

\begin{solution}
The length of \texttt{Y} is $N = 1000$;  Each element of \texttt{Y} is a mean of $k = n = 100$ i.i.d. $\Exp{2}$ r.v.s.
The expectation is $\frac1\lambda = \frac12$ and the variance is $\frac1{k \lambda^2} = \frac1{400} = 0.0025$. \\ \\
Grading scheme:
\begin{itemize}
\item 0.5 for getting both the length $N=1000$ and  $k=100$ correct (no partial credit);
\item 0.5 for $\lambda = 2$, expectation $\frac12$ and the factor $\frac14$ in the variance (no partial credit);
\item 0.5 for the factor $\frac1{k}$ in the variance.
\end{itemize}
\end{solution}
\end{exercise}

\vspace*{20pt}


Recall that each element of  \texttt{Y} is the mean of $k$ i.i.d. $\Exp{\lambda}$ r.v.s.

\begin{exercise}[1]
What is the exact distribution of an element of \texttt{Y}? 

Give its name and its parameters, and explain the answer.
\begin{solution}
The sum of $k$ i.i.d.  $\Exp{\lambda}$ r.v.s. has the $\Gamm{k, \lambda}$ distribution.
Hence, the mean of $k$ i.i.d.  $\Exp{\lambda}$ r.v.s. has the $\Gamm{k, k\lambda}$ distribution.
In this exercise, $k=100$ and $k\lambda = 200$. \\ 
Grading scheme:
\begin{itemize}
\item 0.5 for Gamma with first parameter $k$
\item 0.5 for the second parameter
\end{itemize}
\end{solution}
\end{exercise}


\vspace*{30pt}



Let $(Y_1, \ldots, Y_{\ell})$ be the elements of \texttt{Y} and let $(Z_1, \ldots, Z_{\ell})$ be the elements of  \texttt{Z}. Recall that each $Z_i$ depends on $k$ because $Y_i$ is the mean of $k$ i.i.d. $\Exp{\lambda}$ r.v.s. Let $T$ be the random variable to which $Z_1$ converges in the limit  $k \to \infty$.

\begin{exercise}[1]
What is the distribution of $T$, and why? 

Hence, what is an approximate distribution of an element of \texttt{Y} (e.g. $Y_1$) and why?
\begin{solution}
By the CLT, $Z_1 \;\dot{\sim}\; \Norm{0,1}$. Hence,  $Y_1 = \mu + \sigma/\sqrt{n}Z_1 \;\dot{\sim}\; \Norm{0.5,0.25/n}$. \\ 
Grading scheme:
\begin{itemize}
\item 0.5 for mentioning CLT and the distribution of $Z_1$;
\item 0.5 for the approximate distribution of $Y_1$.
\end{itemize}
\end{solution}
\end{exercise}


\vspace*{20pt}



\begin{exercise}[0.5]
In the last line of the code, we compute $S = \frac1{\ell}\sum_{i=1}^{\ell} Z_i^{37}$.

 If $k \to \infty$ (for fixed $\ell$, e.g. $\ell=3$), does $S$ converge to a constant? 

You do not have to give formal proofs in this subquestion, but you should give clear explanations using the appropriate theorem(s). 
\begin{solution}
In the limit $k \to \infty$, each $Z_i$ has the standard normal distribution by CLT, but the sum of $\ell$ powers of normal distributions has a non-trivial CDF. \\
\begin{itemize}
\item 0.5 for explaining that $S$ does not converge to a constant; no points for just CLT (unless previous question was not answered).
\end{itemize}
\end{solution}
\end{exercise}



\vspace*{20pt}



\begin{exercise}[1]
In the last line of the code, we compute $S = \frac1{\ell}\sum_{i=1}^{\ell} Z_i^{37}$.

If $\ell \to \infty$ (for fixed $k$), does $S$ converge to a constant? 
If so, does it converge to $\E{T^{37}}$?

You do not have to give formal proofs in this subquestion, but you should give clear explanations using the appropriate theorem(s). 
\begin{solution}
By LLN, $S$ does converge to a constant as  $\ell \to \infty$, however, it converges to $\E{Z_1^{37}}$ for that fixed value of $k$. By symmetry, we have  $\E{Z_1^{37}} = 0$. However, the gamma distribution is right-skewed, which implies $\E{T^{37}} > 0$. Hence, it does not converge to $\E{T^{37}}$. \\
Grading scheme:
\begin{itemize}
\item 0.5 for concluding that $S$ converges to a constant using LLN.
\item 0.5 for explaining why the constant is not equal to $\E{T^{37}}$.
\end{itemize}
\end{solution}
\end{exercise}

