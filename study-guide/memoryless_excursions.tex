% arara: pdflatex: { shell: yes }
% arara: pythontex: {verbose: yes, rerun: modified }
% arara: pdflatex: { shell: yes }
% arara: clean: {extensions: [blg, fls, log, out, pytxcode, rel] }

\documentclass[study-guide-sol]{subfiles}
\externaldocument{study-guide}

\opt{solutionfiles,check}{
\Opensolutionfile{hint}
\Opensolutionfile{ans}
}

\begin{document}

%\refcheckxrdoc[a]{memoryless_excursions}

\section{Memoryless exercursions: A confusing problem with memoryless rvs}
\label{sec:memoryl-exerc-conf}
BY give a quick argument to compute $\E M$ and $\E L$ where
$M=\max\{X,Y\}$  and $L=\min\{X,Y\}$ are the maximum and minimum of two iid exponential rvs $X$ and $Y$.
Since  $X$ and $Y$  have the same distribution,
\begin{equation*}
\E L + \E M = \E{L+M}=\E{X+Y} = 2\E X.
\end{equation*}
Therefore,
\begin{equation}
  \label{eq:6}
\E M = 2\E X - \E L.
\end{equation}
Next, by the fact that  $X$ and $Y$ are memoryless,
\begin{equation}
  \label{eq:7}
\E M = \E L  + \E X.
\end{equation}
An interpretation can help to see this. There are two machines, each working on a job in parallel.
Let $X$ and $Y$ be the production times at either machine.
The time the first job finishes is evidently $L=\min\{X, Y\}$.
Then, \emph{due to memorylessness}, the service time of the remaining job `restarts'; this takes an expected time $\E X$ to complete.
Adding these two equations and noting that $\E L$ cancels we get $2\E M = 3 \E X $, hence:
\begin{align}
\E M &= \frac 3 2 \E X, & \E L &= \E M - \E X = \frac 1 2 \E X. \label{eq:35}
\end{align}

This argument seems general enough, so it must hold for discrete memoryless rvs too, i.e., when $X, Y \sim \Geo{p}$. But that is not the case: it is only true when $X, Y \sim \Exp{\lambda}$ and independent. To see what is wrong I tried as many different approaches to this problem I could think of, which  resulted in this text.\footnote{Part of this material was born out of annoyance.  The  book uses one of those typical probability arguments:  slick, half complete, and wrong as soon as one tries it in other situations. In other words, the type of argumnent beginner books should stay clear of. I admit that I was quite irritated about the argument offered by the book.} In~\cref{sec:set-1} we'll derive~\cref{eq:6}  for geometric rvs in multiple  different ways.
Hence, the culprit must be~\cref{eq:7}. Then, in \cref{sec:exerc-expon-distr} we'll show that both equations \emph{are true} for exponenial rvs.
Finally, in~\cref{sec:solution} we find a formula that is similar to $\E M = \E L + \E X$ but that holds for both types of memoryless rvs, whether they are discrete or continuous.

\new{The analysis of} the above problem illustrates many general and useful probability concepts such as  joint CDF, joint PMF/PDF, the fundamental bridge, integration over 2D areas, 2D LOTUS, conditional PMF/PDF, MGFs, and the change of variables formula. It pays off to do the exercises yourself and then study the hinta and solutions carefully.

\new{You'll notice}, hopefully, that I use many different methods to the same problem, and that I take pains to see how the answers of these methods relate. There are at least two reasons for this. Often, a problem can be solved in multiple ways, and one method is not necessarily better than another; better yet,  different methods may augment the understanding of the problem. The second reason is that it is easy to make a mistake in probability. If  different methods give the same answer, the probability of having made a mistake becomes smaller.

\subsection{Discrete memoryless rvs} \label{sec:set-1}

Before  embarking on a problem, it often helps to  refresh our memory. This is what we do first.
Let $X$ be $\sim \Geo{p}$ and write $q = 1-p$.

\begin{exercise}
What is the domain of $X$?
\begin{solution}
Of course $X \in \{0, 1, 2, \ldots\}$.
\end{solution}
\end{exercise}

With some fun tricks with recursions it is possible to quickly derive the most important expressions for geometric rvs:
\begin{align*}
\P{X>0} &= \P{\textrm{failure}} = q\\
\P{X>j} &= q \P{X>j-1} \implies \P{X>j}=q^{j}\P{X>0} = q^{j+1}.\\
\P{X\geq j} &=  \P{X>j-1} = q^{j}.\\
\P{X=j} &= \P{X>j-1} - \P{X>j} = q^{j}-q^{j+1} = (1-q)q^{j} = p q^{j}.\\
\E X &= p\cdot 0 + q(1+\E X) \implies \E X = q/(1-q) = q/p.
\end{align*}
Mind that, even though this is neat, it only work for geometric rvs.

\begin{exercise}
Explain the above.
\begin{solution}
For $X>0$, the first outcome should be a failure. Then, for $j$ failures, we need to fail $j-1$ times and then once more. For $\E X$, if there is a success, we don't need another another experiment. However, in case of a fail, we need another experiment, and we start again. Thus, $\E X = q(1+\E X) \implies (1-q)\E X = q.$
\end{solution}
\end{exercise}

Clearly, such tricks are nice and quick, but they are not general. We should also practice with the general method.

\begin{exercise}
Simplify $\P{X>j}= \sum_{i=j+1}^{\infty} \P{X=i}$ to see that this is equal to $q^{j+1}$.
Realize that from this, $\P{X\geq j}  = \P{X>j-1} = q^{j}$.
\begin{solution}
With the regular method:
  \begin{align*}
\P{X>j}
&= \sum_{i=j+1}^{\infty} \P{X=i} \\
&= p\sum_{i=j+1}^{\infty} q^{i} \\
&= p\sum_{i=0}^{\infty} q^{j+1+i} \\
&= pq^{j+1}\sum_{i=0}^{\infty} q^{i} \\
&= pq^{j+1}\frac 1 {1-q} =
p q^{j+1} \frac 1 p = q^{j+1}.
  \end{align*}
\end{solution}
\end{exercise}


\begin{exercise}\label{ex:12}
Use indicator variables show that
  \begin{equation*}
\E X
= \sum_{i=0}^{\infty} i\P{X=i}
=  \sum_{i=0}^{\infty} \sum_{j=0}^{\infty} \1{j<i} \P{X=i} =p/q.
\end{equation*}
\begin{solution}
  \begin{align*}
\E X
&= \sum_{i=0}^{\infty} i\P{X=i} \\
&=  \sum_{i=0}^{\infty} \sum_{j=0}^{\infty} \1{j<i} \P{X=i} \\
&=  \sum_{j=0}^{\infty} \sum_{i=0}^{\infty} \1{j<i} \P{X=i} \\
&=  \sum_{j=0}^{\infty} \sum_{i=j+1}^{\infty}  \P{X=i} \\
&=  \sum_{j=0}^{\infty}  \P{X>j} \\
&=  \sum_{j=0}^{\infty}  q^{j+1}\\
&=  q \sum_{j=0}^{\infty}  q^{j}\\
&=  q/(1-q) = q/p.
  \end{align*}
\end{solution}
\end{exercise}

\begin{exercise}
Look up the definition of a memoryless rv, and check that $X$ is memoryless.
\begin{solution}
  \begin{align*}
\P{X\geq n+m\given X\geq m}
&= \frac{ \P{X\geq n+m,  X\geq m}}{\P{X\geq m}} \\
&= \frac{ \P{X\geq n+m}}{\P{X\geq m}} \\
&= \frac{ q^{n+m}}{q^m} \\
&= q^n= \P{X\geq n}.
  \end{align*}
\end{solution}
\end{exercise}

\new{With this refresher,} we can derive some useful properties of the minimum $L=\min\{X, Y\}$, where $Y\sim \Geo{p}$ and independent of $X$. For this we use the fundamental bridge and 2D LOTUS, which in general read like
  \begin{equation*} %    \label{eq:60}
\P{g(X,Y}\in A\} = \E{\1{g(X, Y) \in A}} = \sum_{i}\sum_j \1{g(i, j)\in A} \P{X=i, Y=j}.
  \end{equation*}


\begin{exercise}
What is the domain of $L$? Then, show that
\begin{equation*}
\P{L\geq i}=q^{2i} \implies L\sim\Geo{1-q^{2}}.
\end{equation*}
\begin{hint}
For 2D LOTUS, take $g(i,j) = \min\{i, j\}$.
\end{hint}
\begin{solution}
  \begin{align*}
\P{L\geq k}
&= \sum_{i}\sum_{j} \1{\min\{i,j\}\geq k} \P{X=i, Y=j}\\
&= \sum_{i\geq k} \sum_{j\geq k} \P{X=i}\P{Y=j}\\
&=  \P{X\geq k}\P{Y\geq k} = q^{k} q^{k} = q^{2k}.
  \end{align*}
$\P{L>i}$ has the same form as $\P{X>i}$, but now with $q^{2i}$ rather than $q^{i}$.
\end{solution}
\end{exercise}



\begin{exercise}\label{ex:3}
Show that
\begin{equation*}
\E L=q^{2}/(1-q^{2}).
\end{equation*}
\begin{hint}
$X\sim \Geo{1-q} \implies \E X = q/(1-q)$. Now use that $L\sim\Geo{1-q^{2}}$.
\end{hint}
\begin{solution}
  Immediate from the hint and~\cref{ex:12}.
\end{solution}
\end{exercise}

\begin{exercise}
Show that
\begin{equation}   \label{eq:39}
\E L + \E X = \frac q{1-q}\frac{1+2q}{1+q}.
\end{equation}
\begin{hint}
Use that $1-q^2=(1-q)(1+q)$.
\end{hint}
\begin{solution}
  \begin{align*}
\E L     + \E X
&= \frac{q^{2}}{1-q^2} + \frac{q}{1-q} \\
&= \frac q{1-q}\left(\frac{q}{1+q} + 1\right)\\
&= \frac q{1-q} \frac{1+2q}{1+q}
  \end{align*}
\end{solution}
\end{exercise}

\new{Now we can} combine these facts with the properties of the maximum $M=\max\{X, Y\}$.

\begin{exercise}
Show that
\begin{equation*}  \label{eq:10}
2\E X  -\E L = \frac q{1-q} \frac{2+q}{1+q}.
\end{equation*}
\begin{solution}
  \begin{align*}
2\E X - \E L
&= 2\frac q{1-q}  - \frac{q^2}{1-q^2}\\
&= \frac q{1-q}\left(2  - \frac{q}{1+q}\right)\\
&= \frac q{1-q}\left(\frac{2+2q}{1+q}  - \frac{q}{1+q}\right)\\
&= \frac q{1-q}\frac{2+q}{1+q}.
  \end{align*}
\end{solution}
\end{exercise}

Clearly, unless $q=0$, $\E L + \E X \neq 2\E X - \E L$, hence, $\E M$ can only be one of the two and ~\cref{eq:6} and~\cref{eq:7} cannot be both true.

To convince ourselves that~\cref{eq:10}, hence~\cref{eq:6}, is indeed true, we pursue  three ideas.

\new{Here is the first} idea.

\begin{exercise}\label{ex:1}
Show for the PMF of $M$ that
\begin{align*}
p_M(k) &= \P{M=k} = 2 pq^{k}(1-q^{k}) + p^2q^{2k}.
\end{align*}
\begin{hint}
  Use 2D LOTUS on $g(x,y) = \1{\max\{x, y\} = k}$.
\end{hint}
\begin{solution}
  \begin{align*}
\P{M=k}
&= \P{\max\{X, Y\} = k} \\
&=p^{2}\sum_{ij}\1{\max\{i,j\} =k} q^i q^j\\
&=2 p^{2}\sum_{ij}\1{i=k}\1{j < k} q^i q^j + p^{2}\sum_{ij}\1{i=j=k} q^{i} q^j \\
&=2 p^{2}q^{k}\sum_{j < k} q^j + p^{2}q^{2k}\\
&=2 p^{2}q^{k}\frac{1-q^{k}}{1-q} +  p^{2}q^{2k}\\
  \end{align*}
\end{solution}
\end{exercise}

\begin{exercise}
With the previous exercise, show now that $p_M(k) =2 \P{X=k} - \P{L=k}$.
\begin{solution}
It's just algebra
  \begin{align*}
\P{M=k}
&=2 p^{2}q^{k}\frac{1-q^{k}}{1-q} +  p^{2}q^{2k}\\
&=2 pq^{k}(1-q^{k}) + p^{2}q^{2k}\\
&=2 pq^{k} +(p^{2}-2p) q^{2k}\\
&=2 \P{X=k} - \P{L=k},
  \end{align*}
where I use that $p^{2}-2p = p(p-2) = (1-q)(1-q-2)=-(1-q)(1+q)=-(1-q^{2})$.
\end{solution}
\end{exercise}

\begin{exercise}
Finally, show that $\E M = 2 \E X  - \E L$.
\begin{hint}
\end{hint}
\begin{solution}
  \begin{align*}
\E M
&= \sum_k k \P{M=k}     \\
&=  \sum_{k}k (2\P{X=k} - \P{L=k}) \\
&= 2 \E X  - \E L.
  \end{align*}
\end{solution}
\end{exercise}

\new{The second idea.}

\begin{exercise}
First show that  $\P{M \leq k} = (1-q^{k+1})^{2}.$
\begin{solution}
  \begin{align*}
\P{M\leq k}
&= \P{X\leq k, Y\leq k} \\
&= \P{X\leq k}\P{Y\leq k} \\
&= (1-\P{X> k})(1-\P{Y> k}) \\
&= (1-q^{k+1})^{2}.
  \end{align*}
\end{solution}
\end{exercise}

\begin{exercise}
Simplify $\P{M=k} = \P{M\leq k} - \P{M\leq k-1}$ to see that $p_M(k) = 2\P{X=k} - \P{L=k}$.
\begin{solution}
  \begin{align*}
\P{M=k} &= \P{M\leq k} - \P{M\leq k-1}\\
&= 1-2 q^{k+1} + q^{2k+2} - (1 - 2q^{k} + q^{2k})\\
&= 2 q^{k}(1-q) + q^{2k}(q^{2}-1) \\
&= 2 \P{X=k}  - q^{2k}(1-q^{2}) \\
&= 2 \P{X=k}  - \P{L=k}.
  \end{align*}
\end{solution}
\end{exercise}


\new{And here is } the third idea.

\begin{exercise}\label{ex:2}
Explain that
\begin{equation*}
\P{L=i, M=k} = 2p^{2}q^{i+k}\1{k>i}+ p^{2}q^{2i}\1{i=k}).
\end{equation*}
\begin{solution}
  \begin{align*}
\P{L=i, M=k}
&= 2\P{X=i, Y=k}\1{k>i} + \P{X=Y=i}\1{i=k} \\
&= 2p^{2}q^{i+k}\1{k>i}+ p^{2}q^{2i}\1{i=k}.
  \end{align*}
\end{solution}
\end{exercise}


\begin{exercise}
Use~\cref{ex:2} and marginalization to compute the marginal PMF $\P{M=k}$.
\begin{hint}
Marginalize out $L$.
\end{hint}
\begin{solution}
  \begin{align*}
\P{M=k}
&= \sum_{i} \P{L=i, M=k} \\
 &= \sum_{i} (2p^{2}q^{i+k}\1{k>i}+ p^{2}q^{2i}\1{i=k}) \\
 &= 2p^2q^k\sum_{i=0}^{k-1} q^{i}+ p^{2}q^{2k} \\
 &= 2pq^k (1-q^{k})+ p^{2}q^{2k} \\
 &= 2pq^k + (p^{2}-2p) q^{2k},
  \end{align*}
\end{solution}
\end{exercise}

\begin{exercise}
Use~\cref{ex:2} to compute $\P{L=i}$.
\begin{hint}
Marginalize out $M$.
\end{hint}
\begin{solution}
  \begin{align*}
\P{L=i}
&= \sum_{k} \P{L=i, M=k} \\
 &= \sum_{k} (2p^{2}q^{i+k}\1{k>i}+ p^{2}q^{2i}\1{i=k}) \\
 &= 2p^2q^i\sum_{k=i+1}^{\infty} q^{k}+ p^{2}q^{2i} \\
 &= 2p^{2}q^{2i+1} \sum_{k=0}^{\infty}q^{k}+ p^{2}q^{2i} \\
 &= 2pq^{2i+1} + p^{2} q^{2i}\\
 &= pq^{2i}(2q+p)\\
 &= (1-q)q^{2i}(q + 1), \quad p+q=1,\\
 &= (1-q^{2})q^{2i}.
  \end{align*}
\end{solution}
\end{exercise}

\new{In conclusion,} we verified the correctness of $\E M = 2\E X - \E L$ in three different, and useful ways. Let us now focus on exponential rvs rather than geometric rvs.

\subsection{Continuous memoryless rvs}
\label{sec:exerc-expon-distr}

In this section we analyze the correctness of~\cref{eq:6} and \cref{eq:7} for continuous memoryless rvs, i.e., exponentially distributed rvs.
I decided to analyze this in as much detail as I could think of, hoping that this would provide me with a lead to see how to generalize the equation $\E M = \E L + \E X$ such that it covers also the case with  geometric rvs.



\new{First we need} to recall some basic facts about the exponential distribution.

\begin{exercise}
Show that $X$ is memoryless.
\begin{solution}
With $t \geq s$,
\begin{equation*}
\P{X>t|X>s} = \frac{\P{X>t}}{\P{X>s}} = e^{-\lambda t} e^{\lambda s} = e^{-\lambda(t-s)} = \P{X>t - s}.
\end{equation*}

\end{solution}
\end{exercise}


\begin{exercise}
Show that $\E X = 1/\lambda$.
\begin{hint}
  Use partial integration.
\end{hint}
\begin{solution}
It is essential that  you know both methods to solve this integral.
  \begin{align*}
\E X
&= \int_{0}^{\infty}  x f_{X}(x) \d x  \\
&= \int_{0}^{\infty}  x \lambda e^{-\lambda x} \d x  \\
&= - xe^{-\lambda x} \biggr|_{0}^{\infty} + \int_{0}^{\infty}  e^{-\lambda x} \d x  \\
&=  - \frac 1 \lambda e^{-\lambda x} \biggr|_{0}^{\infty} = \frac 1 \lambda.
  \end{align*}
Substitution is also a very important technique to solve such integrals. Here we go again:
  \begin{align*}
\E X
&= \int_{0}^{\infty}  x f_{X}(x) \d x  \\
&= \int_{0}^{\infty}  x \lambda e^{-\lambda x} \d x  \\
&= \frac 1 \lambda \int_{0}^{\infty}  u  e^{- u} \d u,
  \end{align*}
  by the substitution $u=\lambda x \implies \d u = \d (\lambda x) \implies \d u = \lambda \d x \implies \d x = \d u / \lambda$.
  With partial integration (do it!), the integral evaluates to $1$.
\end{solution}
\end{exercise}



\new{Now we can} shift our attention to the rvs $L$ and $M$.


\begin{exercise}
  Show that $F_L(x) = 1-e^{-2\lambda x}$.
\begin{hint}
Using independence and the specific property of the r.v. $L$ that $\{L>x\} \iff \{X>x, Y>x\}$:
\end{hint}
\begin{solution}
With the hint,
 \begin{align*}
G_L(x) = \P{L>x}  &= \P{X>x, Y> x} = G_X(x)^{2} = e^{-2\lambda x}.
  \end{align*}
The result follows since  $F_{L}(x) = 1-G_{L}(x)$.
\end{solution}
\end{exercise}

Clearly, this implies that  $L\sim \Exp{2\lambda}$ and $\E L = 1/(2\lambda) = \E X /2$. Hence, we see that~\cref{eq:35} holds now.
Moreover, with the same trick we see  that the distribution function $F_{M}$ for the maximum $M$ is given by
\begin{equation*}
F_{M}(v) = \P{M\leq v}   = \P{X \leq v, Y \leq v} = (F_{X}(v))^{2} = (1-e^{-\lambda v})^{2}.
\end{equation*}


Of course this is a nice trick, but it is not a method that allows us to compute the distribution for more general functions of $X$ and $Y$.
For more general cases, we have to use the fundamental bridge and LOTUS, that is,
for any set\footnote{If you like maths, you should be quite a bit more careful about what type of set $A$ is acceptable. Here such matters are of no importance.} $A$ in the domain of $X \times Y$
  \begin{equation*}
    \begin{split}
\P{g(X, Y) \in A}
&= \E{\1{g(X,Y)\in A}}
= \iint \1{g(x,y) \in A} f_{X,Y}(x,y) \d x \d y \\
&= \iint_{g^{-1}(A)} f_{X,Y}(x, y) \d x \d y
= \iint_{\{(x,y): g(x,y) \in A\}} f_{X,Y}(x,y) \d x \d y.
    \end{split}
  \end{equation*}
The jpoint CDF $F_{X,Y}$ then follows because $F_{X,Y}(x,y) = \P{X\leq x, Y\leq y} = \E{\1{X\leq x, Y\leq y}}$.
A warning is in place: conceptually this approach is easy, but doing  the integration can be very challenging (or impossible).  However, this expression is very important as this is the preferred way to compute distributions by numerical methods and simulation.



\begin{exercise}
Use the fundamental bridge to rederive the above expression for $F_M(v)$.
\begin{solution}
\begin{align*}
\P{M\leq v}
&= \E{\1{M\leq v}} \\
&=\int_0^{\infty} \int_0^{\infty} \1{x\leq v, y\leq v} f_{XY}(x,y) \d x \d y \\
&= \int_0^{\infty} \int_0^{\infty} \1{x\leq v, y\leq v} f_{X}(x) f_{Y}(y) \d x \d y \\
&= \int_0^{v} f_{X}(x) \d x \int_0^{v}  f_{Y}(y)  \d y \\
&= F_{X}(v) F_Y(v) = (F_X(v))^2.
\end{align*}
\end{solution}
\end{exercise}


\begin{exercise}\label{ex:6}
Show that the density of $M$ has the form $f_{M}(v)=\partial_{v} F_M(v) = 2(1-e^{-\lambda v}) \lambda e^{-\lambda v}$.\footnote{We write $\partial_{v}$ as a shorthand for $\d/\d v$ in the 1D case, and for $\partial/\partial_{v}$ the partial derivative in the 2D case.}
\begin{solution}
\begin{equation*}
  f_M(v) = F_M'(v) = 2 F_X(v) f_X(v) = 2(1-e^{-\lambda v}) \lambda e^{- \lambda v}.
\end{equation*}
\end{solution}
\end{exercise}

\begin{exercise}
Use  the density $f_{M}$ to show  that $\E M = 2\E X - \E L$.
\begin{solution}
\begin{align*}
%  \label{eq:27}
\E M
&= \int_{0}^{\infty} v f_M(v) \d v = \\
&= 2 \lambda \int_{0}^{\infty} v (1-e^{-\lambda v}) e^{-\lambda v} \d v = \\
&= 2 \lambda \int_{0}^{\infty} v e^{-\lambda v} \d v -  2 \lambda \int_{0}^{\infty} v e^{-2 \lambda v} \d v \\
&= 2 \E X - \E L,
\end{align*}
where the last equality follows from the previous exercises.
\end{solution}
\end{exercise}

Recalling that we already obtained  $\E L = \E X/ 2$,  we see that $\E M = 2 \E X - \E L = 3\E X/2$, which settles the truth of ~\cref{eq:35}.



\new{We can also} compute the densities $f_{M}(y)$ (and $f_{L}(x)$ by marginalizing the joint density $f_{L,M}(x,y)$.
However, for this, we first need the joint distribution $F_{L,M}$, and then we can get $f_{L,M}$ by differentiation, i.e., $f_{X,Y} = \partial_{x}\partial_y F_{X,Y}$. Let us try this approach too.


\begin{exercise}
Use the fundamental bridge to show that for $u\leq v$,
\begin{equation*}
  F_{L,M}(u,v) = \P{L\leq u, M\leq v} = 2\int_0^u (F_Y(v)- F_Y(x)) f_X(x) \d x.
\end{equation*}
\begin{solution}
First the joint distribution. With $u\leq v$,
  \begin{align*}
F_{L,M}(u,v) &= \P{L\leq u, M \leq v} \\
&= 2\iint \1{x \leq u, y\leq v, x\leq y} f_{X,Y}(x,y)\d x \d y \\
&= 2\int_0^u \int_x^v f_Y(y) \d y f_X(x) \d x & \text{independence} \\
&= 2\int_0^u (F_Y(v)- F_Y(x)) f_X(x) \d x.
  \end{align*}
\end{solution}
\end{exercise}

\begin{exercise}\label{ex:25}
Take partial derivatives to show that
\begin{equation*}
f_{L,M}(u,v) = 2f_X(u) f_Y(v)\1{u\leq v}.
\end{equation*}
\begin{solution}
Taking partial derivatives,
\begin{align*}
f_{L,M}(u,v)
&=\partial_v\partial_{u}F_{L,M}(u,v) \\
&=2 \partial_v\partial_{u} \int_0^u (F_Y(v)- F_Y(x)) f_X(x) \d x  \\
&=2 \partial_v \left\{(F_Y(v)- F_Y(u)) f_X(u) \right \}  \\
&=2 f_X(u)\partial_v F_Y(v)  \\
&=2 f_X(u)f_Y(v).
\end{align*}
\end{solution}
\end{exercise}


\begin{exercise}
In~\cref{ex:25} marginalize out $L$ to find $f_M$, and marginalize out $M$ to find $f_L$.
\begin{solution}
\begin{align*}
  f_M(v) &=  \int_0^{\infty} f_{L,M}(u,v) \d u \\
&= 2 \int_0^{v} f_{X}(u) f_Y(v) \d u \\
&= 2 f_Y(v) \int_0^{v} f_{X}(u) \d u  \\
&= 2f_Y(v) F_X(v), \\
f_L(u) &=  \int_0^{\infty} f_{L,M}(u,v) \d v \\
&= 2 f_{X}(u) \int_u^{\infty} f_Y(v) \d v \\
&= 2 f_{X}(u) G_{Y}(u).
\end{align*}
\end{solution}
\end{exercise}


\new{We did a} number of checks for the case $X, Y, \iid, \sim\Exp{\lambda}$,
but I have a second way to check the  consistency of our results.
For this I use the idea that the geometric distribution is the discrete analog of the exponential distribution.
Now we study how this works, and that by taking proper limits we can obtain the results for the continuous setting from the discrete setting.

First, let's try to obtain an intuitive understanding of how $X\sim \Geo{\lambda/n}$ approaches  $Y\sim\Exp{\lambda}$ as $n\to\infty$.
For this, divide the interval $[0, \infty)$ into many small intervals of length $1/n$.
Let  $X\sim \Geo{\lambda/n}$ for some $\lambda>0$ and $n\gg 0$. Then take  some $x\geq 0$ and  let $i$ be such that $x\in[i/n, (i+1)/n)$.

\begin{exercise}
Show that
\begin{equation}\label{eq:53}
\P{X/n \approx x} \approx \frac{\lambda} n \left(1-\frac \lambda n\right)^{xn}.
\end{equation}
\begin{solution}
First,
\begin{align*}
\P{X/n \approx x} &= \P{X/n \in [i/n, (i+1)/n]} = \P{X \in [i, i+1]} = p q^{i} \\
&\approx p q^{n x} = \frac{\lambda} n \left(1-\frac \lambda n\right)^{xn},
\end{align*}
since $p=\lambda/n$, $q=1-p=1-\lambda/n$, and $i=nx$.
\end{solution}
\end{exercise}
Next,  introduce the highly intuitive notation\footnote{In your math classes you learned that $\lim_{n\to \infty} 1/n = 0$. Doesn't this definition therefore imply that $\d x = 0$? Well, no, because $\d x$ is not a real number but an infinitesimal. Infinitesimals allow us to consider a quantity that is so small that it cannot be distinguished from 0 within the real numbers.}  $\d x = \lim_{n\to \infty} 1/n$, and use the standard limit\footnote{This is not entirely trivial to prove. If you like mathematics, check the neat proof in Rudin's Principles of mathematical analysis.}  $(1-\lambda/n)^n\to e^{-\lambda}$ as $n\to\infty$ to see that~\cref{eq:53} converges to
\begin{equation*}
\P{X/n \approx x} \to \lambda e^{-\lambda x} \d x = f_{X}(x) \d x,\quad\text{ as } n \to \infty.
\end{equation*}
% In fact, working with infinitesimals such as $\d x$ is one of my favorite methods, for instance to derive densities.
% The intuition is that $\d x$ is some really small number but not zero.
% In a tiny bit more formal words, $0<\d x < 1/n$ for any $n$, but $\d x \neq 0$.
% For details, check out the web; search on `nonstandard calculus'.
% However, while the method is very intuitive, it's easy to mess up. (Power tools are nice, but often only when you know how to handle them and practice a lot.)
% If you don't want to learn this, no problem, it will not be part of the course proper.

If you don't like this trick with $\d x$, here is another method, based on with moment-generating functions.

\begin{exercise}
Derive the moment-generating function $M_{X/n}(s)$ of $X/n$ when $X\sim \Geo{p}$.
Then, let $p = \lambda/n$, and show that $\lim_{n\to\infty}M_{X/n}(s) = M_{Y}(s)$, where $Y \sim \Exp{\lambda}$.
\begin{hint}
  If you recall the Poisson distribution, you know that $e^{\lambda} = \sum_{i=0}^{\infty}\lambda^{i}/i!$.
  In fact, this is precisely Taylor's expansion of $e^{\lambda}$.
\end{hint}
\begin{solution}
  \begin{align*}
M_{X/n}(s)
&= \E{e^{s X/n}} = \sum_{i} e^{s i/n} p q^{i} = p\sum_{i} (qe^{s/n})^{i} = \frac{p}{1-qe^{s/n}}.
  \end{align*}
With $p=\lambda/n$ this becomes
  \begin{align*}
M_{X/n}(s)
&= \frac{\lambda/n}{1-(1-\lambda/n) (1+s/n + 1/n^{2}\times(\cdots))} \\
&= \frac{\lambda/n}{\lambda/n - s/n + 1/n^{2}\times (\cdots)} \\
&= \frac{\lambda}{\lambda - s + 1/n\times(\cdots)} \\
&\to \frac{\lambda}{\lambda - s}, \quad\text{as }  n\to \infty,
  \end{align*}
  where we write $1/n^{2}\times(\cdots)$ for all terms that will disappear when we take the limit $n\to \infty$.
  This is just handy notation to hide details in which we are not interested.
\end{solution}
\end{exercise}

With these limits in place, we can relate the minimum $L=\min\{X, Y\}$ for the discrete and the continuous settings.

\begin{exercise}
Suppose that $X, Y\sim \Geo{\lambda/n}$, then check that $\lim_{n\to\infty}\E{L/n} = 1/2\lambda$.
\begin{hint}
  Use~\cref{ex:3}.
\end{hint}
\begin{solution}
  \begin{align*}
\E{L/n}  &= \frac 1 n \E L = \frac 1 n \frac{q^2}{1-q^{2}} \\
 &= \frac 1 n \frac{(1-\lambda/n)^2}{1-(1-\lambda/n)^{2}} \\
 &= \frac 1 n \frac{1-2 \lambda/n + (\lambda/n)^2}{2 \lambda /n + (\lambda/n)^{2}} \\
 &= \frac{1-2 \lambda/n + (\lambda/n)^2}{2 \lambda + \lambda^{2}/n} \\
&\to \frac 1{2\lambda}.
  \end{align*}
\end{solution}
\end{exercise}

Clearly, $1/2\lambda=\E X/2$ when $X\sim\Exp{\lambda}$.

Here is yet another check on the correctness of $f_M(x)$.
\begin{exercise}
Show that  the PMF $\P{M=k}$ for the discrete $M$ in~\cref{ex:1} converges to $f_M(x)$ of~\cref{ex:6} when $n\to \infty$. Take $k$ suitable.
\begin{solution} Take $p=\lambda/n$,  $q=1-\lambda/n$, and $k \approx x n$, hence $k/n \approx x$. Then,
  \begin{align*}
\P{M/n=k/n}
&= 2 p q^{k/n}(1-q^{k/n}) + p^2q^{2k/n} \\
&= 2 \frac \lambda n \left(1-\frac \lambda n\right)^{k/n}\left(1-\left(1-\frac \lambda n\right)^{k/n}\right) + \frac{\lambda^2}{n^2}\left(1-\frac \lambda n \right)^{2k/n} \\
&\to 2 \lambda \d x e^{-\lambda x} \left(1 - e^{-\lambda x}\right) + \lambda^{2} \d x^{2}e^{-2\lambda}.\label{eq:37}
  \end{align*}
Now observe that the second term, proportional to $\d x^{2}$ can be neglected.
\end{solution}
\end{exercise}


\new{Finally, I have}  a third  way to check the above results, namely by verifying ~\cref{eq:7}, i.e.  $\E{M-L} = \E X$.
For this, we  compute the joint CDF $f_{L, M-L}(x, y)$. With this,  you'll  see directly how to compute $\E{M-L}$.

\begin{exercise}\label{ex:27}
Use the fundamental bridge to obtain
\begin{equation*}
F_{L,M-L}(x,y) = (1- e^{-2\lambda x}) (1-e^{-\lambda y}) =  F_L(x) F_Y(y).
\end{equation*}
\begin{hint}
The idea is not difficult, but the technical details require attention, in particular the limits in the integrations.
\end{hint}
\begin{solution}
\begin{align*}
F_{L, M-L}(x, y)
&=\P{L\leq x, M-L\leq y}\\
&= 2\P{X\leq x, Y-X \leq y, X \leq Y} \\
&= 2\int_0^{\infty} \int_0^{\infty} \1{u\leq x, v-u\leq y, u \leq v} f_{X,Y}(u, v)  \d u \d v\\
&= 2\int_0^{\infty} \int_0^{\infty} \1{u\leq x, v-u\leq y, u\leq v} \lambda^2e^{-\lambda u}e^{-\lambda v}\d u \d v\\
&= 2\int_0^{x} \int_0^{\infty} \1{u\leq v\leq u+y} \lambda^2e^{-\lambda u}e^{-\lambda v}\d v \d u\\
&= 2\int_0^{x} \lambda e^{-\lambda u} \int_u   ^{u + y}  \lambda e^{-\lambda v}\d v \d u\\
&= 2\int_0^{x} \lambda e^{-\lambda u} (-e^{-\lambda v}) \biggr|_u^{u+y}\d u \\
&= 2\int_0^{x} \lambda e^{-\lambda u} (e^{-\lambda u} - e^{-\lambda (u+y)})\d u \\
&= 2\lambda\int_0^{x}  e^{-2 \lambda u}\d u  - 2\lambda \int_{0}^{x} e^{-\lambda (2u+y)}\d u \\
&= 2\lambda\int_0^{x}  e^{-2 \lambda u}\d u  - 2\lambda e^{-\lambda y }\int_{0}^{x} e^{-2 \lambda u}\d u \\
&= (1-e^{-\lambda y}) 2 \lambda \int_0^{x} e^{-2 \lambda u} \d u\\
&= (1-e^{-\lambda y}) ( -e^{-2\lambda u}) \biggr|_0^{x} \\
&= (1-e^{-\lambda y})(1- e^{-2\lambda x}).
\end{align*}
\end{solution}
\end{exercise}

\begin{exercise}
Conclude that $M-L$ and $L$ are independent, and $M-L\sim Y$.
\begin{solution}
As  $F_{L,M-L}(x,y) = F_Y(y) F_L(x)$. So the CDF factors as a function of $x$ only and a function of $y$ only. This implies that $L$ and $M-L$ are independent, and moreover that $F_{M-L}(y) =  F_Y(y)$, so $M-L \sim Y$. We can also see this from the joint PDF:
\begin{equation*}
f_{L, M-L}(x,y) = \partial_x \partial_y (F_Y(y) F_L(x)) = f_Y(y) f_L(x),
\end{equation*}
so the joint PDF (of course) also factors. The independence now follows from BH 7.1.21.
%but it is also okay to conclude it directly from the CDF factoring as a function of $x$ only and a function of $y$ only.
Because $L$ and $M-L$ are independent, the conditional density equals the marginal density:
\begin{equation*}
f_{M-L|L}(y|x) = \frac{f_{L, M-L}(x, y)}{f_L(x)} = \frac{f_Y(y) f_L(x)}{f_L(x)} = f_Y(y).
\end{equation*}
\end{solution}
\end{exercise}

By the  above exercise, we find that $\E{M-L} = \E Y = \E X$, as $X$ and $Y$ are iid.


\new{This makes me} wonder whether $M-L$ and $L$ are also independent for the discrete case, i.e.,  when  $X, Y$ iid and $\sim \Geo{p}$.
Hence, we should check that for \emph{all} $i, j$
\begin{equation}\label{eq:34}
\P{L=i, M-L=j} = \P{L=i} \P{M-L=j}.
\end{equation}


\begin{exercise}
Use~\cref{ex:2} to see that
\begin{equation*}
\P{L=i, M-L=j} = 2p^2q^{2i+j} \1{j\geq 1} + p^2q^{2i}\1{j=0}.
\end{equation*}
\begin{hint}
Realize that $\P{L=i, M-L=j}=  \P{L=i, M=i+j}$. Now fill in the formala of~\cref{ex:2}.
\end{hint}
\end{exercise}

Now for the RHS.
\begin{exercise}
Derive that
\begin{equation*}
\P{M-L=j} = \frac{2p^2q^{j}}{1-q^2}\1{j\geq 1} + \frac{p^2q^{j}}{1-q^2}\1{j=0}.
\end{equation*}
\begin{solution}
Suppose $j>0$ (for $j=0$ the maths is the same). Then,
\begin{align*}
  \P{M-L=j}
&=2\sum_{i=0}^{\infty}\P{X =i, Y=i+j } =
&=2\sum_{i=0}^{\infty}p q^i pq^{i+j} =
&=2p^{2}q^{j}\sum_{i=0}^{\infty}q^{2i} =
&=2p^{2}q^{j}/(1-q^2).
\end{align*}
\end{solution}
\end{exercise}


Recalling that $\P{L=i} = (1-q^2)q^{2i}$, it follows right away from~\cref{eq:34} that $L$  and $M-L$ are independent.
Interestingly, from~\cref{ex:27} we see $M-L\sim Y$ for the continuous case. However, here, for the geometric case, $\P{M-L=j} \neq pq^{j} = \P{Y=j}$.  This explains why $\E M \neq \E L + \E X$ for geometric rvs: we should be more carefull in how to split $M$ in terms of $L$ and $X$.



\new{All in all}, we have checked and double checked all our expressions and limits for the geometric and exponential distribution.
We had success too: the solution of the last exercise provides the key to understand why~\eqref{eq:6} and \eqref{eq:7} are true for exponentially distributed rvs, but not for geometric random variables.
In fact, in the solutions we can see the term corresponding to $X=Y=i$ for $X,Y\sim \Geo{p}$ becomes negligibly small when $n\to 0$.
In other words, $\P{X=Y}>0$  when $X$ and $Y$ are discrete, but $\P{X=Y}=0$ when  $X$ and $Y$ are continuous.
Moreover, by~\cref{ex:27}, $\E M = \E{L + M-L} = \E L + \E{M-L}$, but $\E{M-L} \neq \E X$. So, to resolve our leading problem we should reconsider $\E{M-L}$.


\subsection{The solution}
\label{sec:solution}


Let us now try to repair~\cref{eq:7}, i.e., $\E M = \E L + \E X$,  for the case $X,Y\sim\Geo{p}$. We should be careful about the non-negligible case that $M=L$, so we move, carefully, step by step.

\begin{exercise}
Why is the following true:
\begin{align}
  \label{eq:40}
\E M &= \E L + \E{(M-L) \1{M>L}}  = \E L + 2\E{(Y-X) \1{Y>X}}.
\end{align}
\begin{solution}
Because either $M=L$ or $M>L$. Recall from earlier work that the factor 2 in the second equality follows from the fact that $X,Y$ iid.
\end{solution}
\end{exercise}


\begin{exercise}
Show that
\begin{equation*}\label{eq:41}
2\E{(Y-X) \1{Y>X}} = \frac{2q} {1-q^{2}}.
\end{equation*}
\begin{hint}
  Reread BH.7.2.2. to realize that $\E{M-L} = \E{|X-Y|}$. Relate the latter expectation to the expression in the problem.
\end{hint}
\begin{solution}
  \begin{align*}
\E{(Y-X) \1{Y>X}}
&= p^2\sum_{ij} (j-i)\1{j>i} q^iq^j\\
&= p^2\sum_{i} q^i\sum_{j=i+1}^{\infty}(j-i)q^j\\
&= p^2\sum_{i} q^i q^{i}\sum_{k=1}^{\infty}kq^k, \quad k = j-i\\
&= p\sum_{i}  q^{2i}\E X \\
&= \frac{p}{1-q^{2}}\E X \\
&= \frac{p}{1-q^{2}}\frac q p \\
&= \frac{q}{1-q^{2}}.
  \end{align*}
\end{solution}
\end{exercise}

\begin{exercise}
Combine the above with the expression for $\E L$ of~\cref{ex:3} to obtain \cref{eq:10} for $\E M = 2\E X - \E L$, thereby verifying the correctness of~\cref{eq:40}.
\begin{hint}
\end{hint}
\begin{solution}
\begin{align*}
\E L + 2  \E{(Y-X) \1{Y>X}} = \frac{q^{2}}{1-q^2} +  \frac{2q}{1-q^{2}} = \frac q{1-q}\frac{q+2}{1+q},
\end{align*}
where I use that $1-q^{2}=(1-q)(1+q)$.
\end{solution}
\end{exercise}


While~\cref{eq:40} is correct, I am still not happy with the second part of~\eqref{eq:40} as I find it hard/unintuitive to interpret.
Restarting again from scratch, here is another attempt to rewrite $\E M$ by using $Z\sim \FS{p}$, i.e., $Z$ has the first success distribution with parameter $p$, in other words, $Z \sim X+1$ with $X\sim\Geo{p}$.

\begin{exercise}
Explain that
\begin{equation}
  \label{eq:44}
\E M = \E L + \E{ Z \1{M>L}},
\end{equation}
\begin{solution}
To see why this might be true, I reason like this.
After `seeing' $L$, we want to restart.
Let $Z$ be the time from the restart to $M$.
When $Z\sim \Geo{p}$, it might happen that $Z=0$ (with positive probability $p$).
But if $Z=0$, then $M=L$, and it that case, we should not restart.
Hence, if $Z\sim \Geo{p}$ we are `double counting' when $Z=0$.
By including the condition $M>L$ and by taking $Z\sim\FS{p}$ (so that $Z>0$) I can prevent this.
\end{solution}
\end{exercise}


\begin{exercise}
Show that
\begin{equation*}
\E{Z\1{M>L}} = \frac{2q} {1-q^{2}},
\end{equation*}
i.e., the same as~\cref{eq:41}, hence~\cref{eq:44} is correct.
\begin{hint}
Observe that $Z$   is independent from $X$ and $Y$, hence from $M$ and $L$
\end{hint}
\begin{solution} With the hint:
  \begin{align*}
\E{Z\1{M>L}}  = \E Z \E{\1{M>L}} = \E Z \P{M>L} = \frac 1 p \frac{2pq}{1-q^{2}} = \frac{2q}{1-q^{2}},
  \end{align*}
We know that $\E Z = 1 + \E X = 1 + q/p = 1/p$, while
\begin{align*}
\P{M>L}
&= 1-\P{X=Y} = 1-\sum_{i=0}^{\infty} \P{X=Y=i} \\
&= 1-\frac{p^{2}}{1-q^{2}} = \frac{1 - q^{2} + p^{2}}{1-q^{2}} = \frac{2pq}{1-q^{2}}.
\end{align*}
\end{solution}
\end{exercise}


\new{I am nearly} happy, but I want to see that~\eqref{eq:44}, which is correct for discrete rvs, has also the correct limiting behavior.
\begin{exercise}
Show that $\E{Z/n\1{M>L}} \to 1/\lambda$, which is the expectation of an $\Exp{\lambda}$ rv!
\begin{solution}
By independence,
  \begin{equation*}
\E{Z/n\1{M>L}} = \E{Z/n} \P{M>L}.
  \end{equation*}
Then,
\begin{equation*}
\P{M>L}   = \frac{2pq}{1-q^{2}}
= \frac{2\lambda/n (1-\lambda/n)}{1-(1-\lambda/n)^{2}}
= \frac{2\lambda/n (1-\lambda/n)}{2\lambda/n -\lambda^{2}/n^{2}}
= \frac{2 (1-\lambda/n)}{2 -\lambda/n}  \to 1,
\end{equation*}
and
\begin{equation*}
\E{Z/n} = \frac 1 n \E Z = \frac 1 n \frac 1 p = \frac 1 {n \lambda/ n} = 1/\lambda.
\end{equation*}
\end{solution}
\end{exercise}

Finally I understand why  $\E M = \E L + \E X$ for $X,Y\sim \Exp{\lambda}$ but not for when $X, Y$ are discrete. For discrete rvs, $L$ and $M$ can be equal, while for continuous rvs, this is impossible\footnote{A bit more carefully formulated: the event $\{L=M\}$ has zero probability for continuous rvs.} It took a long time, and a lot of work, to understand how to resolve the confusing problem, but I learned a lot.
In particular, I find~\cref{eq:44} a nice and revealing equation.

Finally, if you like to train with the  tools you learned,  you can try your hand at analyzing the same problem, but now for uniform $X, Y$.


\end{document}


%%% Local Variables:
%%% mode: latex
%%% TeX-master: "erratic_examples"
%%% End:
